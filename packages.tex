\usepackage[utf8]{inputenc}
\usepackage[english]{babel}


\usepackage{tcolorbox}
\usepackage{authblk}  % Lets you add an \affil{} to your title, stating your affiliation {eg. Sigma Mathematics Society}
\usepackage{ragged2e}
\usepackage{csquotes}
\usepackage{pdfpages}

\usepackage[a4paper, total={6.6in, 9in}, textheight=720pt, voffset=35pt, footnotesep=25pt]{geometry} % Adjust margins as necessary

\usepackage{xfrac}
\usepackage{cancel}

\usepackage[inline]{enumitem}

%\usepackage{tgpagella}

\usepackage{blindtext}
\usepackage{lipsum}
\usepackage{verbatim}
\usepackage{hyperref}
\hypersetup{
    citebordercolor = 1 1 1,
    linkbordercolor = 1 1 1,
    filebordercolor = 1 1 1,
    menubordercolor = 1 1 1,
    urlbordercolor = 1 1 1,
    colorlinks  =   true,
    linkcolor   =   blue,
    citecolor   =   magenta,
    urlcolor    =   blue
}

\usepackage{biblatex} % Modify citation format using [style=yourstyle] parameter--eg \usepackage[style=mla-new]{biblatex}
\bibliography{References.bib}
\addbibresource{References.bib}

\usepackage{cancel}
\usepackage{amssymb}
\usepackage{amsmath}
%\usepackage{MnSymbol}
\usepackage{mathrsfs}
\usepackage{mathtools}
% \usepackage{mathabx}

\usepackage{array}
\usepackage{booktabs}
\usepackage{longtable}

\usepackage{graphicx}
\newcommand\sbullet[1][.5]{\mathbin{\vcenter{\hbox{\scalebox{#1}{$\bullet$}}}}}  % Bullet of customisable size
\usepackage{wrapfig}
\usepackage{caption}
\usepackage{subcaption}
\usepackage{tikz}
\usepackage{float}

% Theorem layout

\usepackage{amsthm}

\newtheorem*{theorem*}{Theorem}
\newtheorem{theorem}{Theorem}%[section]
\newtheorem{corollary}[theorem]{Corollary}%[theorem]
\newtheorem{lemma}[theorem]{Lemma}
\newtheorem{claim}[theorem]{Claim}
\newtheorem{conjecture}[theorem]{Conjecture}
\newtheorem{algorithm}[theorem]{Algorithm}
\newtheorem{proposition}[theorem]{Proposition}
\newtheorem{example}[theorem]{Example}
\newenvironment{boxexample}{
    \begin{tcolorbox}\begin{example}
}{
    \end{example}\end{tcolorbox}
}
\newtheorem{problem}{Problem}
\newenvironment{boxproblem}{
    \begin{tcolorbox}[colback=orange!3!white,colframe=orange!70!black]\begin{problem}
}{
    \end{problem}\end{tcolorbox}
}
\newenvironment{boxtheorem}{
    \begin{tcolorbox}[colback=green!3!white,colframe=green!70!black]\begin{theorem}
}{
    \end{theorem}\end{tcolorbox}
}
\newtheorem*{notation}{Notation}
\newenvironment{boxnotation}{
    \begin{tcolorbox}[colback=magenta!3!white,colframe=magenta!70!black]\begin{notation}
}{
    \end{notation}\end{tcolorbox}
}

\theoremstyle{remark}
\newtheorem*{remark}{Remark}
\newtheorem*{solution}{Solution}

\theoremstyle{definition}
\newtheorem{definition}{Definition}[section]

% Structure and Numbering

\usepackage{fancyhdr}
\usepackage{lastpage}

\pagestyle{fancy}
\fancyhf{}

% Tikz
\usepackage{tikz-cd}
\usepackage{tikz-3dplot}
\usetikzlibrary{positioning}
\usetikzlibrary{cd}
\usetikzlibrary{shapes.geometric}
\usepackage{pgfplots}
\usepackage{mathrsfs}
\usetikzlibrary{arrows}