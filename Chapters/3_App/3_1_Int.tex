\section{Integrality}

\subsection{The Algebraic Integers}

First, we recall the notion of integrality over an integral domain.

\begin{boxdefinition}[Integrality]
    Let $R$ be an integral domain, and $S \supset R$ an extension. We say that $s \in S$ is integral over $R$ if one of the following equivalent conditions is satisfied:
    \begin{itemize}
        \item $s$ is a root of a monic polynomial in $R[X]$.
        \item The minimal polynomial of $s$ over $\Frac{R}$ is actually in $R[X]$.
    \end{itemize}
\end{boxdefinition}
We now define what it means for a complex number to be an algebraic integer.
\begin{boxdefinition}[Algebraic Integer]
    We say that a number $\alpha \in \C$ is an algebraic integer if it is integral over $\Z$.
\end{boxdefinition}

\begin{boxnotation}
    Given a conjugacy class $C$ of $G$, we define
    \begin{align}
        \hat{C} := \sum_{g \in C} g
    \end{align}
    to be an element of $\C G$.
\end{boxnotation}

\begin{lemma}
    Let $g \in G$ and let $C = g^G$ % Conjugacy class of g?
    Let $S$ be a simple \CGM. Then, for all $s \in S$, we have an action by scalar multiplication
    \begin{align*}
        \hat{C} \cdot s = \lambda s
    \end{align*}
    where $\displaystyle \lambda = \frac{\abs{C}}{\abs{C_G(g)}} \frac{\chi(g)}{\chi_s(1)} = \abs{C} \frac{\chi_s(g)}{\chi_s(1)}$.
\end{lemma}
\begin{proof}
    Define a function $\phi : S \to S : s \mapsto \hat{C} \cdot s$. Since $\hat{C} \in \Zof{\C G}$, % EXERCISE!
    we have that $\forall x \in G$, $x \cdot \phi(s) = \phi(x \cdot s)$. This makes $\phi$ a \CGM\ homomorphism. Then, by Schur's Lemmas, we know that $\phi = \lambda \id$. This means that... % finish
\end{proof}

\begin{lemma}
    Let $r = \sum_{g \in G} \alpha_g g$ for some $\alpha_g \in \Z$. Suppose that $\exists \lambda, v \in \C G \setminus \set{0}$ such that $rv = \lambda v$. Then, $\lambda$ is an algebraic integer.
\end{lemma}
\begin{proof}
    Let $G = \set{g_1, \ldots, g_n}$. For all $1 \leq i \leq n$,
    \begin{align*}
        r g_i &= \sum_{j=1}^{n} \alpha_{ij} g_j
    \end{align*}
    The key observation here is that $\alpha_{ij} \in \Z$ for all $1 \leq i,j \leq n$. % WHY????
    Then, if $rv = \lambda v$, we have that $\lambda$ is an eigenvalue of the matrix $A := \parenth{\alpha_{ij}}_{1 \leq i,j \leq n} \in \mat{n}{n}{\Z}$. This makes $\lambda$ a root of the characteristic polynomial of $A$ (over $\Z$), which is monic and of degree $n$.
\end{proof}

What we can gather from lemmas $1$ and $2$ is that for any $\chi \in \Irr{G}$ and $g \in G$, the quantity $\lambda = \frac{\abs{G}}{\abs{C_G(g)}} \frac{\chi(g)}{\chi(1)}$ is an algebraic integer. This leads us to the following important connection between algebraic integers and irreducible representations.

\begin{theorem} \label{Ch2:Thm:Char_div_Ord_Grp}
    For all $\chi \in \Irr{G}$, $\chi(1) \vert \abs{G}$.
\end{theorem}
\begin{proof}
    \verb|Check phone|
\end{proof}

\subsection{A Word on Simple Groups}

First, we have an interesting result on $p$-groups.

\begin{lemma} Assume that $\abs{G} = p^n$ for some $p$ prime and $n \in \N$. Then, $\forall \chi \in \Irr{G}$, $\chi(1)$ is a power of $p$.
\end{lemma}
\begin{proof}
    Let $\Irr{G} = \set{\chi_1, \ldots, \chi_k}$, with $\chi_1$ being the trivial character.
    \verb|sorry|
\end{proof}
\begin{corollary}
    If $\abs{G} = p^2$ for some $p$ prime, then $\forall \chi \in \Irr{G}$, $\chi(1) = 1$. In particular, $G$ is abelian.
\end{corollary}
\begin{proof}
    \verb|sorry|
\end{proof}

We now have an interesting result on the characters of simple groups.

\begin{lemma}
    If $G$ is simple, $G$ does not admit an irreducible character of degree $2$.
\end{lemma}
\begin{proof}
    Suppose, for contradiction, that $G$ is simple but admits an irreducible representation $\rho : G \to \GL{2, \C}$. Since $G$ is simple, we must have that $\pker{\rho} = \set{1}$.

    If $G$ is nonabelian, then $G' = G$. This tells us that there are no nontrivial characters of degree $1$. Since the map $\rho \mapsto \pdet{\rho(g)}$ a linear character (ie, has degree $1$), we can conclude that $\pdet{\rho(g)} = 1$ for all $g \in G$. By Theorem \ref{Ch2:Thm:Char_div_Ord_Grp}, we know that $2 \vert \abs{G}$. Then, by Cauchy's Theorem, $\exists g \in G$ such that $g \neq 1$ but $g^2 = 1$. This tells us that $\rho(g) = -\id \in \Zof{G}$, which is absurd, since $G$ is simple. 
\end{proof}
