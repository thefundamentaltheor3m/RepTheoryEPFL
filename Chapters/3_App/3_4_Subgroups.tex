\section{Characters and Subgroups}

In this section, we will explore how we can understand the characters of a group using those of its subgroups and vice-versa. We will begin by discussing the latter, which we will study from the perspective of Restriction. We will then discuss the former, which we will study from the perspective of Induction.

\subsection{Restriction}

For the entirety of this subsection, let $G$ be a finite group and let $H \leq G$. % We are interested in the restriction of this representation to $H$.

\begin{boxdefinition}[Restriction of a Representation]
    Let $\Vp$ be a complex representation of $G$ with associated character $\chi$. The restriction of $\Vp$ to $H$, denoted $V \downarrow H$ or $V_H$, is the set representation $\parenth{V, \rho\vert_H}$. We will denote its associated character $\chi \downarrow H$ or $\chi_H$.
\end{boxdefinition}

\begin{boxnotation}
    To distinguish between inner-products of central functions considered over $H$ and those over $G$, we will denote the former by $\cycl{\cdot, \cdot}_H$. In similar fashion to \eqref{SP:eq:Char_Inner_Product}, we have, for all $\psi_1, \psi_2 \in \Fcof{H, \C}$,
    \begin{align}
        \cycl{\psi_1, \psi_2}_H := \frac{1}{\abs{H}} \sum_{h \in H} \psi_1(h) \overline{\psi_2(h)}
    \end{align}
\end{boxnotation}

\begin{proposition}
    Fix $\psi \in \Irr{H}$. Then, there exists an irreducible character $\chi$ of $G$ such that the inner-product $\cycl{\chi \downarrow H, \psi}_H$ is nonzero. 
\end{proposition}
\begin{proof}
    Let $\Irr{G} = \set{\chi_1, \ldots, \chi_k}$. Consider the regular character $\chi_{\reg}$ of $G$. We then have \verb|(insert reference)| that for all $g \in G$,
    \begin{align*}
        \chi_{\reg} = \sum_{i = 1}^k \chi_i(1) \chi_i
        = \begin{cases}
            \abs{G} & \text{ if } g = 1 \\
            0 & \text{ otherwise}
        \end{cases}
    \end{align*}
    We now show that $\cycl{\chi_{\reg} \downarrow H, \psi}_H$ is nonzero:
    \begin{align*}
        \cycl{\chi_{\reg} \downarrow H, \psi}_H &= \frac{1}{\abs{H}} \sum_{h \in H} \chi_{\reg}(h) \overline{\psi(h)} \\
        &= \frac{1}{\abs{H}} \abs{G} \overline{\psi(1)} \neq 0
    \end{align*}
    Considering the decomposition of $\chi_{\reg}$ as a sum of irreducible characters, we see that one of the summands must have a nonzero inner-product with $\psi$.
\end{proof}

\begin{remark}
    In general, we can compute $\Irr{H}$ from $\Irr{G}$ quite easily if $\brac{G : H}$ is small. However, if $\brac{G : H}$ is large, then this can get difficult.
\end{remark}

\begin{proposition} \label{Ch3:Prop:Char_Rest_Bound_Degree}
    Fix $\chi \in \Irr{G}$, and write $\Irr{H} = \set{\psi_1, \ldots, \psi_r}$. Then, $\exists d_1, \ldots, d_r \in \N$ such that
    \begin{align*}
        \chi \downarrow H = \sum_{i=1}^{r} d_i \psi_i
        \quad\quad
        \text{ and }
        \quad\quad
        \sum_{i=1}^{r} d_i^2 \leq \brac{G : H}
    \end{align*}
    Moreover, the above inequality is an \emph{eq}uality if and only if $\forall g \in G \setminus H$, we have $\chi(g) = 0$.
\end{proposition}
\begin{proof}
    It is clear that $\exists d_1, \ldots, d_r$ such that $\chi \downarrow H = \sum_{i=1}^{r} d_i \psi_i$. To arrive at the inequality, we consider the inner-product of $\chi \downarrow H$ with itself:
    \begin{align*}
        \cycl{\chi \downarrow H, \chi \downarrow H}_H &= \frac{1}{\abs{H}} \sum_{h \in H} \chi(h) \overline{\chi(h)} \\
        &= \frac{\brac{G:H}}{\abs{G}} \sum_{h \in H} \chi(h) \overline{\chi(h)}
    \end{align*}
    Now, since $\chi$ is irreducible (over $G$), we have that
    \begin{align*}
        1 = \cycl{\chi, \chi} &= \frac{1}{\abs{G}} \sum_{g \in G} \chi(g) \overline{\chi(g)} \\
        &= \frac{1}{\abs{H}} \sum_{g \in H} \chi(g) \bar{\chi(g)} + \underbrace{\frac{1}{\abs{G}} \sum_{g \in G \setminus H} \chi(g) \bar{\chi(g)}}_{=: K}
    \end{align*}
    We then have that
    \begin{align*}
        \brac{G : H} = \sum_{i=1}^{r} d_i^2 + \brac{G : H} K
    \end{align*}
    Clearly, this quantity is greater than or equal to $\sum_{i=1}^{r} d_i^2$. Furthermore, $\brac{G : H} K = 0$ iff $K = 0$, which is true iff $\chi(g) = 0$ for all $g \in G \setminus H$.
\end{proof}

\begin{proposition}
    Assume $H$ is a normal subgroup of $G$. Let $V$ be a simple \CGM. Then, all simple constituents\footnote{ie, summands in its direct sum decomposition into simple modules} of its restriction to $H$ have the same dimension.
\end{proposition}
\begin{proof}
    Let $U \leq V$ be a simple $\C H$-submodule of $V$. Since $H$ is normal, for all $g \in G$, the set $gU$ is also a $\C H$-submodule of $V$: indeed, for all $h \in H$, $hg U = g h' U = g U$ where $h' = g\inv h g$.

    Now, let $W := \sum_{g \in G} gU$. Since $W$ is a nonzero $\C G$-submodule of $V$, and since $V$ is simple, we have that $W = V$. Indeed, since $U$ is simple, so is $gU$ for every $g \in G$. This tells us that for all $g_1, g_2 \in G$,
    \begin{align*}
        g_1 U \cap g_2 U = \begin{cases}
            0 & \text{ in some cases} \\
            g_1 U & \text{ otherwise}
        \end{cases}
    \end{align*}
    as $g_1 U \cap g_2 U$ is a submodule of the (simple) module $g_1 U$.\footnote{According to Dr. Rizzoli, "there is no naïve answer" to the question of when the intersection is zero and when it is not.}

    Hence, the sum in the definition of $W$, considered only over indices $i$ such that distinct summands have zero intersection, is direct. Therefore, all \verb|FINISH!|
\end{proof}

\begin{corollary}\label{Ch3:Cor:NSGp_Irred_Constit_Char}
    Assume $H$ is a normal subgroup of $G$. Fix $\chi \in \Irr{G}$. Then, all irreducible constituents of $\chi \downarrow H$ have the same degree.
\end{corollary}

\begin{proposition}
    Let $H$ be normal in $G$, of degree $2$. Fix $\chi \in \Irr{G}$. Then, $\chi \downarrow H$ is either irreducible or the sum of two irreducible characters of equal degree.
\end{proposition}
\begin{proof}
    Let $\Irr{H} = \set{\phi_1, \ldots, \phi_r}$. By Proposition \ref{Ch3:Prop:Char_Rest_Bound_Degree}, we have that $\chi \downarrow H = \sum_{i=1}^{r} d_i \phi_i$ for some $d_1, \ldots, d_r \in \N$. Since $\brac{G : H} = 2$, we have that $\sum_{i=1}^{r} d_i^2 \leq 2$. This implies that either $r = 1$ or $r = 2$ and $d_1 = d_2 = 1$. \verb|Refine a bit|
\end{proof}

\begin{remark}
    Consider the situation described in the above proposition. We have that $\quotient{G}{H} \cong C_2$, the cyclic group of order $2$. By the Representation Theory of Cyclic Groups (cf. \verb|add reference|), we have that there is a linear character $\lambda$ of $\quotient{G}{H}$ such that for all $gH \in \quotient{G}{H}$,
    \begin{align*}
        \lambda(gH) =
        \begin{cases}
            -1 & \text{ if } g \notin H \\
            1 & \text{ if } g \in H
        \end{cases}
    \end{align*}
    This lifts to a character $\bar{\lambda}$ of $G$ given by $\bar{\lambda}(g) = \lambda(gH)$. We then have that for all $\chi \in \Irr{G}$, $\bar{\lambda} \chi$ is irreducible, agreeing with $\chi$ upon restriction to $H$.
\end{remark}
\begin{lemma}
    \verb|either \chi or \bar{\lambda}\chi|
\end{lemma}
\begin{proof}
    We have that for all $g \in G$,
    \begin{align*}
        \parenth{\chi + \bar{\lambda}\chi}\!(g) =
        \begin{cases}
            0 \text{ if } g \notin H \\
            2\chi(g) \text{ if } g \in H
        \end{cases}
    \end{align*}
    Now, let $\chi_2$ be an irreducible character of $G$ such that $\chi_2 \downarrow H = \chi \downarrow H$. We show that $\chi_2$ is either $\chi$ or $\bar{\lambda} \chi$. To that end, we compute the inner-product of $\chi + \bar{\lambda}\chi$ with $\chi_2$:
    \begin{align*}
        \cycl{\chi + \bar{\lambda}\chi, \chi_2} &= \frac{1}{\abs{G}} \sum_{g \in G} \parenth{\chi(g) + \bar{\lambda}\chi(g)} \overline{\chi_2(g)} \\
        &= \frac{1}{\abs{G}} \sum_{g \in H} 2\chi(g) \overline{\chi_2(g)} \\
        %&= \frac{2}{\abs{G}} \sum_{h \in H} \chi(h) \overline{\chi_2(h)} \\
        &= \frac{1}{\abs{H}} \sum_{g \in H} \chi(g) \bar{\chi_2(g)} \\
        &= \cycl{\chi \downarrow H, \chi \downarrow H}_H = 1
    \end{align*}
\end{proof}

\begin{proposition}
    Let $H$ have index $2$ in $G$. Let $\chi \in \Irr{G}$ be expressible as $\psi_1 + \psi_2$, for $\psi_1, \psi_2 \in \Irr{H}$. Let $\phi \in \Irr{G}$ be such that $\cycl{\phi \downarrow H, \psi_i}_H \neq 0$ for $i=1$ or $i=2$. Then, $\phi = \chi$.
\end{proposition}
\begin{proof}
    Since $\brac{G : H} = 1 + 1$, by Proposition \ref{Ch3:Prop:Char_Rest_Bound_Degree}, we have that $\chi(g) = 0$ for all $g \in G \setminus H$. Computing the inner-product, we get that
    \begin{align*}
        \cycl{\phi, \chi} &= \frac{1}{\abs{G}} \sum_{g \in G} \phi(g) \overline{\chi(g)} \\
        &= \frac{1}{\abs{H}} \sum_{h \in H} \phi(h) \overline{\chi(h)} \\
        &= \frac{1}{2} \cycl{\phi \downarrow H, \chi}_H \neq 0
    \end{align*}
    proving that $\phi = \chi$. \verb|Need more details!|
\end{proof}

\subsection{Induction}

In this subsection, we will describe a process whereby we can use the representation theory of the subgroups of $G$ to understand that of $G$.

The situation we will attempt to understand is the following. Given a subgroup $H \leq G$, with $U$ a $\C H$-submodule of $\C H$ (which in turn is a submodule of $\C G$), we will attempt to understand the $\C G$-submodule of $G U$, which we will refer to as the \emph{induced module} of $U$. We will define this more formally very soon.

First, observe that for $H \leq G$, the \CGM\ $\C G$ is naturally a right \CHM\ with action $h(g) = gh$. Hence, taking the tensor product of $\C G$ with a \CHM\ \textit{over $\C H$} is a sensible thing to do.

\begin{boxdefinition}
    Let $H \leq G$ and let $V$ be a \CHM. The induced module of $V$ from $H$ to $G$, denoted $\Ind_H^G(V)$, $V^G$ or $V \uparrow G$, is the \CGM\ $\C G \otimes_{\C H} V$ equipped with scalar multiplication
    \begin{align}
        g \cdot \parenth{x \otimes v} = gh \otimes v
    \end{align}
    for all $g \in G$, $x \in \C G$ and $v \in V$.
\end{boxdefinition}