\section{The Character Theory of Subgroups}

In this section, we will explore how we can understand the characters of a group using those of its subgroups and vice-versa. We will begin by discussing the latter, which we will study from the perspective of Restriction. We will then discuss the former, which we will study from the perspective of Induction.

\subsection{Restriction}

For the entirety of this subsection, let $G$ be a finite group and let $H \leq G$. % We are interested in the restriction of this representation to $H$.

\begin{boxdefinition}[Restriction of a Representation]
    Let $\Vp$ be a complex representation of $G$ with associated character $\chi$. The restriction of $\Vp$ to $H$, denoted $V \downarrow H$, $\Res_H^G(V)$ or $V_H$, is the set representation $\parenth{V, \rho\vert_H}$. We will denote its associated character $\chi \downarrow H$ or $\chi_H$.
\end{boxdefinition}

\begin{boxnotation}
    To distinguish between inner-products of central functions considered over $H$ and those over $G$, we will denote the former by $\cycl{\cdot, \cdot}_H$. In similar fashion to \eqref{SP:eq:Char_Inner_Product}, we have, for all $\psi_1, \psi_2 \in \Fcof{H, \C}$,
    \begin{align}
        \cycl{\psi_1, \psi_2}_H := \frac{1}{\abs{H}} \sum_{h \in H} \psi_1(h) \overline{\psi_2(h)}
    \end{align}
\end{boxnotation}

\begin{proposition}
    Fix $\psi \in \Irr{H}$. Then, there exists an irreducible character $\chi$ of $G$ such that the inner-product $\cycl{\chi \downarrow H, \psi}_H$ is nonzero. 
\end{proposition}
\begin{proof}
    Let $\Irr{G} = \set{\chi_1, \ldots, \chi_k}$. Consider the regular character $\chi_{\reg}$ of $G$. We then have \verb|(insert reference)| that for all $g \in G$,
    \begin{align*}
        \chi_{\reg} = \sum_{i = 1}^k \chi_i(1) \chi_i
        = \begin{cases}
            \abs{G} & \text{ if } g = 1 \\
            0 & \text{ otherwise}
        \end{cases}
    \end{align*}
    We now show that $\cycl{\chi_{\reg} \downarrow H, \psi}_H$ is nonzero:
    \begin{align*}
        \cycl{\chi_{\reg} \downarrow H, \psi}_H &= \frac{1}{\abs{H}} \sum_{h \in H} \chi_{\reg}(h) \overline{\psi(h)} \\
        &= \frac{1}{\abs{H}} \abs{G} \overline{\psi(1)} \neq 0
    \end{align*}
    Considering the decomposition of $\chi_{\reg}$ as a sum of irreducible characters, we see that one of the summands must have a nonzero inner-product with $\psi$.
\end{proof}

\begin{remark}
    In general, we can compute $\Irr{H}$ from $\Irr{G}$ quite easily if $\brac{G : H}$ is small. However, if $\brac{G : H}$ is large, then this can get difficult.
\end{remark}

\begin{proposition} \label{Ch3:Prop:Char_Rest_Bound_Degree}
    Fix $\chi \in \Irr{G}$, and write $\Irr{H} = \set{\psi_1, \ldots, \psi_r}$. Then, $\exists d_1, \ldots, d_r \in \N$ such that
    \begin{align*}
        \chi \downarrow H = \sum_{i=1}^{r} d_i \psi_i
        \quad\quad
        \text{ and }
        \quad\quad
        \sum_{i=1}^{r} d_i^2 \leq \brac{G : H}
    \end{align*}
    Moreover, the above inequality is an \emph{eq}uality if and only if $\forall g \in G \setminus H$, we have $\chi(g) = 0$.
\end{proposition}
\begin{proof}
    It is clear that $\exists d_1, \ldots, d_r$ such that $\chi \downarrow H = \sum_{i=1}^{r} d_i \psi_i$. To arrive at the inequality, we consider the inner-product of $\chi \downarrow H$ with itself:
    \begin{align*}
        \cycl{\chi \downarrow H, \chi \downarrow H}_H &= \frac{1}{\abs{H}} \sum_{h \in H} \chi(h) \overline{\chi(h)} \\
        &= \frac{\brac{G:H}}{\abs{G}} \sum_{h \in H} \chi(h) \overline{\chi(h)}
    \end{align*}
    Now, since $\chi$ is irreducible (over $G$), we have that
    \begin{align*}
        1 = \cycl{\chi, \chi} &= \frac{1}{\abs{G}} \sum_{g \in G} \chi(g) \overline{\chi(g)} \\
        &= \frac{1}{\abs{H}} \sum_{g \in H} \chi(g) \bar{\chi(g)} + \underbrace{\frac{1}{\abs{G}} \sum_{g \in G \setminus H} \chi(g) \bar{\chi(g)}}_{=: K}
    \end{align*}
    We then have that
    \begin{align*}
        \brac{G : H} = \sum_{i=1}^{r} d_i^2 + \brac{G : H} K
    \end{align*}
    Clearly, this quantity is greater than or equal to $\sum_{i=1}^{r} d_i^2$. Furthermore, $\brac{G : H} K = 0$ iff $K = 0$, which is true iff $\chi(g) = 0$ for all $g \in G \setminus H$.
\end{proof}

\begin{proposition}
    Assume $H$ is a normal subgroup of $G$. Let $V$ be a simple \CGM. Then, all simple constituents\footnote{ie, summands in its direct sum decomposition into simple modules} of its restriction to $H$ have the same dimension.
\end{proposition}
\begin{proof}
    Let $U \leq V$ be a simple $\C H$-submodule of $V$. Since $H$ is normal, for all $g \in G$, the set $gU$ is also a $\C H$-submodule of $V$: indeed, for all $h \in H$, $hg U = g h' U = g U$ where $h' = g\inv h g$.

    Now, let $W := \sum_{g \in G} gU$. Since $W$ is a nonzero $\C G$-submodule of $V$, and since $V$ is simple, we have that $W = V$. Indeed, since $U$ is simple, so is $gU$ for every $g \in G$. This tells us that for all $g_1, g_2 \in G$,
    \begin{align*}
        g_1 U \cap g_2 U = \begin{cases}
            0 & \text{ in some cases} \\
            g_1 U & \text{ otherwise}
        \end{cases}
    \end{align*}
    as $g_1 U \cap g_2 U$ is a submodule of the (simple) module $g_1 U$.\footnote{According to Dr. Rizzoli, "there is no naïve answer" to the question of when the intersection is zero and when it is not.}

    Hence, the sum in the definition of $W$, considered only over indices $i$ such that distinct summands have zero intersection, is direct. Therefore, all \verb|FINISH!|
\end{proof}

\begin{corollary}\label{Ch3:Cor:NSGp_Irred_Constit_Char}
    Assume $H$ is a normal subgroup of $G$. Fix $\chi \in \Irr{G}$. Then, all irreducible constituents of $\chi \downarrow H$ have the same degree.
\end{corollary}

\begin{proposition}
    Let $H$ be normal in $G$, of degree $2$. Fix $\chi \in \Irr{G}$. Then, $\chi \downarrow H$ is either irreducible or the sum of two irreducible characters of equal degree.
\end{proposition}
\begin{proof}
    Let $\Irr{H} = \set{\phi_1, \ldots, \phi_r}$. By Proposition \ref{Ch3:Prop:Char_Rest_Bound_Degree}, we have that $\chi \downarrow H = \sum_{i=1}^{r} d_i \phi_i$ for some $d_1, \ldots, d_r \in \N$. Since $\brac{G : H} = 2$, we have that $\sum_{i=1}^{r} d_i^2 \leq 2$. This implies that either $r = 1$ or $r = 2$ and $d_1 = d_2 = 1$. \verb|Refine a bit|
\end{proof}

\begin{remark}
    Consider the situation described in the above proposition. We have that $\quotient{G}{H} \cong C_2$, the cyclic group of order $2$. By the Representation Theory of Cyclic Groups (cf. \verb|add reference|), we have that there is a linear character $\lambda$ of $\quotient{G}{H}$ such that for all $gH \in \quotient{G}{H}$,
    \begin{align*}
        \lambda(gH) =
        \begin{cases}
            -1 & \text{ if } g \notin H \\
            1 & \text{ if } g \in H
        \end{cases}
    \end{align*}
    This lifts to a character $\bar{\lambda}$ of $G$ given by $\bar{\lambda}(g) = \lambda(gH)$. We then have that for all $\chi \in \Irr{G}$, $\bar{\lambda} \chi$ is irreducible, agreeing with $\chi$ upon restriction to $H$.
\end{remark}
\begin{lemma}
    \verb|either \chi or \bar{\lambda}\chi|
\end{lemma}
\begin{proof}
    We have that for all $g \in G$,
    \begin{align*}
        \parenth{\chi + \bar{\lambda}\chi}\!(g) =
        \begin{cases}
            0 \text{ if } g \notin H \\
            2\chi(g) \text{ if } g \in H
        \end{cases}
    \end{align*}
    Now, let $\chi_2$ be an irreducible character of $G$ such that $\chi_2 \downarrow H = \chi \downarrow H$. We show that $\chi_2$ is either $\chi$ or $\bar{\lambda} \chi$. To that end, we compute the inner-product of $\chi + \bar{\lambda}\chi$ with $\chi_2$:
    \begin{align*}
        \cycl{\chi + \bar{\lambda}\chi, \chi_2} &= \frac{1}{\abs{G}} \sum_{g \in G} \parenth{\chi(g) + \bar{\lambda}\chi(g)} \overline{\chi_2(g)} \\
        &= \frac{1}{\abs{G}} \sum_{g \in H} 2\chi(g) \overline{\chi_2(g)} \\
        %&= \frac{2}{\abs{G}} \sum_{h \in H} \chi(h) \overline{\chi_2(h)} \\
        &= \frac{1}{\abs{H}} \sum_{g \in H} \chi(g) \bar{\chi_2(g)} \\
        &= \cycl{\chi \downarrow H, \chi \downarrow H}_H = 1
    \end{align*}
\end{proof}

\begin{proposition}
    Let $H$ have index $2$ in $G$. Let $\chi \in \Irr{G}$ be expressible as $\psi_1 + \psi_2$, for $\psi_1, \psi_2 \in \Irr{H}$. Let $\phi \in \Irr{G}$ be such that $\cycl{\phi \downarrow H, \psi_i}_H \neq 0$ for $i=1$ or $i=2$. Then, $\phi = \chi$.
\end{proposition}
\begin{proof}
    Since $\brac{G : H} = 1 + 1$, by Proposition \ref{Ch3:Prop:Char_Rest_Bound_Degree}, we have that $\chi(g) = 0$ for all $g \in G \setminus H$. Computing the inner-product, we get that
    \begin{align*}
        \cycl{\phi, \chi} &= \frac{1}{\abs{G}} \sum_{g \in G} \phi(g) \overline{\chi(g)} \\
        &= \frac{1}{\abs{H}} \sum_{h \in H} \phi(h) \overline{\chi(h)} \\
        &= \frac{1}{2} \cycl{\phi \downarrow H, \chi}_H \neq 0
    \end{align*}
    proving that $\phi = \chi$. \verb|Need more details!|
\end{proof}

\subsection{Induction}

In this subsection, we will describe a process whereby we can use the representation theory of the subgroups of $G$ to understand that of $G$. For the remainder of this section, let $H$ be a subgroup of $G$.

The situation we will attempt to understand is the following. Given $U$ a $\C H$-submodule of $\C H$ (which in turn is a submodule of $\C G$), we will attempt to understand the $\C G$-submodule of $G U$, which we will refer to as the \emph{induced module} of $U$. We will define this more formally very soon.

First, observe that the \CGM\ $\C G$ is naturally a right \CHM\ with action $h(g) = gh$. Hence, taking the tensor product of $\C G$ with a \CHM\ \textit{over $\C H$} is a sensible thing to do.

\begin{boxdefinition}[Induced Module]
    Let $V$ be a \CHM. The induced module of $V$ from $H$ to $G$, denoted $\Ind_H^G(V)$, $V^G$ or $V \uparrow G$, is the \CGM\ $\C G \otimes_{\C H} V$ equipped with scalar multiplication
    \begin{align}
        g \cdot \parenth{x \otimes v} = gx \otimes v
    \end{align}
    for all $g \in G$, $x \in \C G$ and $v \in V$.
\end{boxdefinition}

We can give a more concrete description of the induced module.

\begin{proposition}
    Let $k = \brac{G : H}$ and let $g_1, \ldots, g_k$ be a complete set of coset representatives (so that $G = \bigsqcup_{i=1}^{k} g_i H$). As a vector space, we have that
    \begin{align}
        \Ind_H^G(V) = \bigoplus_{i=1}^{k} g_i \otimes V
        \label{Ch3:eq:Induced_Module_DSum}
    \end{align}
    where we denote by $g_i \otimes V$ the set of elements of the form $g_i \otimes v$ for $v \in V$.\footnote{Such a vector space is naturally isomorphic to $V$.} Furthermore, $\C G$ transitively permutes the summands $g_i \otimes V$ in \eqref{Ch3:eq:Induced_Module_DSum} via the action
    \begin{align*}
        g \cdot \parenth{g_i \otimes v} = g_j \otimes v
    \end{align*}
    where $g \in G$ is arbitrary and $j$ is such that $g g_i \in g_j H$.
\end{proposition}
\begin{proof}
    As a right \CHM, we have that
    \begin{align*}
        \C G &= \C \parenth{\bigsqcup_{i=1}^{k} g_i H} \\
        &= \bigoplus_{i=1}^{k} \C \parenth{g_i H} \\
        &= \bigoplus_{i=1}^{k} g_i \otimes \C H
    \end{align*}
    Since $\Ind_H^G(V) = \C G \otimes_{\C H} V$, replacing $\C G$ with the above sum, we get that
    \begin{align*}
        \Ind_H^G(V) &= \parenth{\bigoplus_{i=1}^{k} g_i \otimes \C H} \otimes_{\C H} V \\
        &= \bigoplus_{i=1}^{k} g_i \otimes \parenth{\C H \otimes_{\C H} V} \\
        &\cong \bigoplus_{i=1}^{k} g_i \otimes V
    \end{align*}
    as in \eqref{Ch3:eq:Induced_Module_DSum}, proving the first part of the proposition.

    For the second part, fix $g \in G$ and $1 \leq i \leq k$. Let $1 \leq j \leq k$ and $h \in H$ be such that $g g_i = g_j h$. We then have that
    \begin{align*}
        g\parenth{g_i \otimes V} = g g_i \otimes V = g_j h \otimes V = g_j \otimes h V = g_j \otimes V
    \end{align*}
    demonstrating that we have the desired permutations of the summands in \eqref{Ch3:eq:Induced_Module_DSum}. The only thing left to show is that this action is transitive. \verb|ASK PROF TO CLARIFY TRANSITIVITY|
\end{proof}

\begin{corollary}
    As a $\C$-vector space, we have that $\displaystyle \pdim{\Ind_H^G(V)} = \brac{G : H} \cdot \pdim{V}$.
\end{corollary}

\begin{boxexample}
    The induced module of the trivial module $\C$ from $H$ to $G$ is the permutation module of $G$ acting on $\quotient{G}{H}$ by $g \cdot g_i H = g_j H$.
\end{boxexample}

We now turn our attention to the characters of the induced module. We will begin by characterising the character of the induced module.

\begin{proposition}
    If $V$ is a \CHM\ with character $\chi$, the character of the induced module, denoted $\Ind_H^G(\chi)$, is given by
    \begin{align*}
        \Ind_H^G(\chi)(g) = \frac{1}{\abs{H}} \sum_{x\inv g x \in H} \chi(x\inv g x)
    \end{align*}
    for all $g \in G$.
\end{proposition}
\begin{proof}
    Fix $g \in G$. We determine the trace of the action of $g$ on $\Ind_H^G(V)$ as follows.

    Consider the decomposition $\Ind_H^G(V) = \bigoplus_{i=1}^{k} g_i \otimes V$ (as given by \eqref{Ch3:eq:Induced_Module_DSum}, with $\set{g_i}_{i=1}^k$ being a set of representatives for the elements of $\quotient{G}{H}$). Consider the action of $g$ with respect to a basis given by the various bases of $g_i \otimes V$.

    \verb|DRAW MATRIX FROM iPAD|

    If $g$ does not fix $g_i \otimes V$, then the contribution $g_i \otimes V$ to $\Tr{g}$ is zero. If $g$ does fix $g_i \otimes V$ (or, equivalently, if $g_i\inv g g_i \in H$), then the contribution is given by
    \begin{align*}
        \chi(g_i\inv g g_i) = g \parenth{g_i \otimes v} = g_i \parenth{g_i g g\inv} \otimes v = g_i \otimes \parenth{g_i g g\inv}v
    \end{align*}
    for all $v \in V$. We can then conclude that
    \begin{align*}
        \Ind_H^G(\chi)(g) = \sum_{g_i\inv g g_i \in H} \chi(g_i\inv g g_i)
    \end{align*}
    Indeed, this sum can be extended to a sum over \textit{all} $x \in G$ such that $x\inv g x \in H$ (except we need to divide by $H$ when we do so), giving us the desired result. \verb|Eh???|
\end{proof}

\begin{boxexample}
    Let $G = S_3$ and $H = \set{1, \parenth{123}, \parenth{132}}$. Choose $1$ and $\parenth{12}$ as a set of coset representatives for $\quotient{G}{H}$. Let $\chi$ be the character of the trivial module of $H$. Then,
\end{boxexample}

\subsection{The Relationship between Induction and Restriction}

It is no coincidence that we investigated these phenomena in succession. As we shall see, they share an interesting relationship. This turns out to be a consequence of the following important \textbf{universal property}.

\begin{boxproposition}[The Universal Property of the Induced Module]
    Let $V$ be a \CHM. For all \CGM s $W$ and all \CHM\ homomorphisms $f : V \to W \downarrow H$, there exists a unique \CGM\ homomorphism $\tilde{f} : \Ind_H^G(V) \to W$ such that the following diagram commutes:
    \begin{cd}  % Mostly generated by Copilot
        V \arrow{r}{f} \arrow[d, "\iota"'] & W \downarrow H \arrow{d}{\iota'} \\
        \Ind_H^G(V) \arrow[r, dashed, "\exists! \tilde{f}"] & W
    \end{cd}
    where $\iota : V \to \Ind_H^G(V)$ is the map $v \mapsto 1 \otimes v$ and $\iota' : W \downarrow H \to W$ is the map \verb|sorry|.
\end{boxproposition}

\begin{proof}
    Since $f$ is a morphism of \CHM s, the map $\bar{f} : \C G \times V \to W : \parenth{x, v} \mapsto x f(v)$ is $\C H$-stable: for all $x, x' \in \C G$, $v, v' \in V$ and $y \in \C H$,
    \begin{align*} % Thank you Copilot!
        \bar{f}(x + x', v) &= (x + x') f(v) = x f(v) + x' f(v) = \bar{f}(x, v) + \bar{f}(x', v) \\
        \bar{f}(x, v + v') &= x f(v + v') = x f(v) + x f(v') = \bar{f}(x, v) + \bar{f}(x, v') \\
        \bar{f}(xy, v) &= (xy) f(v) = x y f(v) = x f(y v) = x f(y v) = \bar{f}(x, y v)
    \end{align*}
    By the Universal Property of the Tensor Product, $\exists! \tilde{f} : \C G \otimes_{\C H} V \to W : x \otimes V \mapsto x f(v)$. It is clear that $\tilde{f}$ is a morphism of \CGM s \verb|(is it?)|. Finally, one can check that the diagram commutes.
\end{proof}

This gives us the following important relationship between induction and restriction.

\begin{corollary}
    Let $V$ be a \CHM\ and $W$ a \CHM, as above. Then, there is an isomorphism of $\C$-vector spaces between $\Hom_{\C H}\!\parenth{V, W \downarrow H}$ and $\Hom_{\C G}\!\parenth{\Ind_H^G(V), W}$.
\end{corollary}

\begin{proof}
    Fix $\varphi \in \Hom_{\C G}(\Ind_H^G(V), W)$. Then, $\varphi \circ \iota$ is a morphism of \CHM s from $V$ to $W \downarrow H$. One can use the Universal Property to show that $\phi \mapsto \phi \circ \iota$ is $\C$-linear, injective and surjective.
    % [COPILOT EXPLICIT INVERSE:] Conversely, fix $\psi \in \Hom_{\C H}(V, W \downarrow H)$. Then, $\tilde{\psi} \circ \iota$ is a morphism of \CGM s from $\Ind_H^G(V)$ to $W$. \verb|Finish!|
\end{proof}

\begin{lemma}
    Let $W_1$ and $W_2$ be \CGM s and with respective characters $\chi_1$ and $\chi_2$. Then,
    \begin{align*}
        \pdim{\Hom_{\C G}\!\parenth{W_1, w_2}} = \cycl{\chi_1, \chi_2}
    \end{align*}
    where we consider the dimension to be that of a $\C$-vector space.
\end{lemma}
\begin{proof}
    \verb|Exercise in PS 10. TODO: place elsewhere!|
\end{proof}

\begin{boxtheorem}[Frobenius Reciprocity]
    Let $V$ be a \CHM\ and $W$ a \CGM, with characters $\chi_V$ and $\chi_W$ respectively. Then,
    \begin{align*}
        \cycl{\Ind_H^G(\chi_V), \chi_W}_{G} = \cycl{V, \Res_H^G(W)}_{H}
    \end{align*}
\end{boxtheorem}
\begin{proof}
    This is a direct consequence of the previous lemma and the corollary \verb|(reference)|:
    \begin{align*}
        \cycl{\Ind_H^G(\chi_V), \chi_W}_{G} &= \pdim{\Hom_{\C G}\!\parenth{\Ind_H^G(V), W}} \\
        &= \pdim{\Hom_{\C H}\!\parenth{V, W \downarrow H}} \\
        &= \cycl{V, W \downarrow H}_{H} = \cycl{V, \Res_H^G(W)}_{H}
    \end{align*}
\end{proof}

% \begin{boxexample}  % Copilot -- VERIFY!
%     Let $G = S_3$ and $H = \set{1, \parenth{123}, \parenth{132}}$. Let $V$ be the trivial module of $H$ and $W$ the sign module of $G$. Then, we have that
%     \begin{align*}
%         \Ind_H^G(\chi_V) = \chi_W
%     \end{align*}
% \end{boxexample}

\begin{boxexample}[The Quaternion Group]
    Let $G = Q_8$, the quaternion group of order $8$, viewed as a subgroup of $\GL{2, \C}$. We know it is generated by
    \begin{align*}
        a = \begin{bmatrix}
            i & 0 \\ 0 & -i
        \end{bmatrix}
        \quad\quad
        \text{and}
        \quad\quad
        b = \begin{bmatrix}
            0 & 1 \\ -1 & 0
        \end{bmatrix}
    \end{align*}
    Let $H = \cycl{A} \cong C_4$, the cyclic group of order $4$. For $\chi \in \Irr{H}$, we determine $\Ind_H^G(\chi)$.

    Consider the following character table for $H$:
    % \begin{table}[h]
    %     \begin{tabular}{c|ccccc}
    %         & $1$ & $-1$ & $a$ & $b$ & $ab$ \\
    %         \hline
    %         $\chi_1$ & $1$ & $1$ & $1$ & $1$ & $1$ \\
    %         $\chi_2$ & $1$ & $1$ & $1$ & $-1$ & $-1$ \\
    %         $\chi_3$ & $1$ & $1$ & $1$ & $1$ & $-1$ \\
    %         $\chi_4$ & $1$ & $1$ & $1$ & $-1$ & $1$ \\
    %         $\chi_5$ & $2$ & $-2$ & $-2$ & $0$ & $0$
    %     \end{tabular}
    % \end{table}
    The group $H$ has $4$ irreducible characters, which we denote $\eta_1, \eta_2, \eta_3, \eta_4$. Indeed, $\eta_1 = \chi_1 \downarrow H$ and $\eta_2 = \chi_3 \downarrow H = \chi_4 \downarrow H$. Furthermore, Frobenius Reciprocity tells us that $\chi_1$ and $\chi_2$ are both constituents of $\Ind_H^G(\eta_1)$---indeed, they are the only ones. Similarly, $\chi_3$ and $\chi_4$ are both constituents of $\Ind_H^G(\eta_2)$. Finally, no linear character is a constituent of $\Ind_H^G(\eta_3)$ or $\Ind_H^G(\eta_4)$.
\end{boxexample}