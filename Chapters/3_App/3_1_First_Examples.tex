\section{First Examples}

\subsection{A Word on Simple Groups}

First, we have an interesting result on $p$-groups.

\begin{lemma} Assume that $\abs{G} = p^n$ for some $p$ prime and $n \in \N$. Then, $\forall \chi \in \Irr{G}$, $\chi(1)$ is a power of $p$.
\end{lemma}
\begin{proof}
    Let $\Irr{G} = \set{\chi_1, \ldots, \chi_k}$, with $\chi_1$ being the trivial character.
    \verb|sorry|
\end{proof}
\begin{corollary}
    If $\abs{G} = p^2$ for some $p$ prime, then $\forall \chi \in \Irr{G}$, $\chi(1) = 1$. In particular, $G$ is abelian.
\end{corollary}
\begin{proof}
    \verb|sorry|
\end{proof}

We now have an interesting result on the characters of simple groups.

\begin{lemma}
    If $G$ is simple, $G$ does not admit an irreducible character of degree $2$.
\end{lemma}
\begin{proof}
    Suppose, for contradiction, that $G$ is simple but admits an irreducible representation $\rho : G \to \GL{2, \C}$. Since $G$ is simple, we must have that $\pker{\rho} = \set{1}$.

    If $G$ is nonabelian, then $G' = G$. This tells us that there are no nontrivial characters of degree $1$. Since the map $\rho \mapsto \pdet{\rho(g)}$ a linear character (ie, has degree $1$), we can conclude that $\pdet{\rho(g)} = 1$ for all $g \in G$. By Theorem \ref{Ch2:Thm:Char_div_Ord_Grp}, we know that $2 \vert \abs{G}$. Then, by Cauchy's Theorem, $\exists g \in G$ such that $g \neq 1$ but $g^2 = 1$. This tells us that $\rho(g) = -\id \in \Zof{G}$, which is absurd, since $G$ is simple. 
\end{proof}
