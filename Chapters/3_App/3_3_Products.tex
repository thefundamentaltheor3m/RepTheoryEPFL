\section{Products, Products, Products}\label{Ch3:Sec:Products}

Thus far, we have seen two ways of combining representations of a group $G$: the direct sum and the tensor product. While we have investigated the properties of the former in detail, we have yet to explore the latter. This will be the primary focus of this section, which we have included in this chapter because we will explore the properties of tensor products of representations by relating them to those of direct products of groups. In particular, our investigation of tensor products will be from a primarily application-driven perspective.

\subsection{The Multiplicative Properties of (Irreducible) Charcaters}

We begin by inviting the reader to refresh their memory on what we have seen so far by taking a look at Subsection \ref{Ch1:Subsec:Tensor_Prods} and the seventh point of Proposition \ref{Ch2:Prop:Bhv_Char}. Before we proceed any further, we establish certain results about the multiplicative properties of characters, with the multiplication in question being that induced by the standard multiplicative structure of $\FcGC$ (which, in turn, inherits its multiplication from $\C$).

% TODO: Add all that stuff from after Theorem 2.1.17.

\begin{lemma}
    The set $R(G) := \Z \Irr{G} \subseteq \FcGC$ is a subring of $\FcGC$.
\end{lemma}
\begin{proof}
    Observe that $R(G)$ is certainly a group under addition. We claim that it is closed under multiplication, with multiplicative identity given by the trivial character.
    
    Let $\chi_V, \chi_W \in \Irr{G}$, with associated (simple) \CGM s $V$ and $W$. From Proposition \ref{Ch2:Prop:Bhv_Char}, we know that their product $\chi_V \chi_W$ is the character associated to the \CGM\ $V \otimes W$. %% WRITE AS SUM_k m_k chi_k for natural numbers m_k. % Since $V \otimes W$ is simple \verb|[JUSTIFY!]|, we have that $\chi_V \chi_W \in \Irr{G}$, proving closure under multiplication.
\end{proof}

\begin{corollary}
    The set $L(G) := \set{\chi \in \Irr{G} : \chi(1) = 1}$ of degree $1$ characters is a multiplicative subgroup of $R(G)$.
\end{corollary}
\begin{proof}
    Obviously, the trivial character is linear, making $L(G)$ nonempty. Further, for all $\chi_V, \chi_W \in L(G)$ (with associated \CGM s $V$ and $W$), clearly, $\chi_V \chi_W$ is a character (namely, that of $V \otimes W$), and $\parenth{\chi_V\chi_W}\!(1) = 1$, proving closure under multiplication. Finally, the inverse of any character $\chi_V$ (with associated \CGM\ $V$) is given by the character $\chi_{V^*}$ of its dual \CGM\ $V^*$, which is also linear, proving the desired result.
\end{proof}

\begin{lemma}
    Let $\chi_V, \chi_W \in \Irr{G}$, with $\chi_V(1) = 1$. Then, $\chi_V \chi_W \in \Irr{G}$.
\end{lemma}
\begin{proof}
    Recall, from Proposition \ref{Ch2:Prop:Bhv_Irred_Char}, that $\chi_V \chi_W$ is irreducible iff $\cycl{\chi_V \chi_W, \chi_V \chi_W} = 1$. Indeed, we have that
    \begin{align*}
        \cycl{\chi_V \chi_W, \chi_V \chi_W} &= \frac{1}{\abs{G}} \sum_{g \in G} \chi_V(g) \cdot \chi_W(g) \cdot \overline{\chi_V(g) \cdot \chi_W(g)} \\
        &= \frac{1}{\abs{G}} \sum_{g \in G} \cancel{\chi_V(g)} \cdot \chi_W(g) \cdot \cancel{\overline{\chi_V(g)}} \cdot \overline{\chi_W(g)} \\
        &= \frac{1}{\abs{G}} \sum_{g \in G} \chi_W(g) \cdot \overline{\chi_W(g)} = 1
    \end{align*}
    We justify the second equality by seeing that $\chi_V(g) \cdot \overline{\chi_V(g)} = 1$ because $\overline{\chi_V(g)} = \chi_V\!\parenth{g\inv}$. Indeed, since $\chi_V$ is of degree $1$, it is, in fact, \textit{equal} to its underlying representation, making it multiplicative (ie, $\chiof{V}{g} \cdot \chiof{V}{g\inv} = \chiof{V}{g g\inv} = 1$). The last equality follows from the fact that $\chi_W \in \Irr{G}$.
\end{proof}

\subsection{Tensor Products and Direct Products}

\begin{boxdefinition}[Tensor Products of Modules of Different Groups]
    Let $K$ be a field, and let $G$ and $H$ be finite groups. Let $V$ be a $KG$-module and let $W$ be a $KH$-module. The tensor product $V \otimes_K W$ of $V$ and $W$ over $K$ is the $K(G \times H)$-module with scalar multiplication
    \begin{align}
        \parenth{g, h} \cdot \parenth{v \otimes w} = \parenth{g \cdot v} \otimes \parenth{h \cdot w}
    \end{align}
    for all $g \in G$, $h \in H$, $v \in V$, and $w \in W$.
\end{boxdefinition}

\begin{proposition}
    Let $G, H$ be finite groups and let $V$ be a \CGM\ and $W$ a \CHM. Then, the character $\chi_{V \otimes W}$ associated to the \CGHM\ $V \otimes W$ is given by
    \begin{align*}
        \chiof{V \otimes W}{g, h} = \chi_V(g) \chi_W(h)
    \end{align*}
    for all $g \in G$ and $h \in H$.
\end{proposition}
\begin{proof}
    We essentially consider $V$ and $W$ as \CGHM s and then apply the definition of the tensor product. \verb|Finish!|
\end{proof}

\begin{boxnotation}
    Let $G, H$ be finite groups and let $V$ be a \CGM\ and $W$ a \CHM. Then, the character of the tensor product $V \otimes W$ is denoted $\chi_V \times \chi_W$.
\end{boxnotation}

\begin{boxtheorem}
    $\Irr{G \times H} = \set{\chi \times \psi : \chi \in \Irr{G}, \psi \in \Irr{H}}$.
\end{boxtheorem}
\begin{proof}
    Let $\Irr{G} = \set{\chi_1, \ldots, \chi_k}$ and let $\Irr{H} = \set{\psi_1, \ldots, \psi_r}$. The conjugacy classes of $G \times H$ are of the form $g^G \times h^H$ for $g \in G$ and $h \in H$ \verb|(exercise)|. We show that $\chi_i \times \psi_j$ is irreducible for all $1 \leq i \leq k, 1 \leq j \leq r$, with the characters being distinct (ie, if $i \neq \ell$ or $j \neq m$ then $\chi_i \times \psi_j \neq \chi_\ell \times \psi_m$).  % Why sufficient? -- Because basis. Justify!

    We use the fact that a character is irreducible iff its character is $1$ \verb|(insert reference)|. Indeed, for all $1 \leq i, \ell \leq k$ and $1 \leq j, m \leq r$,
    \begin{align*}
        \cycl{\chi_i \times \psi_j, \chi_\ell} &= \frac{1}{\abs{G \times H}} \sum_{(g, h) \in G \times H} \chi_i(g) \psi_j(h) \chi_\ell\!\parenth{g\inv}\psi_m\!\parenth{h\inv} \\
        &= \parenth{\frac{1}{\abs{G \times H}} \sum_{g \in G} \chi_i(g) \chi_\ell\!\parenth{g\inv}} \parenth{\sum_{h \in H} \psi_j(h) \psi_m\!\parenth{h\inv}} \\
        &= \delta_{i\ell} \delta_{jm}
    \end{align*}
    \verb|Fill in missing computations---it's not hard.|
\end{proof}

\begin{boxexample}[The Acyclic Group of Order $9$]
    Let $G = H = C_3 = \cycl{g}$, the cyclic group of order $3$ generated by $g$, acting irreducibly on $V = W = \R^2$ via
    \begin{align*}
        g \cdot \begin{bmatrix} x \\ y \end{bmatrix} = \begin{bmatrix} -y \\ xy \end{bmatrix}
    \end{align*}
    In other words, let its associated representation $\rho : G \to \GL{2, \R}$ be given by the matrix
    \begin{align*}
        \rho(g) = \begin{bmatrix} 0 & -1 \\ 1 & -1 \end{bmatrix}
    \end{align*}
    with respect to the standard basis. 
\end{boxexample}

It is useful to think about powers of characters.

\begin{boxnotation}
    Let $G$ be a finite group and let $\chi$ be a character of $G$. Then, for all $n \in \N$, the $n$th power of $\chi$ is the character $\chi^n$ of $G$ given by
    \begin{align*}
        \chi^n = \underbrace{\chi \times \cdots \times \chi}_{n \text{ times}}
    \end{align*}
\end{boxnotation}

\begin{theorem}
    Let $G$ be a finite group and let $\chi$ be a faithful character of $G$. Suppose that $\abs{\set{\chi(g) : g \in G} = r}$. For all $\psi \in \Irr{G}$, $\exists 0 \leq i \leq r - 1$ such that $\cycl{\psi, \chi^i} \neq 0$.
\end{theorem}
\begin{proof}
    Let $G = \set{g_1 = 1, g_2, \ldots, g_r}$. \verb|FINISH!|
\end{proof}