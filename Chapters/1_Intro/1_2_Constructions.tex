\section{Invariant Constructions}

In this section, we briefly examine how ordinary linear algebraic constructions can interact with representations. We are particularly interested in the notion of \textit{invariance}, wherein a construction respects the structure of the representation(s) involved.

\subsection{Direct Sums of Representations}

The most elementary operation we can think about when we have two objects is \textit{putting them together}. One of the most meaningful ways of doing so in the context of linear algebra is the direct sum of two vector spaces. It turns out that this extends rather naturally to representations.

\begin{boxdefinition}[The Direct Sum of Two Representations]
    \letgvv. We define the direct sum of $\parenth{V,\rho}$ and $\parenth{V', \rho'}$ to be the pair $\parenth{V \+ V', \rho \+ \rho'}$, where $V \+ V'$ is the direct sum of $V$ and $V'$ as vector spaces and $\rho \+ \rho' : G \to \GL{V \+ V'}$ maps every $g \in G$ to the map
    \begin{align*}
        \parenth{\rho \+ \rho'}\!\parenth{g}\!\parenth{v \+ v'} &= \rho(g)(v) \+ \rho'(g)(v') \in \GL{V}
    \end{align*}
\end{boxdefinition}
\begin{proposition}
    \letgvv.
    \begin{enumerate}[label = \normalfont \arabic*., noitemsep]
        \item The direct sum $\parenth{V \+ V', \rho \+ \rho'}$ of $\parenth{V,\rho}$ and $\parenth{V', \rho'}$ is, indeed, a representation of $G$.
        \item $V$ and $V'$ are $G$-invariant subspaces\footnote{Technically, isomorphic to the subspaces $V \+ \set{0}$ and $\set{0} \+ V'$, but we overlook such distinctions.} of $V \+ V'$.
    \end{enumerate}
\end{proposition}
\begin{proof}
    \hfill
    \begin{enumerate}[noitemsep]
        \item Fix $g,h \in G$. For all $v \+ v' \in V \+ V'$,
        \begin{align*}
            \parenth{\rho \+ \rho'}\!(gh)(v \+ v') &= \rho(gh)(v) \+ \rho'(gh)(v') \\
            &= \rho(g)\!\parenth{\rho(h)(v)} \+ \rho'(g)\!\parenth{\rho'(h)(v')} \\
            &= \parenth{\rho \+ \rho'}\!(g)\!\parenth{\parenth{\rho \+ \rho'}\!(h)(v \+ v')}
        \end{align*}
        proving that $\rho \+ \rho'$ is multiplicative. Then, for any $g \in G$, $\parenth{\rho \+ \rho'}\!(g)$ has inverse $\parenth{\rho \+ \rho'}\!\parenth{g\inv}$. Hence, $\rho \+ \rho'$ is a homomorphism from $G$ to $\GL{V \+ V'}$.

        \item Fix $g \in G$ and $v \in V$. Clearly, $\parenth{\rho \+ \rho'}\!(g)(v) = \rho(g)(v)$. Since $\rho(g) \in \GL{V}$, it follows that $\rho(g)(v) \in V$. The proof that $V'$ is $G$-invariant is identical.
    \end{enumerate}
    \vspace{-1em}
\end{proof}

The above proposition gives us another reason to consider the direct sum to be an ``invariant'' construction: while it enriches both the vector space structure and the representation structure of a summand by adding another representation into the mix, it does not take anything away from the constructions that already exist.

With direct sums, we also have similar notions to reducibility. These are given by the following.

\begin{boxdefinition}[Complete Reducibility] \label{Ch1:Def:Comp_Red}
    A representation is said to be completely reducible if it is expressible as a direct sum of irreducible representations.
\end{boxdefinition}

\begin{boxdefinition}[Indecomposability]
    A nonzero representation is said to be indecomposable if it is inexpressible as a direct sum of two proper, nonzero subrepresentations.
\end{boxdefinition}
Nonzero representations that are not indecomposable are said to be decomposable.

\subsection{Complementary Subrepresentations}

It is a well-known fact from Linear Algebra that for any finite-dimensional vector space $V$, for any subspace $W \leq V$, there exists a \textit{complementary} subspace $W' \leq V$ such that $W \+ W' = V$. Over Euclidean spaces, for example, we have the very important notion of \textit{orthogonal} complements.

We can define a similar notion for representations, too.
\begin{boxdefinition}[Complementary Subrepresentation]
    \letgv. Let $\parenth{W, \rho\vert_W}$ be a subrepresentation of $\parenth{V, \rho}$. A complementary subrepresentation of $\parenth{W, \rho\vert_W}$ is a subrepresentation $\parenth{U, \rho\vert_U}$ such that $V = U \+ W$.
\end{boxdefinition}
This notion of complementarity is, indeed, compatible with the notion of direct sums of representations.
\begin{proposition}
    \letgv. Let $\parenth{W, \rho\vert_W}$ and $\parenth{U, \rho\vert_U}$ be complementary subrepresentations. Then, their direct sum $\parenth{V, \rho\vert_W \+ \rho\vert_U}$ is equivalent to $\parenth{V, \rho}$ as a representation of $G$.
\end{proposition}
\begin{proof}
    It suffices to show that $\rho = \rho\vert_W \+ \rho\vert_U$. Then, the identity map would give an equivalence of representations. Indeed, every $v \in V$ is expressible uniquely as a direct sum $w \+ u$ for some $w \in W$ and $u \in U$. So, for all $g \in G$,
    \begin{align*}
        \rho(g)(v) &= \rho(g)(w \+ u) \\
        &= \rho(g)(w) \+ \rho(g)(u) \\
        &= \rho\vert_W(g)(w) \+ \rho\vert_U(g)(u) \\
        &= \parenth{\rho\vert_W \+ \rho\vert_U}\!(g)(w \+ u)
    \end{align*}
    where the sum in the second equality is direct because $W$ and $U$ are $\rho(g)$-invariant.
\end{proof}

We now recall an important result from Linear Algebra.

\begin{definition}[Projection]
    Let $V$ be a vector space and let $T : V \to V$ be linear. Observe that we have the following equivalence:
    \begin{align}
        T^2 = T &\iff \forall w \in \pim{T},\ T(w) = w \label{Ch1:Eq:Proj_Op_Equiv}
    \end{align}
    If $T$ satisfies either one of the above conditions, $T$ is said to be a projection.
\end{definition}

We do not prove \eqref{Ch1:Eq:Proj_Op_Equiv}, but we do prove the following lemma, which will prove to be useful.

\begin{lemma} \label{Ch1:Lem:LinAlg_Proj_Dsum}
    Let $V$ be a vector space. For all projections $T : V \to V$, $V = \pker{T} \+ \pim{T}$.
\end{lemma}
\begin{proof}
    Let $T : V \to V$ be a projection. We then have the following.
    \begin{description}
        \item[\underline{$\pim{T} \cap \pker{T} = \set{0}$:}] Fix $w \in \pim{T} \cap \pker{T}$. Since $w \in \pim{T}$, $\exists v \in V$ such that $w = T(v)$. Furthermore, since $w \in \pker{T}$, $T(w) = 0$. Since $w = \Tof{v}$, this is equivalent to saying that $\Tof{\Tof{v}} = 0$. But, by \eqref{Ch1:Eq:Proj_Op_Equiv}, $\Tof{\Tof{v}} = \Tof{v}$. Hence, $\Tof{v} = 0$. Then, since $\Tof{v} = w$, it follows that $w = 0$.
        \item[\underline{$V = \pker{T} + \pim{T}$:}] Fix $v \in V$. We write $v = \Tof{v} + \parenth{v - \Tof{v}}$. Clearly, $\Tof{v} \in \pim{T}$. Further, $\Tof{v - \Tof{v}} = \Tof{v} - \Tof{v} = 0$. Hence, $v - \Tof{v} \in \pker{T}$.
    \end{description}
    Therefore, we do, indeed, have $V = \pker{T} \+ \pim{T}$.
\end{proof}

It turns out that this gives us an important criterion for decomposability.

\begin{corollary}\label{Ch1:Cor:G_lin_proj_ker_im}
    \letgv. If $T : \Vp \to \Vp$ is a $G$-linear projection, then $V = \pker{T} \+ \pim{T}$ is a direct sum of subrepresentations.
\end{corollary}
\begin{proof}
    The result follows immediately from Lemma \ref{Ch1:Lem:LinAlg_Proj_Dsum} and Proposition \ref{Ch1:Prop:ker_im_subreps}.
\end{proof}

One also has a converse criterion for $G$-linearity.

\begin{proposition} \label{Ch1:Prop:Proj_Inv_Lin}
    \letgv, and let $T : V \to V$ be a projection. If $\pker{T}$ and $\pim{T}$ are both $G$-invariant, then $T$ is $G$-linear.
\end{proposition}
\begin{proof}
    Since $T$ is a projection, we know that $V = \pker{T} \+ \pim{T}$. Now, fix $g \in G$ and $v \in V$. We know $v$ can uniquely be expressed as $u + w$, where $u \in \pker{T}$ and $w \in \pim{T}$. Then,
    \begin{align*}
        \Tof{\rho(g)(v)} &= \Tof{\underbrace{\rho(g)(u)}_{\in \pker{T}} + \underbrace{\rho(g)(w)}_{\in \pim{T}}} \\
        &= \rho(g)(w) \\
        &= \rho(g)\!\parenth{\Tof{v}}
    \end{align*}
    proving that $T$ is, indeed, $G$-linear.
\end{proof}

\begin{boxexample}
    Consider the situation in Example \ref{Ch1:Eg:Cyclic_Subrep}. As we discussed briefly at the beginning of Subsection \ref{Ch1:Subsec:Subreps}, we can view $\Vp$ as a subrepresentation of $\parenth{V', \rho'}$. Now, consider the linear map $S : V' \to V' : \parenth{x,y,z} \mapsto \parenth{x,y,0}$, where $\parenth{x,y,z}$ are coordinates with respect to the standard basis. This is clearly a projection operator with image $V$, the $\parenth{x,y}$ plane, and kernel the $z$-axis. These are both clearly $G$-invariant, making $S$ a $G$-linear projection.
\end{boxexample}

\subsection{Maschke's Theorem}

Given the theme of this section---namely, understanding the compatibility of ordinary linear-algebraic constructions with representation structures---one might wonder under what conditions (if any) we have the existence of a complementary subrepresentations. The answer lies in Maschke's Theorem, which is the first major result of the course.

\begin{boxtheorem}[Maschke's Theorem] \label{SP:Thm:Maschke}
    \letfg, $K$ a field such that $\pchar{K} \nmid \abs{G}$, and $\Vp$ a representation of $G$ over $K$. Then, any subrepresentation of $V$ admits a complementary subrepresentation.
\end{boxtheorem}
\begin{proof}
    Let $W \leq V$ be $G$-invariant. The idea is to construct a $G$-linear map from $V$ to $V$ with image $W$. Then, by Corollary \ref{Ch1:Cor:G_lin_proj_ker_im}, its kernel would give a complementary subrepresentation.

    From Linear Algebra, we know that $W$ admits a complementary (but not necessarily $G$-invariant) subspace $U \leq V$. Then, every $v \in V$ can uniquely be expressed as a sum $u + w$, where $u \in U$ and $w \in W$. Define $T : V \to V : u + w \mapsto w$. Clearly, $T$ is a projection operator with image $W$ and kernel $U$.

    If $T$ were $G$-linear, we would be done with the proof; unfortunately, $T$ does not have to be $G$-linear. We therefore ``convert'' $T$ into a $G$-linear projection $S : V \to V$ by \textit{averaging over $G$}. Specifically, define
    \begin{align}
        S := \frac{1}{\abs{G}} \sum_{g \in G} \rho(g) \circ T \circ \rho(g)\inv \label{Ch1:Eq:Avging_Maschke_Pf}
    \end{align}
    which is well-defined because $\abs{G} \neq 0$ in $K$. We then show the following.
    \begin{description}
        \item[\underline{$S$ is a projection with image $W$.}] Fix $v \in V$ and express it as $u + w$ for a unique $u \in U$ and $w \in W$. Then, for all $g \in G$,
        \begin{itemize}[label = $-$, noitemsep]
            \item $\Tof{\rho(g)\inv(v)} \in W$ because $T$ is a projection with image $W$.
            \item $\rho(g)\!\parenth{\Tof{\rho(g)\inv(v)}} \in W$ because $\Tof{\rho(g)\inv(v)} \in W$ and $W$ is $G$-invariant.
        \end{itemize}
        Combined with the fact that $W$ is closed under addition, this proves that $\pim{S} \subseteq W$. Conversely, for all $w \in W$ and $g \in G$,
        \begin{itemize}[label = $-$, noitemsep]
            \item $\parenth{\rho(g)\inv}\!(w) = \rhoof{g\inv}\!(w) \in W$ because $W$ is $G$-invariant.
            \item $\Tof{ \rhoof{g\inv}\!(w)} \in W$ because $ \rhoof{g\inv}\!(w) \in W$ and $W$ is $T$-invariant.
            \item $\rho(g)\!\parenth{\Tof{ \rhoof{g\inv}\!(w)}} \in W$ because $W$ is $G$-invariant.
        \end{itemize}
        Combined, again, with the fact that $W$ is closed under addition, this proves that $W \subseteq \pim{S}$. Therefore, we have that $W = \pim{S}$.

        Finally, since $T\vert_W = \id_W$, we have that $\forall w \in \pim{S} = W$,
        \begin{align*}
            S(w) &= \frac{1}{\abs{G}} \sum_{g \in G} \rho(g)\!\parenth{\Tof{\underbrace{\rho(g)\inv(w)}_{\in W}}} \\
            &= \frac{1}{\abs{G}} \sum_{g \in G} \parenth{\rho(g) \circ \rho(g)\inv}\!(w) \\
            &= \frac{1}{\abs{G}} \sum_{g \in G} w = w
        \end{align*}
        proving that $S$ is, indeed, a projection.

        \item[\underline{$S$ is $G$-linear.}]
        Fix $v \in V$ and $h \in G$. We have
        \begin{align*}
            \Sof{\rho(h)(v)} &= \frac{1}{\abs{G}} \sum_{g \in G} \parenth{\rho(g) \circ T \circ \rho(g)\inv}\!\parenth{\rho(h)(v)} \\
            &=  \frac{1}{\abs{G}} \sum_{g \in G} \parenth{\rho(g) \circ T \circ \rhoof{g\inv h}}\!(v)
        \end{align*}
        We now perform a change of variables. Observe that the map $g \mapsto h\inv g : G \to G$ is an automorphism. Hence, writing $g' = h\inv g$, we have
        \begin{align*}
            \Sof{\rho(h)(v)} &=  \frac{1}{\abs{G}} \sum_{g' \in G} \parenth{\rhoof{hg'} \circ T \circ \rhoof{\parenth{g'}\inv}}\!(v) \\
            &= \rhoof{h}\!\parenth{ \frac{1}{\abs{G}} \sum_{g' \in G} \parenth{\rhoof{g'} \circ T \circ \rhoof{g'}\inv}}\!(v) \\
            &= \rho(h)\!\parenth{\Sof{v}}
        \end{align*}
        proving that $S$ is, indeed, $G$-linear.
    \end{description}
    Therefore, by Corollary \ref{Ch1:Cor:G_lin_proj_ker_im}, $\pker{S}$ is a complementary subrepresentation of $W$.
\end{proof}

We also have the following important corollary.

\begin{corollary} \label{Ch1:Cor:Maschke}
    \letfg, $K$ a field such that $\pchar{K} \nmid \abs{G}$. Then, every representation of $G$ over $K$ is completely reducible.
\end{corollary}
\begin{proof}
    Let $\Vp$ be a representation of $G$ over $K$. If $\Vp$ is irreducible, we are done; else, it admits a nonzero, proper subrepresentation, which, by Maschke's Theorem, admits a complementary subrepresentation that is also proper and nonzero. If both of these are irreducible, then we are done; else, repeat this process.
\end{proof}

\begin{remark}
    Nowhere in Definition \ref{Ch1:Def:Comp_Red} do we specify that the decomposition must be finite.
\end{remark}

We note that both hypotheses of Maschke's Theorem---namely, that $G$ is a finite group and that $\pchar{K} \nmid \abs{G}$---are essential for Theorem \ref{SP:Thm:Maschke} (and hence Corollary \ref{Ch1:Cor:Maschke}) to hold.

\begin{boxnexample}[Failure of Maschke's Theorem when $\pchar{K} \mid \abs{G}$]
    Let $G = \cycl{a}$ be a cyclic group of prime order $p$. Let $V = \F_p^2$, and define $\rho : G \to \GL{2,\F_p}$ by
    \begin{align*}
        \rhoof{a^r} &= \begin{bmatrix}
            1 & r \\ 0 & 1
        \end{bmatrix}
        \quad \text{for $0 \leq r \leq p - 1$.}
    \end{align*}
    \begin{enumerate}
        \item $\Vp$ is a representation of $G$ over $\F_p$.
        \item $\Vp$ is not irreducible.
        \item $\Vp$ is not completely reducible.
    \end{enumerate}
\end{boxnexample}

It turns out that Maschke's Theorem also has a \textit{converse}.

\begin{theorem}[Converse of Maschke's Theorem]
    \letfg\ such that every finite-dimensional representation of $G$ over some field $K$ is completely reducible. Then, $\pchar{K} \nmid \abs{G}$.
\end{theorem}
\begin{proof}
    Consider the regular representation $\parenth{K[G], \rho}$ of $G$ over $K$, with basis $\B = \set{e_g : g \in G}$. The idea is to take advantage of the $G$-invariant properties of $\B$.

    Consider the subspace
    \begin{align*}
        W := \set{\sum_{g \in G} \alpha_g e_g : \sum_{g \in G} \alpha_g = 0}
    \end{align*}
    of dimension $\pdim{V} - 1$. It turns out that $W$ is $G$-invariant: for all $\sum_{g \in G} \alpha_g e_g \in W$ and $x \in G$, we have
    \begin{align*}
        \rho(x)\!\parenth{\sum_{g \in G} \alpha_g e_g} &= \sum_{g \in G} \alpha_g e_{xg} \in W
    \end{align*}
    (where the sum of the coefficients $\alpha_g$ is still zero). Then, by assumption, $\exists U \leq V$ that is both $G$-invariant and complementary to $W$. This means that $U$ must be of dimension $1$, and is hence the span of a single vector $u \in U$.

    We study the action of $G$ on $U$. Fix $x \in G$, and write $u = \sum_{g \in G} \beta_g e_g$ for $\beta_g \in K$. Then,
    \begin{align*}
        \rho(x)(u) - u
        &= \sum_{g \in G} \underbrace{\beta_g e_{xg} - \beta_g e_g}_{\in W}
    \end{align*}
    meaning that $\rho(x)(u) - u \in W$. But, $\rho(x)(u) - u \in U$ as well. Since $U \cap W = \set{0}$, this means that $\rho(x)(u) = u$ for all $x \in G$. Hence, the action of $G$ on $U$ is \textit{trivial}. Therefore, for all $x \in G$,
    \begin{align*}
        \sum_{g \in G} \beta_g e_{xg} &= \sum_{g \in G} \beta_g e_g
    \end{align*}
    Comparing coefficients, we conclude that $\beta_{x\inv g} = \beta_g$ for all $x,g \in G$. Letting $x = g$, we get, in particular, that $\forall g \in G$, $\beta_g = \beta_1$. Therefore,
    \begin{align*}
        && u &= \beta_1 \sum_{g \in G} e_g &&& \\
        &\implies& \sum_{g \in G} e_g &\notin W &&& \\
        &\implies& \abs{G} &\neq 0 &&& \\
        &\implies& \pchar{K} &\nmid \abs{G} &&&
    \end{align*}
    as required.
    % Maybe explain a bit more clearly.
\end{proof}

\subsection{The $G$-Invariant Inner-Product}

It turns out that we also have a notion of inner-products being compatible with representation strutures.

\begin{boxdefinition}[$G$-Invariant Inner-Product]
    \letgv\ such that $V$ admits an inner-product $\cycl{\cdot, \cdot}$.We say that $\cycl{\cdot, \cdot}$ is $G$-invariant if $\forall g \in G$ and $\forall x, y \in V$,
    \begin{align*}
        \cycl{x,y} &= \cycl{\rho(g)(x), \rho(g)(y)}
    \end{align*}
\end{boxdefinition}

\begin{proposition}
    \letgv\ over $\C$ such that $\pim{\rho} \leq \U{V}$ (equivalently, $\Vp$ admits a $G$-invariant inner-product). Then, $V$ is completely reducible. % PS 1, Qn 3
\end{proposition}