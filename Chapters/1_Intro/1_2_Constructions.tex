\section{Invariant Constructions}

In this section, we briefly examine how ordinary linear algebraic constructions can interact with representations. We are particularly interested in the notion of \textit{invariance}, wherein a construction respects the structure of the representation(s) involved.

\subsection{Direct Sums of Representations}

The most elementary operation we can think about when we have two objects is \textit{putting them together}. One of the most meaningful ways of doing so in the context of linear algebra is the direct sum of two vector spaces. It turns out that this extends rather naturally to representations.

\begin{boxdefinition}[The Direct Sum of Two Representations]
    \letgvv. We define the direct sum of $\parenth{V,\rho}$ and $\parenth{V', \rho'}$ to be the pair $\parenth{V \+ V', \rho \+ \rho'}$, where $V \+ V'$ is the direct sum of $V$ and $V'$ as vector spaces and $\rho \+ \rho' : G \to \GL{V \+ V'}$ is the map given by
    \begin{align*}
        \parenth{\rho \+ \rho'}\!\parenth{g}\!\parenth{v \+ v'} &= \rho(g)(v) \+ \rho'(g)(v')
    \end{align*}
    for all $g \in G$.
\end{boxdefinition}

\begin{proposition}
    \letgvv.
    \begin{enumerate}[label = \normalfont \arabic*., noitemsep]
        \item The direct sum $\parenth{V \+ V', \rho \+ \rho'}$ of $\parenth{V,\rho}$ and $\parenth{V', \rho'}$ is, indeed, a representation of $G$.
        \item $V$ and $V'$ are $G$-invariant subspaces\footnote{Technically, isomorphic to the subspaces $V \+ \set{0}$ and $\set{0} \+ V'$, but we overlook such distinctions.} of $V \+ V'$.
    \end{enumerate}
\end{proposition}
\begin{proof}
    \hfill
    \begin{enumerate}[noitemsep]
        \item Fix $g,h \in G$. For all $v \+ v' \in V \+ V'$,
        \begin{align*}
            \parenth{\rho \+ \rho'}\!(gh)(v \+ v') &= \rho(gh)(v) \+ \rho'(gh)(v') \\
            &= \rho(g)\!\parenth{\rho(h)(v)} \+ \rho'(g)\!\parenth{\rho'(h)(v')} \\
            &= \parenth{\rho \+ \rho'}\!(g)\!\parenth{\parenth{\rho \+ \rho'}\!(h)(v \+ v')}
        \end{align*}
        proving that $\rho \+ \rho'$ is multiplicative. Then, for any $g \in G$, $\parenth{\rho \+ \rho'}\!(g)$ has inverse $\parenth{\rho \+ \rho'}\!\parenth{g\inv}$. Hence, $\rho \+ \rho'$ is a homomorphism from $G$ to $\GL{V \+ V'}$.

        \item Fix $g \in G$ and $v \in V$. Clearly, $\parenth{\rho \+ \rho'}\!(g)(v) = \rho(g)(v)$. Since $\rho(g) \in \GL{V}$, it follows that $\rho(g)(v) \in V$. The proof that $V'$ is $G$-invariant is identical.
    \end{enumerate}
\end{proof}

The above proposition gives us another reason to consider the direct sum to be an ``invariant'' construction: while it enriches both the vector space structure and the representation structure of a summand by adding another representation into the mix, it does not take anything away from the constructions that already exist.

With direct sums, we also have similar notions to reducibility. These are given by the following.

\begin{boxdefinition}[Complete Reducibility]
    A representation is said to be completely reducible if it is expressible as a direct sum of irreducible representations.
\end{boxdefinition}

\begin{boxdefinition}[Indecomposability]
    A nonzero representation is said to be indecomposable if it is inexpressible as a direct sum of two proper, nonzero subrepresentations.
\end{boxdefinition}
Nonzero representations that are not indecomposable are said to be decomposable.

\subsection{Complementary Subrepresentations and Maschke's Theorem}

It is a well-known fact from Linear Algebra that for any finite-dimensional vector space $V$, for any subspace $W \leq V$, there exists a \textit{complementary} subspace $W' \leq V$ such that $W \+ W' = V$. Over Euclidean spaces, for example, we have the very important notion of \textit{orthogonal} complements.

We can define a similar notion for representations, too.
\begin{boxdefinition}[Complementary Subrepresentation]
    \letgv. Let $\parenth{W, \rho\vert_W}$ be a subrepresentation of $\parenth{V, \rho}$. A complementary subrepresentation of $\parenth{W, \rho\vert_W}$ is a subrepresentation $\parenth{U, \rho\vert_U}$ such that $V = U \+ W$.
\end{boxdefinition}
This notion of complementarity is, indeed, compatible with the notion of direct sums of representations.
\begin{lemma}
    \letgv. Let $\parenth{W, \rho\vert_W}$ and $\parenth{U, \rho\vert_U}$ be complementary subrepresentations. Then, their direct sum $\parenth{V, \rho\vert_W \+ \rho\vert_U}$ is equivalent to $\parenth{V, \rho}$ as a representation of $G$.
\end{lemma}
\begin{proof}
    It suffices to show that $\rho = \rho\vert_W \+ \rho\vert_U$. Then, the identity map would give an equivalence of representations. Indeed, every $v \in V$ is expressible uniquely as a direct sum $w \+ u$ for some $w \in W$ and $u \in U$. So, for all $g \in G$,
    \begin{align*}
        \rho(g)(v) &= \rho(g)(w \+ u) \\
        &= \rho(g)(w) \+ \rho(g)(u) \\
        &= \rho\vert_W(g)(w) \+ \rho\vert_U(g)(u) \\
        &= \parenth{\rho\vert_W \+ \rho\vert_U}\!(g)(w \+ u)
    \end{align*}
    where the sum in the second equality is direct because $W$ and $U$ are $\rho(g)$-invariant.
\end{proof}

Given the theme of this section---namely, understanding the compatibility of ordinary linear-algebraic constructions with representation structures---one might wonder under what conditions (if any) we have the existence of a complementary subrepresentations. The answer lies in Maschke's Theorem, which is the first major result of the course.

\begin{boxtheorem}[Maschke's Theorem] \label{SP:Thm:Maschke}
    \letfg, $K$ a field such that $\pchar{K} \nmid \abs{G}$, and $\Vp$ a representation of $G$ over $K$. Then, any subrepresentation of $V$ admits a complementary subrepresentation.
\end{boxtheorem}


\subsection{The $G$-Invariant Inner-Product}

