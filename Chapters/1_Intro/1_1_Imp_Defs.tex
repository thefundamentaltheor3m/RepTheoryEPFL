\section{Important Definitions}



\subsection{What is a Representation?}

It turns out that representations can be defined quite broadly, sidestepping the geometric niceties (or are they constraints?) of Euclidean spaces.

\begin{boxdefinition}[Group Representation]
    Let $G$ be a group. A representation of $G$ is a pair $\parenth{V, \rho}$ of a vector space $V$ and a group homomorphism $\rho : G \to \GL{V}$.
\end{boxdefinition}

Here, $\GL{V}$ refers to the \textbf{G}eneral \textbf{L}inear group over $V$, consisting of all vector space automorphisms of $V$ equipped with the binary operation of composition.

\begin{definition}[Degree of a Representation]
    \letgv. We define the degree of $V$ to be the dimension of $V$ over its base field.
\end{definition}

There exist innumerable examples of representations throughout mathematics. Below, we give some important ones.

\begin{boxexample}[Important Classes of Representations]
    \hfill
    \begin{enumerate}
        \item \underline{The trivial representation.} \letg\ and $V$ be any vector space. The map $\rho : G \to \GL{V} : g \mapsto \id_{V}$ is a representation.
        
        \item \underline{The zero representation.} \letg\ and let $V = \set{0}$ be the zero vector space over an arbitrary field $K$. The trivial representation over $V$ is known as the zero representation.
        
        \item \underline{The sign representation.} Let $G = S_n$, the symmetric group on $n$ elements, and let $V = K$, a field. Then, $\GL{V} = K^{\times}$, the multiplicative group of $K$. Denoting by $\xi$ the canonical morphism from $\Z$ to $K$, the map
        \begin{align*}
            \rho : G \to \GL{V} : \sigma \mapsto \xiof{\sgn{\sigma}}
        \end{align*}
        is a representation, where $\operatorname{sgn} : G \to \set{-1,1}$ denotes the sign homomorphism.

        \item \underline{Permutation representations.} Let $G$ be a group acting on a finite set $X$, and let $V = K[X]$, the free vector space (over some field $K$) generated by $X$. Consider a $K$-basis $\set{e_x \in V : x \in X}$ of $V$. Then, the map $\rho : G \to \GL{V}$ given by
        \begin{align*}
            \rhoof{g}\!\parenth{e_x} &= e_{g(x)}
        \end{align*}
        is a representation.

        \item \underline{The regular representation.} Let $G$ be a \textit{finite} group. The permutation representation corresponding to the canonical action of $G$ on itself by left-multiplication gives a representation of $G$ over $K[G]$, the free vector space generated by $G$ (as a set) over any field $K$.
    \end{enumerate}
\end{boxexample}

\begin{boxnexample}
    \letg\ and let $V$ be a \underline{nonzero} vector space over an arbitrary field. The map $g \mapsto 0 : G \to \parenth{V \to V}$ is not a representation because the zero map $0 : V \to V$ is not invertible.
\end{boxnexample}

A useful perspective to adopt is that a representation is merely an action of a group on a vector space. And, just as faithful actions are an important class of actions, it will, later on, turn out to be important to have a corresponding notion for representations as well.

\begin{definition}[Faithfulness] \label{Ch1:Def:Faithfulness}
    \letgv. We say $\Vp$ is faithful if $\pker{\rho}$ is trivial.
\end{definition}

In the next subsection, we begin to develop the theory of morphisms of representations, which will be crucial to the study of interactions and relationships between representations.

\subsection{Morphisms of Representations}

\begin{boxdefinition}[Homomorphism of Representations]
    \letg\ and let $\parenth{V,\rho}$ and $\parenth{V',\rho'}$ be two representations of $G$. A homomorphism of representations $T : V \to V$ is a linear map $T : V \to V'$ such that $\forall g \in G$,
    \begin{align}
        T \circ \rhoof{g} &= \rho'(g) \circ T
        \label{Ch1:Eq:Def_Rep_Morph}
    \end{align}
    or equivalently, the following diagram commutes:
    \begin{cd}
        V \arrow[r, "\rho(g)"] \arrow[d, "T"'] & V \arrow[d, "T"] \\
        V' \arrow[r, "\rho'(g)"'] & V'
        \label{Ch1:eq:cd_rep_map}
    \end{cd}
    Such a map $T$ is said to be \textit{$G$-linear}.
\end{boxdefinition}

\begin{remark}
    The term $G$-linear comes from the fact that a homomorphism of representations satisfies the property that $\Tof{g(v)} = \gof{\Tof{v}}$, where the notation $\gof{\cdot}$ represents the action of some $g \in G$, encoded by a representation. In this sense, $T$ is somehow ``linear over $G$''.
\end{remark}

A natural way to define two representations to be equal, or `isomorphic,' is as follows.

\begin{definition}[Equivalence of Representations]
    \letg\ and let $\parenth{V,\rho}$ and $\parenth{V',\rho'}$ be two representations of $G$. We say that $\parenth{V,\rho}$ and $\parenth{V',\rho'}$ are equivalent, denoted $\parenth{V,\rho} \sim \parenth{V',\rho'}$, if there exists a homomorphism $T : \parenth{V,\rho} \to \parenth{V',\rho'}$ that is invertible as a linear map---ie, that gives a linear isomorphism between $V$ and $V'$.
\end{definition}

Representations of the same group over the same vector space need not be equivalent.

\begin{boxexample}[Non-Equivalent Representations of the Klein $4$-Group]
    Let $G = C_2 \times C_2$ be the Klein $4$-group (where $C_2 = \cycl{x}$ is the cyclic group of order $2$). Let $\alpha = \parenth{x, 1}$ and $\beta = \parenth{1, x}$. Together, they generate $G$. \\

    Now, let $K$ be a field. Consider a degree $1$ representation $\rho : G \to K^\times$. We know that $\rhoof{G}$ must be a subgroup of $K^\times$ such that $\abs{\rhoof{G}} \in \set{1, 2, 4}$. If $\pchar{K} = 2$, then $\rho$ must be the trivial representation, since $2 \nmid \abs{K^\times}$. Else, all four maps $\rho$ satisfying
    \begin{align*}
        \parenth{\rhoof{\alpha}, \rhoof{\beta}} = \parenth{\pm 1, \pm 1}
    \end{align*}
    give \textit{non-equivalent} representations of $G$ in $K^\times$. In particular, we see the non-equivalence because $K^\times$ is commutative.
\end{boxexample}

The point of morphisms of representations is to be able to move from one vector space to another without losing the structural information captured by the representation. This is precisely illustrated in \eqref{Ch1:eq:cd_rep_map}.

\begin{boxexample}[Representations of Cyclic Groups over $\R^2$ and $\R^3$] \label{Ch1:Eg:Cyclic_Subrep}
    Consider the cyclic group $C_n = \cycl{g}$ of order $n$. Let $V = \R^2, V' = \R^3$. Together with the respective maps
    \begin{align*}
        \rho : G \to \GL{\R^2} &: g^m \mapsto \begin{bmatrix}
            \pcos{2\pi/m} & -\psin{2\pi/m} \\
            \psin{2\pi/m} & \pcos{2\pi/m}
        \end{bmatrix} \\
        \rho' : G \to \GL{\R^3} &: g^m \mapsto \begin{bmatrix}
            \pcos{2\pi/m} & -\psin{2\pi/m} & 0 \\
            \psin{2\pi/m} & \pcos{2\pi/m} & 0 \\
            0 & 0 & 1
        \end{bmatrix}
    \end{align*}
    they give representations of $C_n$. Consider now the inclusion $T : \R^2 \to \R^3$ whose matrix with respect to the standard bases of $\R^2$ and $\R^3$ is $\begin{bmatrix}
        1 & 0 \\ 0 & 1 \\ 0 & 0
    \end{bmatrix}$. One can see that $T$ gives a map from $\parenth{V, \rho}$ to $\parenth{V', \rho'}$. Indeed, the corestriction of $T$ to its image is a linear isomorphism, which gives an equivalence between $\parenth{V, \rho}$ and $\parenth{T(V), \rho}$, where we restrict the domains of each $\rhoof{g^m}$ to $T(V)$.
\end{boxexample}
The above example leads to an interesting question. Can we think of one representation as being ``contained'' in another?

It turns out that we can.

\subsection{Subrepresentations} \label{Ch1:Subsec:Subreps}

We have the objects; we have the morphisms. It is only natural to think about what the subobjects would be in the context of group representations. And if Example \ref{Ch1:Eg:Cyclic_Subrep} is any indication, they involve something more than just an inclusion. There is some structural property of a sub-vector space of a representation that makes it \textit{compatible} with the representation structure. In the case of Example \ref{Ch1:Eg:Cyclic_Subrep}, for instance, this is the fact that the representation $\rho'$ acted only ``horizontally"--ie, ``parallel" to the subspace $T(V)$.

More generally, it turns out that the property we really require a subspace to have in order to be `compatible' with the representation structure is the following.

\begin{boxdefinition}[$G$-Invariance]
    \letgv. We say that a sub-vector space $W \leq V$ is $G$-invariant if for all $w \in W$ and $g \in G$,
    \begin{align*}
        \rhoof{g}\!(w) \in W
    \end{align*}
    In other words, $W$ is $G$-invariant if $W$ is $\rhoof{g}$-invariant for every $g \in G$.
\end{boxdefinition}

One can make the following observation. \letg, $\parenth{V,\rho}$ a representation of $G$, and $W \leq V$ a $G$-invariant subspace. Then, $\forall g \in G$, $\rhoof{g} \in \GL{W}$. That is, $\rhoof{g}$ is a linear automorphism of $W$ whose inverse, $\rhoof{g\inv}$, is \textit{also} a linear automorphism of $W$. This then leads to the following definition of a subrepresentation.

\begin{boxdefinition}[Subrepresentation]
    \letgv. A subrepresentation of $V$ is a pair $\parenth{W, \rhow}$ consisting of a $G$-invariant subspace $W \leq V$ and the map
    \begin{align*}
        \rhow : G \to \GL{W} : g \mapsto \rhoof{g}\!\vert_W
    \end{align*}
\end{boxdefinition}
It is very important to note that the map $\rhow$ is \textit{not actually a restriction of $\rho$ to a specific domain}. Rather, it is a map that restricts the domain of $\rhoof{g}$ for every $g \in G$.

One can also observe easily that a subrepresentation is given uniquely by a $G$-invariant subspace. Hence, we will often abuse notation and not distinguish between the pair $\parenth{W, \rhow}$ (which is actually a representation) and simply $W$ (which is merely a subspace).

\begin{boxexample} \label{Ch1:Eg:Regular_is_Subrep}
    Let $G$ be a finite group and $K$ a field. Consider the regular representation $\rho : G \to K[G]$. Let $\set{e_g : g \in G}$ denote a basis of $K[G]$. Then, the subspace $W := \Span{\sum_{g \in G} e_g}$ is $G$-invariant.
\end{boxexample}

It turns out that morphisms of representations also give us subrepresentations.

\begin{proposition}\label{Ch1:Prop:ker_im_subreps}
    \letgv. Let $T : \Vp \to \Vp$ be a homomorphism of representations. Then, the subspaces $\pker{T}$ and $\pim{T}$ of $V$ are $G$-invariant.
\end{proposition}
\begin{proof}
    Fix $g \in G$ and $v \in \pker{T}$. We know $\Tof{\rho(g)(v)} = \rho(g)\!\parenth{\Tof{v}}$. Since $\Tof{v} = 0$, $\Tof{\rho(g)(v)} = 0$. Hence, $\rho(g)(v) \in \pker{T}$, proving that $\pker{T}$ is $G$-invariant.

    Now, fix $w \in \pim{T}$. Then, $w = \Tof{u}$ for some $u \in V$. Clearly, $\rho(g)(w) = \rho(g)\!\parenth{\Tof{u}} = \Tof{\rho(g)(u)} \in \pim{T}$, proving that $\pim{T}$ is $G$-invariant as well.
\end{proof}

\subsection{Irreducibility}

Having discussed the subobjects of representations (namely, subrepresentation), it is only natural to wish to describe whether a representation ever contains a nontrivial subrepresentation. I say ``nontrivial" because any representation naturally admits two (uninteresting) subrepresentations: the trivial representation and itself.

Akin to the definition of simple groups, where we answer a similar question, we have the following definition that captures this idea.

\begin{boxdefinition}[Irreducibility]
    \letg\ and $\parenth{V, \rho}$ a nonzero representation of $G$. We say $\parenth{V,\rho}$ is irreducible if $V$ contains no proper, nonzero $G$-invariant subspaces.
\end{boxdefinition}
In similar fashion, we say a nonzero representation is reducible if it is not irreducible.

Given that MATH-314 focuses on \textit{finite} groups, the following result is quite useful.

\begin{proposition}
    \letgv. If $G$ is finite and $\parenth{V,\rho}$ is irreducible, then $V$ is finite-dimensional.
\end{proposition}
\begin{proof}
    Since $\parenth{V, \rho}$ is irreducible, in particular, $V \supsetneq \set{0}$---ie, $\exists v \in V$ such that $v \neq 0$. Let $W := \Span{\set{\rhoof{g}\!(v) : g \in G}}$. Since $0 \neq v \in W$, $W$ is a nonzero subspace of $V$. Furthermore, since $G$ is finite, $W$ is finite-dimensional. We show that $W$ is, in fact, $G$-invariant. Then, since $V$ is irreducible, $W$ could not possibly be a proper subspace of $V$, meaning that $W = V$, making $V$ finite-dimensional as well.

    Fix $h \in G$, and consider an arbitrary element $w = \sum_{g \in G} \lambda_g \rho(g)(v) \in W$. Then,
    \begin{align*}
        \rho(h)(w) &= \sum_{g \in G} \lambda_g \rho(h)\!\parenth{\rho(g)(v)} \\
        &= \sum_{g \in G} \lambda_g \parenth{\rho(h) \circ \rho(g)}\!(v) \\
        &= \sum_{g \in G} \lambda_g \rhoof{hg}\!(v) \in W
    \end{align*}
    proving that $W$ is $\rhoof{h}$-invariant for every $h \in G$, making it a $G$-invariant subspace of $V$. Therefore, as argued above, $W = V$, proving that $V$ is finite-dimensional.
\end{proof}

\begin{boxexample}[Simple Examples of Irreducible Representations] \label{Ch1:Eg:Irreds}
    \hfill
    \begin{enumerate}
        \item Any representation of degree $1$ is irreducible.
        \begin{proof}
            A vector space of dimension $1$ has no proper, nonzero subspaces---$G$-invariant or otherwise. Hence, a representation of degree $1$ \textit{must} be irreducible
        \end{proof}
        \item Let $K$ be a field. The trivial embedding $\SL{n,K} \inj \GL{n,K}$ gives an irreducible representation of $\SL{n,K}$ over $K^n$.
        \begin{proof}
            Assume $n > 1$ (else, the result follows from the previous point). For the sake of contradiction, suppose there exists a nonzero, $\SL{n,K}$-invariant subspace $W$ of $K^n$ having dimension $m < n$. Let $\B = \set{e_1, \ldots, e_m}$ be a basis of $W$, extending to a basis $\bar{\B} = \set{e_1, \ldots, e_m, e_{m+1}, \ldots, e_n}$ of $V$. Consider the linear map $T \in \SL{n,K}$ having matrix
            \begin{align*}
                \brac{T}_{\bar{\B}} &=
                \begin{bmatrix}
                    & & & \parenth{-1}^{n+1} \\
                    & & \iddots & \\
                    & -1 & & \\
                    1 & & &
                \end{bmatrix}
            \end{align*}
            with respect to $\bar{B}$. Clearly, $\Tof{e_1} = e_n$, even though $e_1 \in W$ and $e_n \notin W$, contradicting the $\SL{n,K}$-invariance of $W$.
        \end{proof}
    \end{enumerate}
\end{boxexample}
\begin{boxnexample}
    Let $G$ be a finite group and $K$ a field. Consider the regular representation $\parenth{K[G], \rho}$. In the notation of Example \ref{Ch1:Eg:Regular_is_Subrep}, we know that $W := \Span{\sum_{g \in G} e_g}$ is $G$-invariant. If $\abs{G} > 1$, then $W$ is a proper subspace of $K[G]$, as it has dimension $1$ (whereas $K[G]$ has dimension $\abs{G}$). Furthermore, $W$ is nonzero. Hence, $\parenth{K[G], \rho}$ is not irreducible (unless $\abs{G} = 1$, in which case it follows from the first point of Example \ref{Ch1:Eg:Irreds} that $\parenth{K[G], \rho}$ is irreducible).
\end{boxnexample}

We also have the following interesting criterion for irreducibility of representations of finite groups over $\C$.

\begin{lemma}
    \letfg\ and let $\parenth{\C^2, \rho}$ be a representation of $G$ over $\C$. If there exist $g, h \in G$ such that $\rho(g)$ and $\rho(h)$ do not commute, then $\parenth{\C^2, \rho}$ is irreducible.
\end{lemma}
\begin{proof}
    %% (PS 1, Qn 9)
    We argue by contraposition. Suppose that $\parenth{\C^2, \rho}$ is reducible. Let $W \leq \C^2$ denote a proper, nonzero, $G$-invariant subspace. Clearly, $\pdim{W} = 1$. This means that as a $\C$-vector space, $W \cong \C$. Indeed, $\GL{W} \cong \C^{\times}$. Seeing as $\C^{\times}$ is abelian, there cannot possibly exist such a $g$ and $h$. So, if they do exist, then $\parenth{\C^2, \rho}$ must be irreducible.
\end{proof}

\subsection{Lifting Representations}

In this short subsection, we will investigate the lifting property of representations, which will allow us to construct representations of groups from those of their factor groups. For the purposes of this subsection, let $G$ be any group, $N \nsg G$ a normal subgroup of $G$, $H$ the factor group $\quotient{G}{N}$ and $\pi : G \surj H$ the canonical surjection.

\begin{boxdefinition}[Lifting of Representations]
    Let $\Vp$ be a representation of $H$. We define a lift of $\rho$ to be the map $\rho' = \rho \circ \pi : G \to \GL{V}$, where $\pi : G \surj \quotient{G}{N}$ is the canonical surjection. Diagrammatically,
    \begin{cd}
        G \arrow[r, "\rho'"] \arrow[d, "\pi"', two heads] & \GL{V} \\
        H \arrow[ur, "\rho"']
    \end{cd}
\end{boxdefinition}

The lift of a representation has several useful properties.

\begin{proposition}\label{Ch1:Prop:Lifting}
    Let $\Vp$ be a representation of $H$. Then, $\parenth{V, \rho'}$ is a representation of $G$, where $\rho' = \rho \circ \pi$ is a lift of $\rho$. Furthermore, $\Vp$ is irreducible if and only if $\parenth{V, \rho'}$ is.
\end{proposition}
\begin{proof}
    As a composition of homomorphisms, $\rho'$ is a homomorphism from $G$ to $\GL{V}$, making $\parenth{V, \rho'}$ a representation of $G$. We now show that $\Vp$ is irreducible iff $\parenth{V, \rho'}$ is.
    \begin{description}
        \item[$\parenth{\implies}$] Assume $\Vp$ is irreducible. Suppose $V$ contains a proper, nonzero subspace $W$ that is invariant under the action of $G$ via $\rho'$. Then, by the construction of $\rho'$, we know that $W$ must be invariant under the action of $H = \pi(G)$ via $\rho$. However, this would make $W$ $H$-invariant, contradicting the irreducibility of $\Vp$. Hence, $\parenth{V, \rho'}$ cannot contain any proper, nonzero subrepresentations.

        \item[$\parenth{\impliedby}$] Assume $\parenth{V, \rho'}$ is irreducible. Suppose $V$ contains a proper, nonzero subspace $W$ that is invariant under the action of $H$ via $\rho$. Then, $W$ must be invariant under the action of $G$ via $\rho' = \rho \circ \pi$, because $\pi$ maps any $g \in G$ to an element of $H$, making $W$ invariant under the action via $\rho$ of $\pi(g) \in H$. However, this would make $W$ $G$-invariant, contradicting the irreducibility of $\parenth{V, \rho'}$. Hence, $\Vp$ cannot contain any proper, nonzero subrepresentations.
    \end{description}
\end{proof}

We can do even better when $N = \brac{G, G}$, the commutator subgroup of $G$.

\begin{corollary}\label{Ch1:Cor:Degree1Lifts}
    Let $k$ be a field. Then, there is a one-to-one correspondence between degree $1$ representations of $G$ over $k$ and those of $H$.
\end{corollary}
\begin{proof}
    Degree $1$ representations of $G$ and $H$ are precisely group homomorphisms from $G$ and $H$ to $k^{\times}$. Seeing as $k^{\times}$ is abelian and $H$ is the abelianisation of $G$, any homomorphism from $G$ to $k^{\times}$ must factor uniquely through $H$. This gives us a map from degree $1$ representations of $G$ to degree $1$ representations of $H$. The inverse of this map is precisely the lift of any degree $1$ representation of $H$ to one of $G$, as described in Proposition~\ref{Ch1:Prop:Lifting}. These are the maps that take the solid arrows to the dashed arrows (\textcolor{brown}{brown to brown}, \textcolor{pink}{pink to pink}) in the following diagrams:
    \begin{cd*}
    {G} \arrow[d, two heads, gray] \arrow[r, brown] & {k^{\times}} &  &  & {G} \arrow[d, two heads, gray] \arrow[r, dashed, pink] & {k^{\times}} \\
    {H} \arrow[ru, dashed, brown]          &    &  &  & {H} \arrow[ru, pink]          &   
    \end{cd*}
\end{proof}

These properties will prove especially useful when we study character theory. Specifically, they will help us determine the values of irreducible characters of certain large groups using those of much smaller factor groups. For now, though, we close this subsection and move on to studying the compatibility of linear algebraic constructions with representation structures.