\section{Important Definitions}



\subsection{What is a Representation?}

It turns out that representations can be defined quite broadly, sidestepping the geometric niceties (or are they constraints?) of Euclidean spaces.

\begin{boxdefinition}[Group Representation]
    Let $G$ be a group. A representation of $G$ is a pair $\parenth{V, \rho}$ of a vector space $V$ and a group homomorphism $\rho : G \to \GL{V}$.
\end{boxdefinition}

Here, $\GL{V}$ refers to the \textbf{G}eneral \textbf{L}inear group over $V$, consisting of all vector space automorphisms of $V$ equipped with the binary operation of composition.

\begin{definition}[Degree of a Representation]
    \letgv. We define the degree of $V$ to be the dimension of $V$ over its base field.
\end{definition}

There exist innumerable examples of representations throughout mathematics. Below, we give some important ones.

\begin{boxexample}[Important Classes of Representations]
    \hfill
    \begin{enumerate}
        \item \underline{The trivial representation.} \letg\ and $V$ be any vector space. The map $\rho : G \to \GL{V} : g \mapsto \id_{V}$ is a representation.
        
        \item \underline{The sign representation.} Let $G = S_n$, the symmetric group on $n$ elements, and let $V = K$, a field. Then, $\GL{V} = K^{\times}$, the multiplicative group of $K$. Denoting by $\xi$ the canonical map from $\Z$ to $K$, the map
        \begin{align*}
            \rho : G \to \GL{V} : \sigma \mapsto \xiof{\sgn{\sigma}}
        \end{align*}
        is a representation, where $\operatorname{sgn} : G \to \set{-1,1}$ denotes the sign homomorphism.

        \item \underline{Permutation representations.} Let $G$ be a group acting on a finite set $X$, and let $V = K[X]$, the free vector space (over some field $K$) generated by $X$. Consider a $K$-basis $\set{e_x \in V : x \in X}$ of $V$. Then, the map $\rho : G \to \GL{V}$ given by
        \begin{align*}
            \rhoof{g}\!\parenth{e_x} &= e_{g(x)}
        \end{align*}
        is a representation.

        \item \underline{The regular representation.} Let $G$ be a \textit{finite} group. The permutation representation corresponding to the canonical action of $G$ on itself by left-multiplication gives a representation of $G$ over $K[G]$, the free $K$-vector space generated by the set $G$.
    \end{enumerate}
    Note that the map $g \mapsto 0 : G \to \parenth{V \to V}$ is not a representation, because the zero map $0 : V \to V$ is not invertible.
\end{boxexample}

As it turns out, we also have a notion of morphisms of representations.

\subsection{Morphisms of Representations}

\begin{boxdefinition}[Homomorphism of Representations]
    \letg\ and let $\parenth{V,\rho}$ and $\parenth{V',\rho'}$ be two representations of $G$. A homomorphism of representations $T : V \to V$ is a linear map $T : V \to V'$ such that $\forall g \in G$,
    \begin{align*}
        T \circ \rhoof{g} &= \rho'(g) \circ T
    \end{align*}
    or equivalently, the following diagram commutes:
    \begin{cd}
        V \arrow[r, "\rho(g)"] \arrow[d, "T"'] & V \arrow[d, "T"] \\
        V' \arrow[r, "\rho'(g)"'] & V'
        \label{Ch1:eq:cd_rep_map}
    \end{cd}
\end{boxdefinition}

A natural way to define two representations to be equal, or `isomorphic,' is as follows.

\begin{definition}[Equivalence of Representations]
    \letg\ and let $\parenth{V,\rho}$ and $\parenth{V',\rho'}$ be two representations of $G$. We say that $\parenth{V,\rho}$ and $\parenth{V',\rho'}$ are equivalent, denoted $\parenth{V,\rho} \sim \parenth{V',\rho'}$, if there exists a homomorphism $T : \parenth{V,\rho} \to \parenth{V',\rho'}$ that is invertible as a linear map---ie, that gives a linear isomorphism between $V$ and $V'$.
\end{definition}

The point of morphisms of representations is to be able to move from one vector space to another without losing the structural information captured by the representation. This is precisely illustrated in \eqref{Ch1:eq:cd_rep_map}.

\subsection{Subrepresentations}

We have the objects; we have the morphisms. It is only natural to think about what the subobjects would be in the context of group representations.

It turns out that the intuition behind subrepresentations comes entirely from the notion of the invariance of a subspace under the action of a group.

\begin{definition}[$G$-Invariance]
    \letgv. We say that a sub-vector space $W \leq V$ is $G$-invariant if for all $w \in W$ and $g \in G$,
    \begin{align*}
        \rhoof{g}\!(w) \in W
    \end{align*}
    In other words, $W$ is $G$-invariant if $W$ is $\rhoof{g}$-invariant for every $g \in G$.
\end{definition}

One can make the following observation. \letg, $\parenth{V,\rho}$ a representation of $G$, and $W \leq V$ a $G$-invariant subspace. Then, $\forall g \in G$, $\rhoof{g} \in \GL{W}$. That is, $\rhoof{g}$ is a linear automorphism of $W$ whose inverse, $\rhoof{g\inv}$, is \textit{also} a linear automorphism of $W$. This then leads to the following definition of a subrepresentation.

\begin{boxdefinition}[Subrepresentation]
    \letgv. A subrepresentation of $V$ is a pair $\parenth{W, \rhow}$ consisting of a $G$-invariant subspace $W \leq V$ and the map
    \begin{align*}
        \rhow : G \to \GL{W} : g \mapsto \rhoof{g}\!\vert_W
    \end{align*}
\end{boxdefinition}
It is very important to note that the map $\rhow$ is \textit{not actually a restriction of $\rho$ to a specific domain}. Rather, it is a map that restricts the domain of $\rhoof{g}$ for every $g \in G$.

One can also observe easily that a subrepresentation is given uniquely by a $G$-invariant subspace. Hence, we will often abuse notation and not distinguish between the pair $\parenth{W, \rhow}$ (which is actually a representation) and simply $W$ (which is merely a subspace).

\begin{boxexample}
    Let $G$ be a finite group and $K$ a field. Consider the regular representation $\rho : G \to K[G]$. Let $\set{e_g : g \in G}$ denote a basis of $K[G]$. Then, the subspace $W := \Span{\sum_{g \in G} e_g}$ is $G$-invariant.
\end{boxexample}
