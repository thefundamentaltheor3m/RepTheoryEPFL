%% WEEK 1

\chapter{An Introduction to the Theory of Representations of Groups}
\thispagestyle{empty}

As I understand it, the fundamental idea behind Representation Theory is to study the actions of groups on vector spaces. While arbitrary vector spaces over arbitrary fields might not have naturally visualisable geometric properties, representations of groups in the ones that do can greatly illustrate the nature of these groups, especially to individuals like myself who delight in (somewhat literally) \textit{seeing} mathematics come alive.

\begin{wrapfigure}[9]{r}{0.48\linewidth}
    \centering
    \vspace{-2.25em}
    \begin{tikzpicture}
        \drawplane
        \drawsquare{1.5}
        \draw[-{Stealth}, red, dashed] (1.4,1.6) to [bend right = 45] (-1.4,1.6);
        \draw[-{Stealth}, red, dashed] (-1.6,1.4) to [bend right = 45] (-1.6,-1.4);
        \draw[-{Stealth}, red, dashed] (-1.4,-1.6) to [bend right = 45] (1.4,-1.6);
        \draw[-{Stealth}, red, dashed] (1.6,-1.4) to [bend right = 45] (1.6,1.4);
    \end{tikzpicture}
\end{wrapfigure}

A key motivating example in the study of representation theory would be the representations of Dihedral groups over $\R^2$. It is very natural to (at least informally) view the Dihedral group $D_n$ of order $2n$ as the group of symmetries of the regular $n$-gon; in other words, elements of $D_n$ have natural actions on a regular $n$-gon that preserve its structure. For instance, $D_4$ contains an element that rotates a square clockwise by $90^\circ$, an action under which the square is, of course, invariant.

If one were to now plot this square in $\R^2$, then action of the same element on the square can be extended to an orthogonal transformation of $\R^2$ that maps the $x$-axis to the $y$-axis and vice-versa, but in a manner preserving orientation (ie, that \textit{rotates the plane clockwise by $90^\circ$}). In a similar fashion, one can extend the actions of all dihedral groups $D_n$ to actions on the entirety of $\R^2$. More precisely, to every element of a dihedral group, one can ascribe a specific \textit{matrix} that transforms $\R^2$ in a manner preserving the regular $n$-gon.

This motivates the formal definition of a representation.

\section{Important Definitions}



\subsection{What is a Representation?}

It turns out that representations can be defined quite broadly, sidestepping the geometric niceties (or are they constraints?) of Euclidean spaces.

\begin{boxdefinition}[Group Representation]
    Let $G$ be a group. A representation of $G$ is a pair $\parenth{V, \rho}$ of a vector space $V$ and a group homomorphism $\rho : G \to \GL{V}$.
\end{boxdefinition}

Here, $\GL{V}$ refers to the \textbf{G}eneral \textbf{L}inear group over $V$, consisting of all vector space automorphisms of $V$ equipped with the binary operation of composition.

\begin{definition}[Degree of a Representation]
    \letgv. We define the degree of $V$ to be the dimension of $V$ over its base field.
\end{definition}

There exist innumerable examples of representations throughout mathematics. Below, we give some important ones.

\begin{boxexample}[Important Classes of Representations]
    \hfill
    \begin{enumerate}
        \item \underline{The trivial representation.} \letg\ and $V$ be any vector space. The map $\rho : G \to \GL{V} : g \mapsto \id_{V}$ is a representation.
        
        \item \underline{The zero representation.} \letg\ and let $V = \set{0}$ be the zero vector space over an arbitrary field $K$. The trivial representation over $V$ is known as the zero representation.
        
        \item \underline{The sign representation.} Let $G = S_n$, the symmetric group on $n$ elements, and let $V = K$, a field. Then, $\GL{V} = K^{\times}$, the multiplicative group of $K$. Denoting by $\xi$ the canonical morphism from $\Z$ to $K$, the map
        \begin{align*}
            \rho : G \to \GL{V} : \sigma \mapsto \xiof{\sgn{\sigma}}
        \end{align*}
        is a representation, where $\operatorname{sgn} : G \to \set{-1,1}$ denotes the sign homomorphism.

        \item \underline{Permutation representations.} Let $G$ be a group acting on a finite set $X$, and let $V = K[X]$, the free vector space (over some field $K$) generated by $X$. Consider a $K$-basis $\set{e_x \in V : x \in X}$ of $V$. Then, the map $\rho : G \to \GL{V}$ given by
        \begin{align*}
            \rhoof{g}\!\parenth{e_x} &= e_{g(x)}
        \end{align*}
        is a representation.

        \item \underline{The regular representation.} Let $G$ be a \textit{finite} group. The permutation representation corresponding to the canonical action of $G$ on itself by left-multiplication gives a representation of $G$ over $K[G]$, the free vector space generated by $G$ (as a set) over any field $K$.
    \end{enumerate}
\end{boxexample}

\begin{boxnexample}
    \letg\ and let $V$ be a \underline{nonzero} vector space over an arbitrary field. The map $g \mapsto 0 : G \to \parenth{V \to V}$ is not a representation because the zero map $0 : V \to V$ is not invertible.
\end{boxnexample}

A useful perspective to adopt is that a representation is merely an action of a group on a vector space. And, just as faithful actions are an important class of actions, it will, later on, turn out to be important to have a corresponding notion for representations as well.

\begin{definition}[Faithfulness] \label{Ch1:Def:Faithfulness}
    \letgv. We say $\Vp$ is faithful if $\pker{\rho}$ is trivial.
\end{definition}

In the next subsection, we begin to develop the theory of morphisms of representations, which will be crucial to the study of interactions and relationships between representations.

\subsection{Morphisms of Representations}

\begin{boxdefinition}[Homomorphism of Representations]
    \letg\ and let $\parenth{V,\rho}$ and $\parenth{V',\rho'}$ be two representations of $G$. A homomorphism of representations $T : V \to V$ is a linear map $T : V \to V'$ such that $\forall g \in G$,
    \begin{align}
        T \circ \rhoof{g} &= \rho'(g) \circ T
        \label{Ch1:Eq:Def_Rep_Morph}
    \end{align}
    or equivalently, the following diagram commutes:
    \begin{cd}
        V \arrow[r, "\rho(g)"] \arrow[d, "T"'] & V \arrow[d, "T"] \\
        V' \arrow[r, "\rho'(g)"'] & V'
        \label{Ch1:eq:cd_rep_map}
    \end{cd}
    Such a map $T$ is said to be \textit{$G$-linear}.
\end{boxdefinition}

\begin{remark}
    The term $G$-linear comes from the fact that a homomorphism of representations satisfies the property that $\Tof{g(v)} = \gof{\Tof{v}}$, where the notation $\gof{\cdot}$ represents the action of some $g \in G$, encoded by a representation. In this sense, $T$ is somehow ``linear over $G$''.
\end{remark}

A natural way to define two representations to be equal, or `isomorphic,' is as follows.

\begin{definition}[Equivalence of Representations]
    \letg\ and let $\parenth{V,\rho}$ and $\parenth{V',\rho'}$ be two representations of $G$. We say that $\parenth{V,\rho}$ and $\parenth{V',\rho'}$ are equivalent, denoted $\parenth{V,\rho} \sim \parenth{V',\rho'}$, if there exists a homomorphism $T : \parenth{V,\rho} \to \parenth{V',\rho'}$ that is invertible as a linear map---ie, that gives a linear isomorphism between $V$ and $V'$.
\end{definition}

Representations of the same group over the same vector space need not be equivalent.

\begin{boxexample}[Non-Equivalent Representations of the Klein $4$-Group]
    Let $G = C_2 \times C_2$ be the Klein $4$-group (where $C_2 = \cycl{x}$ is the cyclic group of order $2$). Let $\alpha = \parenth{x, 1}$ and $\beta = \parenth{1, x}$. Together, they generate $G$. \\

    Now, let $K$ be a field. Consider a degree $1$ representation $\rho : G \to K^\times$. We know that $\rhoof{G}$ must be a subgroup of $K^\times$ such that $\abs{\rhoof{G}} \in \set{1, 2, 4}$. If $\pchar{K} = 2$, then $\rho$ must be the trivial representation, since $2 \nmid \abs{K^\times}$. Else, all four maps $\rho$ satisfying
    \begin{align*}
        \parenth{\rhoof{\alpha}, \rhoof{\beta}} = \parenth{\pm 1, \pm 1}
    \end{align*}
    give \textit{non-equivalent} representations of $G$ in $K^\times$. In particular, we see the non-equivalence because $K^\times$ is commutative.
\end{boxexample}

The point of morphisms of representations is to be able to move from one vector space to another without losing the structural information captured by the representation. This is precisely illustrated in \eqref{Ch1:eq:cd_rep_map}.

\begin{boxexample}[Representations of Cyclic Groups over $\R^2$ and $\R^3$] \label{Ch1:Eg:Cyclic_Subrep}
    Consider the cyclic group $C_n = \cycl{g}$ of order $n$. Let $V = \R^2, V' = \R^3$. Together with the respective maps
    \begin{align*}
        \rho : G \to \GL{\R^2} &: g^m \mapsto \begin{bmatrix}
            \pcos{2\pi/m} & -\psin{2\pi/m} \\
            \psin{2\pi/m} & \pcos{2\pi/m}
        \end{bmatrix} \\
        \rho' : G \to \GL{\R^3} &: g^m \mapsto \begin{bmatrix}
            \pcos{2\pi/m} & -\psin{2\pi/m} & 0 \\
            \psin{2\pi/m} & \pcos{2\pi/m} & 0 \\
            0 & 0 & 1
        \end{bmatrix}
    \end{align*}
    they give representations of $C_n$. Consider now the inclusion $T : \R^2 \to \R^3$ whose matrix with respect to the standard bases of $\R^2$ and $\R^3$ is $\begin{bmatrix}
        1 & 0 \\ 0 & 1 \\ 0 & 0
    \end{bmatrix}$. One can see that $T$ gives a map from $\parenth{V, \rho}$ to $\parenth{V', \rho'}$. Indeed, the corestriction of $T$ to its image is a linear isomorphism, which gives an equivalence between $\parenth{V, \rho}$ and $\parenth{T(V), \rho}$, where we restrict the domains of each $\rhoof{g^m}$ to $T(V)$.
\end{boxexample}
The above example leads to an interesting question. Can we think of one representation as being ``contained'' in another?

It turns out that we can.

\subsection{Subrepresentations} \label{Ch1:Subsec:Subreps}

We have the objects; we have the morphisms. It is only natural to think about what the subobjects would be in the context of group representations. And if Example \ref{Ch1:Eg:Cyclic_Subrep} is any indication, they involve something more than just an inclusion. There is some structural property of a sub-vector space of a representation that makes it \textit{compatible} with the representation structure. In the case of Example \ref{Ch1:Eg:Cyclic_Subrep}, for instance, this is the fact that the representation $\rho'$ acted only ``horizontally"--ie, ``parallel" to the subspace $T(V)$.

More generally, it turns out that the property we really require a subspace to have in order to be `compatible' with the representation structure is the following.

\begin{boxdefinition}[$G$-Invariance]
    \letgv. We say that a sub-vector space $W \leq V$ is $G$-invariant if for all $w \in W$ and $g \in G$,
    \begin{align*}
        \rhoof{g}\!(w) \in W
    \end{align*}
    In other words, $W$ is $G$-invariant if $W$ is $\rhoof{g}$-invariant for every $g \in G$.
\end{boxdefinition}

One can make the following observation. \letg, $\parenth{V,\rho}$ a representation of $G$, and $W \leq V$ a $G$-invariant subspace. Then, $\forall g \in G$, $\rhoof{g} \in \GL{W}$. That is, $\rhoof{g}$ is a linear automorphism of $W$ whose inverse, $\rhoof{g\inv}$, is \textit{also} a linear automorphism of $W$. This then leads to the following definition of a subrepresentation.

\begin{boxdefinition}[Subrepresentation]
    \letgv. A subrepresentation of $V$ is a pair $\parenth{W, \rhow}$ consisting of a $G$-invariant subspace $W \leq V$ and the map
    \begin{align*}
        \rhow : G \to \GL{W} : g \mapsto \rhoof{g}\!\vert_W
    \end{align*}
\end{boxdefinition}
It is very important to note that the map $\rhow$ is \textit{not actually a restriction of $\rho$ to a specific domain}. Rather, it is a map that restricts the domain of $\rhoof{g}$ for every $g \in G$.

One can also observe easily that a subrepresentation is given uniquely by a $G$-invariant subspace. Hence, we will often abuse notation and not distinguish between the pair $\parenth{W, \rhow}$ (which is actually a representation) and simply $W$ (which is merely a subspace).

\begin{boxexample} \label{Ch1:Eg:Regular_is_Subrep}
    Let $G$ be a finite group and $K$ a field. Consider the regular representation $\rho : G \to K[G]$. Let $\set{e_g : g \in G}$ denote a basis of $K[G]$. Then, the subspace $W := \Span{\sum_{g \in G} e_g}$ is $G$-invariant.
\end{boxexample}

It turns out that morphisms of representations also give us subrepresentations.

\begin{proposition}\label{Ch1:Prop:ker_im_subreps}
    \letgv. Let $T : \Vp \to \Vp$ be a homomorphism of representations. Then, the subspaces $\pker{T}$ and $\pim{T}$ of $V$ are $G$-invariant.
\end{proposition}
\begin{proof}
    Fix $g \in G$ and $v \in \pker{T}$. We know $\Tof{\rho(g)(v)} = \rho(g)\!\parenth{\Tof{v}}$. Since $\Tof{v} = 0$, $\Tof{\rho(g)(v)} = 0$. Hence, $\rho(g)(v) \in \pker{T}$, proving that $\pker{T}$ is $G$-invariant.

    Now, fix $w \in \pim{T}$. Then, $w = \Tof{u}$ for some $u \in V$. Clearly, $\rho(g)(w) = \rho(g)\!\parenth{\Tof{u}} = \Tof{\rho(g)(u)} \in \pim{T}$, proving that $\pim{T}$ is $G$-invariant as well.
\end{proof}

\subsection{Irreducibility}

Having discussed the subobjects of representations (namely, subrepresentation), it is only natural to wish to describe whether a representation ever contains a nontrivial subrepresentation. I say ``nontrivial" because any representation naturally admits two (uninteresting) subrepresentations: the trivial representation and itself.

Akin to the definition of simple groups, where we answer a similar question, we have the following definition that captures this idea.

\begin{boxdefinition}[Irreducibility]
    \letg\ and $\parenth{V, \rho}$ a nonzero representation of $G$. We say $\parenth{V,\rho}$ is irreducible if $V$ contains no proper, nonzero $G$-invariant subspaces.
\end{boxdefinition}
In similar fashion, we say a nonzero representation is reducible if it is not irreducible.

Given that MATH-314 focuses on \textit{finite} groups, the following result is quite useful.

\begin{proposition}
    \letgv. If $G$ is finite and $\parenth{V,\rho}$ is irreducible, then $V$ is finite-dimensional.
\end{proposition}
\begin{proof}
    Since $\parenth{V, \rho}$ is irreducible, in particular, $V \supsetneq \set{0}$---ie, $\exists v \in V$ such that $v \neq 0$. Let $W := \Span{\set{\rhoof{g}\!(v) : g \in G}}$. Since $0 \neq v \in W$, $W$ is a nonzero subspace of $V$. Furthermore, since $G$ is finite, $W$ is finite-dimensional. We show that $W$ is, in fact, $G$-invariant. Then, since $V$ is irreducible, $W$ could not possibly be a proper subspace of $V$, meaning that $W = V$, making $V$ finite-dimensional as well.

    Fix $h \in G$, and consider an arbitrary element $w = \sum_{g \in G} \lambda_g \rho(g)(v) \in W$. Then,
    \begin{align*}
        \rho(h)(w) &= \sum_{g \in G} \lambda_g \rho(h)\!\parenth{\rho(g)(v)} \\
        &= \sum_{g \in G} \lambda_g \parenth{\rho(h) \circ \rho(g)}\!(v) \\
        &= \sum_{g \in G} \lambda_g \rhoof{hg}\!(v) \in W
    \end{align*}
    proving that $W$ is $\rhoof{h}$-invariant for every $h \in G$, making it a $G$-invariant subspace of $V$. Therefore, as argued above, $W = V$, proving that $V$ is finite-dimensional.
\end{proof}

\begin{boxexample}[Simple Examples of Irreducible Representations] \label{Ch1:Eg:Irreds}
    \hfill
    \begin{enumerate}
        \item Any representation of degree $1$ is irreducible.
        \begin{proof}
            A vector space of dimension $1$ has no proper, nonzero subspaces---$G$-invariant or otherwise. Hence, a representation of degree $1$ \textit{must} be irreducible
        \end{proof}
        \item Let $K$ be a field. The trivial embedding $\SL{n,K} \inj \GL{n,K}$ gives an irreducible representation of $\SL{n,K}$ over $K^n$.
        \begin{proof}
            Assume $n > 1$ (else, the result follows from the previous point). For the sake of contradiction, suppose there exists a nonzero, $\SL{n,K}$-invariant subspace $W$ of $K^n$ having dimension $m < n$. Let $\B = \set{e_1, \ldots, e_m}$ be a basis of $W$, extending to a basis $\bar{\B} = \set{e_1, \ldots, e_m, e_{m+1}, \ldots, e_n}$ of $V$. Consider the linear map $T \in \SL{n,K}$ having matrix
            \begin{align*}
                \brac{T}_{\bar{\B}} &=
                \begin{bmatrix}
                    & & & \parenth{-1}^{n+1} \\
                    & & \iddots & \\
                    & -1 & & \\
                    1 & & &
                \end{bmatrix}
            \end{align*}
            with respect to $\bar{B}$. Clearly, $\Tof{e_1} = e_n$, even though $e_1 \in W$ and $e_n \notin W$, contradicting the $\SL{n,K}$-invariance of $W$.
        \end{proof}
    \end{enumerate}
\end{boxexample}
\begin{boxnexample}
    Let $G$ be a finite group and $K$ a field. Consider the regular representation $\parenth{K[G], \rho}$. In the notation of Example \ref{Ch1:Eg:Regular_is_Subrep}, we know that $W := \Span{\sum_{g \in G} e_g}$ is $G$-invariant. If $\abs{G} > 1$, then $W$ is a proper subspace of $K[G]$, as it has dimension $1$ (whereas $K[G]$ has dimension $\abs{G}$). Furthermore, $W$ is nonzero. Hence, $\parenth{K[G], \rho}$ is not irreducible (unless $\abs{G} = 1$, in which case it follows from the first point of Example \ref{Ch1:Eg:Irreds} that $\parenth{K[G], \rho}$ is irreducible).
\end{boxnexample}

We also have the following interesting criterion for irreducibility of representations of finite groups over $\C$.

\begin{lemma}
    \letfg\ and let $\parenth{\C^2, \rho}$ be a representation of $G$ over $\C$. If there exist $g, h \in G$ such that $\rho(g)$ and $\rho(h)$ do not commute, then $\parenth{\C^2, \rho}$ is irreducible.
\end{lemma}
\begin{proof}
    %% (PS 1, Qn 9)
    We argue by contraposition. Suppose that $\parenth{\C^2, \rho}$ is reducible. Let $W \leq \C^2$ denote a proper, nonzero, $G$-invariant subspace. Clearly, $\pdim{W} = 1$. This means that as a $\C$-vector space, $W \cong \C$. Indeed, $\GL{W} \cong \C^{\times}$. Seeing as $\C^{\times}$ is abelian, there cannot possibly exist such a $g$ and $h$. So, if they do exist, then $\parenth{\C^2, \rho}$ must be irreducible.
\end{proof}

\subsection{Lifting Representations}

In this short subsection, we will investigate the lifting property of representations, which will allow us to construct representations of groups from those of their factor groups. For the purposes of this subsection, let $G$ be any group, $N \nsg G$ a normal subgroup of $G$, $H$ the factor group $\quotient{G}{N}$ and $\pi : G \surj H$ the canonical surjection.

\begin{boxdefinition}[Lifting of Representations]
    Let $\Vp$ be a representation of $H$. We define a lift of $\rho$ to be the map $\rho' = \rho \circ \pi : G \to \GL{V}$, where $\pi : G \surj \quotient{G}{N}$ is the canonical surjection. Diagrammatically,
    \begin{cd}
        G \arrow[r, "\rho'"] \arrow[d, "\pi"', two heads] & \GL{V} \\
        H \arrow[ur, "\rho"']
    \end{cd}
\end{boxdefinition}

The lift of a representation has several useful properties.

\begin{proposition}\label{Ch1:Prop:Lifting}
    Let $\Vp$ be a representation of $H$. Then, $\parenth{V, \rho'}$ is a representation of $G$, where $\rho' = \rho \circ \pi$ is a lift of $\rho$. Furthermore, $\Vp$ is irreducible if and only if $\parenth{V, \rho'}$ is.
\end{proposition}
\begin{proof}
    As a composition of homomorphisms, $\rho'$ is a homomorphism from $G$ to $\GL{V}$, making $\parenth{V, \rho'}$ a representation of $G$. We now show that $\Vp$ is irreducible iff $\parenth{V, \rho'}$ is.
    \begin{description}
        \item[$\parenth{\implies}$] Assume $\Vp$ is irreducible. Suppose $V$ contains a proper, nonzero subspace $W$ that is invariant under the action of $G$ via $\rho'$. Then, by the construction of $\rho'$, we know that $W$ must be invariant under the action of $H = \pi(G)$ via $\rho$. However, this would make $W$ $H$-invariant, contradicting the irreducibility of $\Vp$. Hence, $\parenth{V, \rho'}$ cannot contain any proper, nonzero subrepresentations.

        \item[$\parenth{\impliedby}$] Assume $\parenth{V, \rho'}$ is irreducible. Suppose $V$ contains a proper, nonzero subspace $W$ that is invariant under the action of $H$ via $\rho$. Then, $W$ must be invariant under the action of $G$ via $\rho' = \rho \circ \pi$, because $\pi$ maps any $g \in G$ to an element of $H$, making $W$ invariant under the action via $\rho$ of $\pi(g) \in H$. However, this would make $W$ $G$-invariant, contradicting the irreducibility of $\parenth{V, \rho'}$. Hence, $\Vp$ cannot contain any proper, nonzero subrepresentations.
    \end{description}
\end{proof}

We can do even better when $N = \brac{G, G}$, the commutator subgroup of $G$.

\begin{corollary}\label{Ch1:Cor:Degree1Lifts}
    Let $k$ be a field. Then, there is a one-to-one correspondence between degree $1$ representations of $G$ over $k$ and those of $H$.
\end{corollary}
\begin{proof}
    Degree $1$ representations of $G$ and $H$ are precisely group homomorphisms from $G$ and $H$ to $k^{\times}$. Seeing as $k^{\times}$ is abelian and $H$ is the abelianisation of $G$, any homomorphism from $G$ to $k^{\times}$ must factor uniquely through $H$. This gives us a map from degree $1$ representations of $G$ to degree $1$ representations of $H$. The inverse of this map is precisely the lift of any degree $1$ representation of $H$ to one of $G$, as described in Proposition~\ref{Ch1:Prop:Lifting}. These are the maps that take the solid arrows to the dashed arrows (\textcolor{brown}{brown to brown}, \textcolor{pink}{pink to pink}) in the following diagrams:
    \begin{cd*}
    {G} \arrow[d, two heads, gray] \arrow[r, brown] & {k^{\times}} &  &  & {G} \arrow[d, two heads, gray] \arrow[r, dashed, pink] & {k^{\times}} \\
    {H} \arrow[ru, dashed, brown]          &    &  &  & {H} \arrow[ru, pink]          &   
    \end{cd*}
\end{proof}

These properties will prove especially useful when we study character theory. Specifically, they will help us determine the values of irreducible characters of certain large groups using those of much smaller factor groups. For now, though, we close this subsection and move on to studying the compatibility of linear algebraic constructions with representation structures.
\section{Invariant Constructions}

In this section, we briefly examine how ordinary linear algebraic constructions can interact with representations. We are particularly interested in the notion of \textit{invariance}, wherein a construction respects the structure of the representation(s) involved.

\subsection{Direct Sums of Representations}

The most elementary operation we can think about when we have two objects is \textit{putting them together}. One of the most meaningful ways of doing so in the context of linear algebra is the direct sum of two vector spaces. It turns out that this extends rather naturally to representations.

\begin{boxdefinition}[The Direct Sum of Two Representations]
    \letgvv. We define the direct sum of $\parenth{V,\rho}$ and $\parenth{V', \rho'}$ to be the pair $\parenth{V \+ V', \rho \+ \rho'}$, where $V \+ V'$ is the direct sum of $V$ and $V'$ as vector spaces and $\rho \+ \rho' : G \to \GL{V \+ V'}$ is the map given by
    \begin{align*}
        \parenth{\rho \+ \rho'}\!\parenth{g}\!\parenth{v \+ v'} &= \rho(g)(v) \+ \rho'(g)(v')
    \end{align*}
    for all $g \in G$.
\end{boxdefinition}

\begin{proposition}
    \letgvv.
    \begin{enumerate}[label = \normalfont \arabic*., noitemsep]
        \item The direct sum $\parenth{V \+ V', \rho \+ \rho'}$ of $\parenth{V,\rho}$ and $\parenth{V', \rho'}$ is, indeed, a representation of $G$.
        \item $V$ and $V'$ are $G$-invariant subspaces\footnote{Technically, isomorphic to the subspaces $V \+ \set{0}$ and $\set{0} \+ V'$, but we overlook such distinctions.} of $V \+ V'$.
    \end{enumerate}
\end{proposition}
\begin{proof}
    \hfill
    \begin{enumerate}[noitemsep]
        \item Fix $g,h \in G$. For all $v \+ v' \in V \+ V'$,
        \begin{align*}
            \parenth{\rho \+ \rho'}\!(gh)(v \+ v') &= \rho(gh)(v) \+ \rho'(gh)(v') \\
            &= \rho(g)\!\parenth{\rho(h)(v)} \+ \rho'(g)\!\parenth{\rho'(h)(v')} \\
            &= \parenth{\rho \+ \rho'}\!(g)\!\parenth{\parenth{\rho \+ \rho'}\!(h)(v \+ v')}
        \end{align*}
        proving that $\rho \+ \rho'$ is multiplicative. Then, for any $g \in G$, $\parenth{\rho \+ \rho'}\!(g)$ has inverse $\parenth{\rho \+ \rho'}\!\parenth{g\inv}$. Hence, $\rho \+ \rho'$ is a homomorphism from $G$ to $\GL{V \+ V'}$.

        \item Fix $g \in G$ and $v \in V$. Clearly, $\parenth{\rho \+ \rho'}\!(g)(v) = \rho(g)(v)$. Since $\rho(g) \in \GL{V}$, it follows that $\rho(g)(v) \in V$. The proof that $V'$ is $G$-invariant is identical.
    \end{enumerate}
\end{proof}

The above proposition gives us another reason to consider the direct sum to be an ``invariant'' construction: while it enriches both the vector space structure and the representation structure of a summand by adding another representation into the mix, it does not take anything away from the constructions that already exist.

With direct sums, we also have similar notions to reducibility. These are given by the following.

\begin{boxdefinition}[Complete Reducibility]
    A representation is said to be completely reducible if it is expressible as a direct sum of irreducible representations.
\end{boxdefinition}

\begin{boxdefinition}[Indecomposability]
    A nonzero representation is said to be indecomposable if it is inexpressible as a direct sum of two proper, nonzero subrepresentations.
\end{boxdefinition}
Nonzero representations that are not indecomposable are said to be decomposable.

\subsection{Complementary Subrepresentations and Maschke's Theorem}

It is a well-known fact from Linear Algebra that for any finite-dimensional vector space $V$, for any subspace $W \leq V$, there exists a \textit{complementary} subspace $W' \leq V$ such that $W \+ W' = V$. Over Euclidean spaces, for example, we have the very important notion of \textit{orthogonal} complements.

We can define a similar notion for representations, too.
\begin{boxdefinition}[Complementary Subrepresentation]
    \letgv. Let $\parenth{W, \rho\vert_W}$ be a subrepresentation of $\parenth{V, \rho}$. A complementary subrepresentation of $\parenth{W, \rho\vert_W}$ is a subrepresentation $\parenth{U, \rho\vert_U}$ such that $V = U \+ W$.
\end{boxdefinition}
This notion of complementarity is, indeed, compatible with the notion of direct sums of representations.
\begin{lemma}
    \letgv. Let $\parenth{W, \rho\vert_W}$ and $\parenth{U, \rho\vert_U}$ be complementary subrepresentations. Then, their direct sum $\parenth{V, \rho\vert_W \+ \rho\vert_U}$ is equivalent to $\parenth{V, \rho}$ as a representation of $G$.
\end{lemma}
\begin{proof}
    It suffices to show that $\rho = \rho\vert_W \+ \rho\vert_U$. Then, the identity map would give an equivalence of representations. Indeed, every $v \in V$ is expressible uniquely as a direct sum $w \+ u$ for some $w \in W$ and $u \in U$. So, for all $g \in G$,
    \begin{align*}
        \rho(g)(v) &= \rho(g)(w \+ u) \\
        &= \rho(g)(w) \+ \rho(g)(u) \\
        &= \rho\vert_W(g)(w) \+ \rho\vert_U(g)(u) \\
        &= \parenth{\rho\vert_W \+ \rho\vert_U}\!(g)(w \+ u)
    \end{align*}
    where the sum in the second equality is direct because $W$ and $U$ are $\rho(g)$-invariant.
\end{proof}

Given the theme of this section---namely, understanding the compatibility of ordinary linear-algebraic constructions with representation structures---one might wonder under what conditions (if any) we have the existence of a complementary subrepresentations. The answer lies in Maschke's Theorem, which is the first major result of the course.

\begin{boxtheorem}[Maschke's Theorem] \label{SP:Thm:Maschke}
    \letfg, $K$ a field such that $\pchar{K} \nmid \abs{G}$, and $\Vp$ a representation of $G$ over $K$. Then, any subrepresentation of $V$ admits a complementary subrepresentation.
\end{boxtheorem}


\subsection{The $G$-Invariant Inner-Product}



\section{Group Algebras and Modules}

In this section, we study an important class of field algebras, namely, group algebras, and an important class of modules over said algebras, namely, group modules.

\subsection{Preliminaries}

\begin{boxdefinition}[Group Algebra]
    \letfg\ and let $K$ be a field. The group algebra $KG$ is the $K$-algebra obtained by endowing the free vector space $K[G]$ generated by $G$ (as a set) with the multiplication
    \begin{align*}
        \parenth{\sum_{g \in G} \alpha_g e_g} \cdot \parenth{\sum_{g \in G} \beta_g e_g} &:= \sum_{g \in G} \sum_{h \in G} \alpha_{g} \beta_h e_{gh}
    \end{align*}
\end{boxdefinition}
\begin{remark} \hfill
    \begin{enumerate}[noitemsep]
        \item For ease of notation, we often denote elements $e_g$ of the basis as simply $g$.
        \item It is easy to verify that $KG$ is, indeed, a $K$-algebra, with the multiplicative identity given by $e_1$ (where $1 \in G$ is the identity).
        \item The map $g \mapsto e_g : G \to KG$ gives a trivial embedding of $G$ in $KG$.
    \end{enumerate}
\end{remark}

We have a similar notion of group modules.

\begin{boxdefinition}[Group Module] \label{Ch1:Def:KG-Module}
    \letg\ and let $V$ be a vector space over a field $K$. We say that $V$ is a $KG$-module if we can define a multiplication $g \cdot v$ for some $g \in G$ and $v \in V$ that satisfies the following conditions for all $u, v \in V$, $g, h \in G$ and $\lambda \in K$:
    \begin{enumerate}[noitemsep]
        \item $g \cdot v \in V$
        \item $\parenth{gh} \cdot v = g \cdot \parenth{h \cdot v}$
        \item $1 \cdot v = v$
        \item $g \cdot \parenth{\lambda v} = \lambda \parenth{g \cdot v}$
        \item $g \cdot \parenth{u + v} = g \cdot u + g \cdot v$
    \end{enumerate}
\end{boxdefinition}

Note that a $KG$-module is, indeed, a module over $KG$.

\begin{proposition}
    \letg\ and let $V$ be a vector space over a field $K$. If $V$ is a $KG$-module with multiplication $\cdot$ (as per Definition \ref{Ch1:Def:KG-Module}), then for $v \in V$, the multiplication
    \begin{align*}
        \parenth{\sum_{g \in G} \lambda_g e_g} \cdot v &:= \sum_{g \in G} \lambda_g \parenth{g \cdot v}
    \end{align*}
    endows $V$ with a module structure over $K[G]$.
\end{proposition}

Furthermore, it turns out that we can move from modules to representations and vice-versa quite easily.

\begin{proposition}
    \letg\ and let $V$ be a vector space over a field $K$.
    \begin{enumerate}[label = \normalfont \arabic*.]
        \item If $\rho : G \to \GL{V}$ gives a representation of $G$, then $V$ is a $KG$-module with multiplication given by $g \cdot v = \rhoff{g}{v}$ for all $g \in G$ and $v \in V$.
        \item If $V$ is a $KG$-module with multiplication $\cdot$, the map $\rho : G \to \GL{V}$ given by $\rhoff{g}{v} := g \cdot v$ is a representation.
    \end{enumerate}
\end{proposition}

The proofs of the above propositions are trivial and merely involve manually checking several basic conditions. Hence, we omit them.

We now give a basic `dictionary' of sorts to go back and forth between the language of group modules and that of representations:

\begin{table}[!h]
    \centering
    \begin{tabular}{c|c}
        \textbf{$KG$-Modules} & \textbf{Representations} \\ \hline
        Simple & Irreducible \\
        Semi-Simple & Completely Irreducible \\
        Submodule & Subrepresentation \\
        Viewing $KG$ as a $KG$-Module & The Regular Representation \\
        Isomorphism & Equivalence of Representations \\
        Dimension (as a $K$-vector space) & Degree
    \end{tabular}
\end{table}

We illustrate the above equivalence by stating Maschke's Theorem in the language of $KG$-Modules.

\begin{lemma}[Maschke's Theorem, Module Version]
    \letfg, $K$ a field whose characteristic does not divide the order of $G$. Then, any $KG$-Module $V$ is semi-simple.
\end{lemma}

\subsection{Schur's Lemmas}

In this subsection, we explore several versions of an important result by Schur. We begin by stating it in its most general form.

\begin{theorem}[Schur's Lemmas for Rings]
    Let $A$ be a ring and let $S, T$ be simple $A$-modules.
    \begin{enumerate}[label = \normalfont \arabic*., noitemsep]
        \item If $S$ and $T$ are non-isomorphic, then $\Hom_A(S, T) = \set{0}$.
        \item If $S$ and $T$ are isomorphic, then $\Hom_A(S,T)$ is a division ring.
    \end{enumerate}
\end{theorem}
\begin{proof}
    We rely on the fact that for all $\phi \in \Hom_A(S, T)$, $\pker{\phi} \leq S$ and $\pim{\phi} \leq T$.
    \begin{enumerate}
        \item Let $S$ and $T$ be non-isomorphic. Fix $\phi \in \Hom_A(S,T)$. Since $S$ is simple, we must have that $\pker{\phi} \in \set{\set{0}, S}$. If $\pker{\phi} = \set{0}$, then $\pim{\phi} = T$, meaning $S \cong T$, a contradiction.
        \item Let $\phi \in \Hom_A(S,T) \setminus \set{0}$. Then, $\pker{\phi} \neq S$, meaning that $\pker{\phi} = \set{0}$. Then, $\pim{\phi} = T$, making $\phi$ an isomorphism. In particular, this means that $\phi$ admits an inverse, making $\Hom_A(S,T)$ a division ring.
    \end{enumerate}
\end{proof}

It turns out we can do a bit better when dealing with a specific class of rings, namely, algebras over fields.

\begin{theorem}[Schur's Lemmas for Algebras]
    Let $K$ be an algebraically closed field and $A$ a $K$-algebra. Let $S$ and $T$ be simple $A$-modules.
    \begin{enumerate}[label = \normalfont \arabic*., noitemsep]
        \item If $S \not\cong T$, then $\Hom_A(S, T) = \set{0}$.
        \item If $S \cong T$, then $K \cong \Hom_A(S, T)$ via the map $\alpha \mapsto \alpha \cdot \id$.
    \end{enumerate}
\end{theorem}
\begin{proof}
    \hfill
    \begin{enumerate}
        \item As before.
        \item We do not distinguish $S$ and $T$ in this proof.
        
        Fix $\phi \in \Hom_A(S,S)$. Then, $\phi$ can be viewed as an element of $\Mn{n}{K}$, where $n = \pdim{S}$. Since $K$ is algebraically closed, $\phi$ admits an eigenvalue $\lambda \in K$. Now, consider the map $\phi - \lambda \id \in \Hom_A(S,S)$. Clearly, $\pker{\phi - \lambda \id} \neq \set{0}$, since it contains all eigenvectors with eigenvalue $\lambda$. Since $S$ is simple, it must be that $\pker{\phi - \lambda \id} = S$, meaning $\phi - \lambda \id = 0$. In other words, $\phi = \lambda \id$.  % Explain the logic of this in more detail.
    \end{enumerate}
\end{proof}

We also have a converse when working with algebras.

\begin{theorem}[Converse of Schur's Lemma for Algebras]
    Let $K$ be a field, $A$ a $K$-algebra and $M$ a completely reducible $A$-module. If $\Hom_A(M,M) = K$, then $M$ is simple.
\end{theorem}
\begin{proof}
    \verb|sorry|
\end{proof}

\begin{boxtheorem}[Schur's Lemmas for Finite Groups, over $\C$] \label{SP:Thm:Schur_fin_G_over_C}
    \letfg\ and let $S$ and $T$ be simple $\C G$ modules that are finite-dimensional (as vector spaces) over $K$, with associated representations $\rho_S : G \to \GL{S}$ and $\rho_T : G \to \GL{T}$.
    \begin{enumerate}[label = \normalfont \arabic*.]
        \item If $S \not\cong T$, then for all $\C$-linear maps $f : S \to T$, the map
        \begin{align*}
            \hat{f} := \frac{1}{\abs{G}} \sum_{g \in G} \rho_T(g) \circ f \circ \rho_S\!\parenth{g\inv}
        \end{align*}
        is identically zero.

        \item If $S \cong T$, then for all $\C$-linear maps $f : S \to T$, we have
        \begin{align*}
            \hat{f} &:= \frac{1}{\abs{G}} \sum_{g \in G} \rho_T(g) \circ f \circ \rho_S\!\parenth{g\inv} \\
            &= \frac{1}{\pdim{S}} \Tr{f} \cdot \id_S
        \end{align*}
    \end{enumerate}
\end{boxtheorem}
\begin{proof}
    Let $f : S \to T$ be $\C$-linear. We show that $\hat{f} \in \Hom_{\C G}(S, T)$: for all $h \in G$,
    \begin{align*}
        \rho_T(h) \circ \hat{f} &= \rho_T(h)\!\parenth{\frac{1}{\abs{G}} \sum_{g \in G} \rho_T(g) \circ f \circ \rho_S\!\parenth{g\inv}} \\
        &= \frac{1}{\abs{G}} \sum_{g \in G} \rho_T(hg) \circ f \circ \rho_S\!\parenth{g\inv} \circ \rho_S\!\parenth{h\inv} \circ \rho_S(h) \\
        &= \frac{1}{\abs{G}} \sum_{g \in G} \parenth{\rho_T(hg) \circ f \circ \rho_S\!\parenth{g\inv h\inv}} \circ \circ \rho_S(h) \\
        &= \hat{f} \circ \rho_S(h)
    \end{align*}
    proving that $\hat{f}$ is, indeed, a homomorphism of $\C G$-modules.
\end{proof}

It turns out that Schur's Lemmas are powerful tools in the study of certain classes of representations. We investigate one such class in the next subsection.

\subsection{Representations of Finite Abelian Groups over $\C$}

It is natural to wonder what the purpose was of studying group algebras and modules. It turns out that one of the reasons the correspondence between representations and group modules is so powerful is that it allows the application of Schur's Lemmas to representation theoretic problems. For instance, in the following Lemma, we classify all irreducible representations of finite abelian groups over $\C$.

\begin{lemma} \label{Ch1:Lem:Fin_Ab_over_C}
    Let $G$ be a fininte abelian group. Then, all irreducible $\C G$-modules are of dimension $1$. Equivalently, all irreducible representations of $G$ over $\C$ are of degree $1$.
\end{lemma}
\begin{proof}
    Let $V$ be an irreducible $\C G$-module. Since $G$ is abelian, for all $g,h \in G$ and $v \in V$, $\parenth{gh} \cdot v = \parenth{hg} \cdot v$. Therefore, for some fixed $h \in G$, the following map is $\C G$-linear:
    \begin{align*}
        \phi_h : V \to V : v \mapsto h \cdot v
    \end{align*}
    By Theorem \ref{SP:Thm:Schur_fin_G_over_C}, we know that $\exists \lambda_h \in \C$ such that $\widehat{\phi_h} = \phi_h = \lambda_h \cdot \id_V$. Hence, any subspace of $V$ must be a $\C G$-submodule. But, since $V$ is irreducible, $V$ cannot admit any nonzero, proper $\C G$-submodules unless $V$ is of ($\C$-)dimension $1$.
\end{proof}

\begin{boxexample}
    Let $G = C_n = \cycl{a}$ be the cyclic group of order $n$. Then, there are precisely $n$ irreducible representations of $G$ over $\C$.
    \begin{proof}
        Let $\rho : \C \to \C^\times$ be an irreducible representation of $G$ over $\C$. Let $x := \rho(a)$. It must be that $x^n = 1$, making $x$ an $n$th root of unity. In other words, $\exists 1 \leq k \leq n$ such that $x = e^{\frac{2\pi i}{k}}$. Therefore, there are precisely $n$ possible choices of $x$. Each choice corresponds to a different possible representation of $G$ over $\C$.
    \end{proof}
\end{boxexample}

We have below a very useful application of Lemma \ref{Ch1:Lem:Fin_Ab_over_C} to the study of representations over $\C$ of arbitrary finite groups.

\begin{proposition}
    \letfgv\ over $\C$. For all $g \in G$, there is a basis of $V$ with respect to which $\rho(g)$ has matrix $\diag{\eps_1, \cdots, \eps_n}$, with $\eps_i^{\ord{g}} = 1$ for all $1 \leq i \leq n$.
\end{proposition}
\begin{proof}
    Fix $g \in G$, and consider the representation $\rho' : \cycl{g} \to \GL{V}$ given by $\rho' = \rho\vert_{\cycl{g}}$. Then, $\rho'$ is a representation of a finite abelian group.

    By Maschke's Theorem, $\rho' = \sigma_1 \+ \cdots \+ \sigma_k$ for irreducible subrepresentations $\sigma_1, \ldots, \sigma_m$ of $\cycl{g}$. By Lemma \ref{Ch1:Lem:Fin_Ab_over_C}, we know that $\pdeg{\sigma_i} = 1$ for each $i$, and hence, that $m = n$. Picking $\B$ to be the basis corresponding to this decomposition of $\rho'$, we get that the matrix of $\rho$ with respect to $\B$ is, indeed, of the desired form.
\end{proof}

As it turns out, we can combine the theory developed here with the Structure Theorem for Finite Abelian Groups to get an interesting result.

\begin{lemma}
    Let $G$ be a finite abelian group, expressed as a product $C_{n_1} \times \cdots \times C_{n_r}$ of cyclic groups $C_{n_i}$ of order $n_i > 1$. Then, $G$ has a faithful representation of degree $r$ over $\C$.
\end{lemma}
\begin{proof}
    Consider the space $V = \C^r = \C_1 \+ \cdots \+ \C_r$ (where each $\C_i$ is the one-dimensional subspace spanned by the $i$th element of some chosen $\C$-basis for $V$). Let $C_{n_i} = \cycl{g_i}$ and denote by $e_i$ the corresponding generators $\parenth{1, \ldots, 1, g_i, 1, \ldots, 1}$ of $G$. Define the map
    \begin{align}
        \rho : G \to \C : e_i \mapsto R_{i}
    \end{align}
    where $R_i$ is the rotation by $2\pi / n_i$ acting on the subspace $\C_i \cong \C$. In other words, with respect to the isomorphism $\GL{\C_i} \cong \C^{\times}$, the map $R_i$ corresponds to $\pexp{{2\pi}/{n_i}}$.\footnote{To be perfectly precise, $e_i$ is mapped not to $R_i$ but to the image of $R_i$ in the inclusion $\GL{\C_i} \to \GL{V}$ that extends $R_i$ by acting as the identity on components other than $i$ and as $R_i$ on component $i$. We use the word `component' to refer to a one-dimensional direct summand $\C_j$ of $V$.}

    $\rho$ has the following effect on group elements: for two group elements acting on the same component of $V$, $\rho$ maps their product to the composition of their associated rotations, and for elements acting on different components, $\rho$ combines their componentwise actions into a single action across two components. Therefore, $\rho$ is a group homomorphism, and hence, $\Vp$ is a representation of $G$ over $\C$.

    We now show that $\rho$ is injective. If $g \in G$ acts identically on all of $V$, it must, in particular, act identically on each component. But, the action of $g$ on each $\C_j$ is merely the action of the $j$th component of $g$ on $\C_j$. It is easy to see that the componentwise actions of $\rho$ on $V$ are all faithful, meaning that each component of $g$ is the identity in its respective cyclic group. Therefore, $g$ must be the identity in $G$, making $\rho$ a faithful representation.
\end{proof}
