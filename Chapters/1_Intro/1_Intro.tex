%% WEEK 1

\chapter{An Introduction to the Theory of Representations of Groups}
\thispagestyle{empty}

As I understand it, the fundamental idea behind Representation Theory is to study the actions of groups on vector spaces. While arbitrary vector spaces over arbitrary fields might not have naturally visualisable geometric properties, representations of groups in the ones that do can greatly illustrate the nature of these groups, especially to individuals like myself who delight in (somewhat literally) \textit{seeing} mathematics come alive.

\begin{wrapfigure}[9]{r}{0.48\linewidth}
    \centering
    \vspace{-2.25em}
    \begin{tikzpicture}
        \drawplane
        \drawsquare{1.5}
        \draw[-{Stealth}, red, dashed] (1.4,1.6) to [bend right = 45] (-1.4,1.6);
        \draw[-{Stealth}, red, dashed] (-1.6,1.4) to [bend right = 45] (-1.6,-1.4);
        \draw[-{Stealth}, red, dashed] (-1.4,-1.6) to [bend right = 45] (1.4,-1.6);
        \draw[-{Stealth}, red, dashed] (1.6,-1.4) to [bend right = 45] (1.6,1.4);
    \end{tikzpicture}
\end{wrapfigure}

A key motivating example in the study of representation theory would be the representations of Dihedral groups over $\R^2$. It is very natural to (at least informally) view the Dihedral group $D_n$ of order $2n$ as the group of symmetries of the regular $n$-gon; in other words, elements of $D_n$ have natural actions on a regular $n$-gon that preserve its structure. For instance, $D_4$ contains an element that rotates a square clockwise by $90^\circ$, an action under which the square is, of course, invariant.

If one were to now plot this square in $\R^2$, then action of the same element on the square can be extended to an orthogonal transformation of $\R^2$ that maps the $x$-axis to the $y$-axis and vice-versa, but in a manner preserving orientation (ie, that \textit{rotates the plane clockwise by $90^\circ$}). In a similar fashion, one can extend the actions of all dihedral groups $D_n$ to actions on the entirety of $\R^2$. More precisely, to every element of a dihedral group, one can ascribe a specific \textit{matrix} that transforms $\R^2$ in a manner preserving the regular $n$-gon.

This motivates the formal definition of a representation.

\section{Important Definitions}

\subsection{What is a Representation?}

It turns out that representations can be defined quite broadly, sidestepping the geometric niceties (or are they constraints?) of Euclidean spaces.

\begin{boxdefinition}[Group Representation]
    Let $G$ be a group. A representation of $G$ is a pair $\parenth{V, \rho}$ of a vector space $V$ and a group homomorphism $\rho : G \to \GL{V}$.
\end{boxdefinition}

Here, $\GL{V}$ refers to the \textbf{G}eneral \textbf{L}inear group over $V$, consisting of all vector space automorphisms of $V$ equipped with the binary operation of composition.

\begin{definition}[Degree of a Representation]
    \letgv. We define the degree of $V$ to be the dimension of $V$ over its base field.
\end{definition}

There exist innumerable examples of representations throughout mathematics. Below, we give some important ones.

\begin{example}[Important Classes of Representations]
    \hfill
    \begin{enumerate}
        \item \underline{The zero representation.} \letg\ and $V$ be any vector space. The map $\rho : G \to \GL{V}$ that maps any $g \in G$ to the zero map in $\GL{V}$ is a representation.
        
        \item \underline{The trivial representation.} \letg\ and $V$ be any vector space. The map $\rho : G \to \GL{V} : g \mapsto \id_{V}$ is a representation.
        
        \item \underline{The sign representation.} Let $G = S_n$, the symmetric group on $n$ elements, and let $V = K$, a field. Then, $\GL{V} = K^{\times}$, the multiplicative group of $K$. Denoting by $\xi$ the canonical map from $\Z$ to $K$, the map
        \begin{align*}
            \rho : G \to \GL{V} : \sigma \mapsto \xiof{\sgn{\sigma}}
        \end{align*}
        is a representation, where $\operatorname{sgn} : G \to \set{-1,1}$ denotes the sign homomorphism.

        \item \underline{Permutation representations.} Let $G$ be a group acting on a finite set $X$, and let $V = K[X]$, the free vector space (over some field $K$) generated by $X$. Consider a $K$-basis $\set{e_x \in V : x \in X}$ of $V$. Then, the map $\rho : G \to \GL{V}$ given by
        \begin{align*}
            \rhoof{g}\!\parenth{e_x} &= e_{g(x)}
        \end{align*}
        is a representation.

        \item \underline{The regular representation.} Let $G$ be a \textit{finite} group. The permutation representation corresponding to the canonical action of $G$ on itself by left-multiplication gives a representation of $G$ over $K[G]$, the free $K$-vector space generated by the set $G$.
    \end{enumerate}
\end{example}

As it turns out, we also have a notion of morphisms of representations.

\subsection{Morphisms of Representations}

\begin{boxdefinition}[Homomorphism of Representations]
    \letg\ and let $\parenth{V,\rho}$ and $\parenth{V',\rho'}$ be two representations of $G$. A homomorphism of representations $T : V \to V$ is a linear map $T : V \to V'$ such that $\forall g \in G$,
    \begin{align}
        T \circ \rhoof{g} &= \rho'(g) \circ T
    \end{align}
    or equivalently, the following diagram commutes:
    \begin{cd}
        V \arrow[r, "\rho(g)"] \arrow[d, "T"'] & V \arrow[d, "T"] \\
        V' \arrow[r, "\rho'(g)"'] & V'
        \label{Ch1:eq:cd_rep_map}
    \end{cd}
\end{boxdefinition}

A natural way to define two representations to be equal, or `isomorphic,' is as follows.

\begin{definition}[Equivalence of Representations]
    \letg\ and let $\parenth{V,\rho}$ and $\parenth{V',\rho'}$ be two representations of $G$. We say that $\parenth{V,\rho}$ and $\parenth{V',\rho'}$ are equivalent, denoted $\parenth{V,\rho} \sim \parenth{V',\rho'}$, if there exists a homomorphism $T : \parenth{V,\rho} \to \parenth{V',\rho'}$ that is invertible as a linear map---ie, that gives a linear isomorphism between $V$ and $V'$.
\end{definition}

The point of morphisms of representations is to be able to move from one vector space to another without losing the structural information captured by the representation. This is precisely illustrated in \eqref{Ch1:eq:cd_rep_map}.

\subsection{Subrepresentations}


\section{Invariant Constructions}

In this section, we briefly examine how ordinary linear algebraic constructions can interact with representations. We are particularly interested in the notion of \textit{invariance}, wherein a construction respects the structure of the representation(s) involved.

\subsection{Direct Sums of Representations}

The most elementary operation we can think about when we have two objects is \textit{putting them together}. One of the most meaningful ways of doing so in the context of linear algebra is the direct sum of two vector spaces. It turns out that this extends rather naturally to representations.

\begin{boxdefinition}[The Direct Sum of Two Representations]
    \letgvv. We define the direct sum of $\parenth{V,\rho}$ and $\parenth{V', \rho'}$ to be the pair $\parenth{V \+ V', \rho \+ \rho'}$, where $V \+ V'$ is the direct sum of $V$ and $V'$ as vector spaces and $\rho \+ \rho' : G \to \GL{V \+ V'}$ maps every $g \in G$ to the map
    \begin{align*}
        \parenth{\rho \+ \rho'}\!\parenth{g}\!\parenth{v \+ v'} &= \rho(g)(v) \+ \rho'(g)(v') \in \GL{V}
    \end{align*}
\end{boxdefinition}
\begin{proposition}
    \letgvv.
    \begin{enumerate}[label = \normalfont \arabic*., noitemsep]
        \item The direct sum $\parenth{V \+ V', \rho \+ \rho'}$ of $\parenth{V,\rho}$ and $\parenth{V', \rho'}$ is, indeed, a representation of $G$.
        \item $V$ and $V'$ are $G$-invariant subspaces\footnote{Technically, isomorphic to the subspaces $V \+ \set{0}$ and $\set{0} \+ V'$, but we overlook such distinctions.} of $V \+ V'$.
    \end{enumerate}
\end{proposition}
\begin{proof}
    \hfill
    \begin{enumerate}[noitemsep]
        \item Fix $g,h \in G$. For all $v \+ v' \in V \+ V'$,
        \begin{align*}
            \parenth{\rho \+ \rho'}\!(gh)(v \+ v') &= \rho(gh)(v) \+ \rho'(gh)(v') \\
            &= \rho(g)\!\parenth{\rho(h)(v)} \+ \rho'(g)\!\parenth{\rho'(h)(v')} \\
            &= \parenth{\rho \+ \rho'}\!(g)\!\parenth{\parenth{\rho \+ \rho'}\!(h)(v \+ v')}
        \end{align*}
        proving that $\rho \+ \rho'$ is multiplicative. Then, for any $g \in G$, $\parenth{\rho \+ \rho'}\!(g)$ has inverse $\parenth{\rho \+ \rho'}\!\parenth{g\inv}$. Hence, $\rho \+ \rho'$ is a homomorphism from $G$ to $\GL{V \+ V'}$.

        \item Fix $g \in G$ and $v \in V$. Clearly, $\parenth{\rho \+ \rho'}\!(g)(v) = \rho(g)(v)$. Since $\rho(g) \in \GL{V}$, it follows that $\rho(g)(v) \in V$. The proof that $V'$ is $G$-invariant is identical.
    \end{enumerate}
    \vspace{-1em}
\end{proof}

The above proposition gives us another reason to consider the direct sum to be an ``invariant'' construction: while it enriches both the vector space structure and the representation structure of a summand by adding another representation into the mix, it does not take anything away from the constructions that already exist.

With direct sums, we also have similar notions to reducibility. These are given by the following.

\begin{boxdefinition}[Complete Reducibility]
    A representation is said to be completely reducible if it is expressible as a direct sum of irreducible representations.
\end{boxdefinition}

\begin{boxdefinition}[Indecomposability]
    A nonzero representation is said to be indecomposable if it is inexpressible as a direct sum of two proper, nonzero subrepresentations.
\end{boxdefinition}
Nonzero representations that are not indecomposable are said to be decomposable.

\subsection{Complementary Subrepresentations}

It is a well-known fact from Linear Algebra that for any finite-dimensional vector space $V$, for any subspace $W \leq V$, there exists a \textit{complementary} subspace $W' \leq V$ such that $W \+ W' = V$. Over Euclidean spaces, for example, we have the very important notion of \textit{orthogonal} complements.

We can define a similar notion for representations, too.
\begin{boxdefinition}[Complementary Subrepresentation]
    \letgv. Let $\parenth{W, \rho\vert_W}$ be a subrepresentation of $\parenth{V, \rho}$. A complementary subrepresentation of $\parenth{W, \rho\vert_W}$ is a subrepresentation $\parenth{U, \rho\vert_U}$ such that $V = U \+ W$.
\end{boxdefinition}
This notion of complementarity is, indeed, compatible with the notion of direct sums of representations.
\begin{proposition}
    \letgv. Let $\parenth{W, \rho\vert_W}$ and $\parenth{U, \rho\vert_U}$ be complementary subrepresentations. Then, their direct sum $\parenth{V, \rho\vert_W \+ \rho\vert_U}$ is equivalent to $\parenth{V, \rho}$ as a representation of $G$.
\end{proposition}
\begin{proof}
    It suffices to show that $\rho = \rho\vert_W \+ \rho\vert_U$. Then, the identity map would give an equivalence of representations. Indeed, every $v \in V$ is expressible uniquely as a direct sum $w \+ u$ for some $w \in W$ and $u \in U$. So, for all $g \in G$,
    \begin{align*}
        \rho(g)(v) &= \rho(g)(w \+ u) \\
        &= \rho(g)(w) \+ \rho(g)(u) \\
        &= \rho\vert_W(g)(w) \+ \rho\vert_U(g)(u) \\
        &= \parenth{\rho\vert_W \+ \rho\vert_U}\!(g)(w \+ u)
    \end{align*}
    where the sum in the second equality is direct because $W$ and $U$ are $\rho(g)$-invariant.
\end{proof}

We now recall an important result from Linear Algebra.

\begin{definition}[Projection]
    Let $V$ be a vector space and let $T : V \to V$ be linear. Observe that we have the following equivalence:
    \begin{align}
        T^2 = T &\iff \forall w \in \pim{T},\ T(w) = w \label{Ch1:Eq:Proj_Op_Equiv}
    \end{align}
    If $T$ satisfies either one of the above conditions, $T$ is said to be a projection.
\end{definition}

We do not prove \eqref{Ch1:Eq:Proj_Op_Equiv}, but we do prove the following lemma, which will prove to be useful.

\begin{lemma} \label{Ch1:Lem:LinAlg_Proj_Dsum}
    Let $V$ be a vector space. For all projections $T : V \to V$, $V = \pker{T} \+ \pim{T}$.
\end{lemma}
\begin{proof}
    Let $T : V \to V$ be a projection. We then have the following.
    \begin{description}
        \item[\underline{$\pim{T} \cap \pker{T} = \set{0}$:}] Fix $w \in \pim{T} \cap \pker{T}$. Since $w \in \pim{T}$, $\exists v \in V$ such that $w = T(v)$. Furthermore, since $w \in \pker{T}$, $T(w) = 0$. Since $w = \Tof{v}$, this is equivalent to saying that $\Tof{\Tof{v}} = 0$. But, by \eqref{Ch1:Eq:Proj_Op_Equiv}, $\Tof{\Tof{v}} = \Tof{v}$. Hence, $\Tof{v} = 0$. Then, since $\Tof{v} = w$, it follows that $w = 0$.
        \item[\underline{$V = \pker{T} + \pim{T}$:}] Fix $v \in V$. We write $v = \Tof{v} + \parenth{v - \Tof{v}}$. Clearly, $\Tof{v} \in \pim{T}$. Further, $\Tof{v - \Tof{v}} = \Tof{v} - \Tof{v} = 0$. Hence, $v - \Tof{v} \in \pker{T}$.
    \end{description}
    Therefore, we do, indeed, have $V = \pker{T} \+ \pim{T}$.
\end{proof}

It turns out that this gives us an important criterion for decomposability.

\begin{corollary}\label{Ch1:Cor:G_lin_proj_ker_im}
    \letgv. If $T : \Vp \to \Vp$ is a $G$-linear projection, then $V = \pker{T} \+ \pim{T}$ is a direct sum of subrepresentations.
\end{corollary}
\begin{proof}
    The result follows immediately from Lemma \ref{Ch1:Lem:LinAlg_Proj_Dsum} and Proposition \ref{Ch1:Prop:ker_im_subreps}.
\end{proof}

\subsection{Maschke's Theorem}

Given the theme of this section---namely, understanding the compatibility of ordinary linear-algebraic constructions with representation structures---one might wonder under what conditions (if any) we have the existence of a complementary subrepresentations. The answer lies in Maschke's Theorem, which is the first major result of the course.

\begin{boxtheorem}[Maschke's Theorem] \label{SP:Thm:Maschke}
    \letfg, $K$ a field such that $\pchar{K} \nmid \abs{G}$, and $\Vp$ a representation of $G$ over $K$. Then, any subrepresentation of $V$ admits a complementary subrepresentation.
\end{boxtheorem}
\begin{proof}
    Let $W \leq V$ be $G$-invariant. The idea is to construct a $G$-linear map $T : V \to V$ with image $W$. Then, by Corollary \ref{Ch1:Cor:G_lin_proj_ker_im}, $\pker{T}$ would give a complementary subrepresentation.
\end{proof}

\subsection{The $G$-Invariant Inner-Product}

