\section{Group Algebras and Modules}

In this section, we study an important class of field algebras, namely, group algebras, and an important class of modules over said algebras, namely, group modules.

\subsection{Preliminaries}

\begin{boxdefinition}[Group Algebra]
    \letfg\ and let $K$ be a field. The group algebra $KG$ is the $K$-algebra obtained by endowing the free $K$-vector space $K[G]$ generated by $G$ (as a set) with the multiplication
    \begin{align*}
        \parenth{\sum_{g \in G} \alpha_g e_g} \cdot \parenth{\sum_{g \in G} \beta_g e_g} &:= \sum_{g \in G} \sum_{h \in G} \alpha_{g} \beta_h e_{gh}
    \end{align*}
\end{boxdefinition}
\begin{remark} \hfill
    \begin{enumerate}[noitemsep]
        \item It is easy to verify that $KG$ is, indeed, a $K$-algebra, with the multiplicative identity given by $1 \cdot e_1$ (where $1 \in K$ is the identity in $K$ and $e_1$ is the basis vector corresponding to $1 \in G$, the identity in $G$).
        \item The map $g \mapsto e_g : G \to KG$ gives a trivial embedding of $G$ in $KG$.
        \item Going forward, for ease of notation, we will denote each basis element $e_g$ as simply $g$. In particular, we will denote the identity as simply $1$, slightly abusing notation by omitting the distinction between $1 \in K$ and $1 \in G$.
    \end{enumerate}
\end{remark}

We have a similar notion of group modules.

\begin{boxdefinition}[Group Module] \label{Ch1:Def:KG-Module}
    \letg\ and let $V$ be a vector space over a field $K$. We say that $V$ is a $KG$-module if we can define a multiplication $g \cdot v$ for some $g \in G$ and $v \in V$ that satisfies the following conditions for all $u, v \in V$, $g, h \in G$ and $\lambda \in K$:
    \begin{enumerate}[noitemsep]
        \item $g \cdot v \in V$
        \item $\parenth{gh} \cdot v = g \cdot \parenth{h \cdot v}$
        \item $1 \cdot v = v$
        \item $g \cdot \parenth{\lambda v} = \lambda \parenth{g \cdot v}$
        \item $g \cdot \parenth{u + v} = g \cdot u + g \cdot v$
    \end{enumerate}
\end{boxdefinition}
As expected, we have the following result. The proof merely involves checking several conditions, so we omit it here.
\begin{proposition}
    \letg\ and let $V$ be a vector space over a field $K$. If $V$ is a $KG$-module with multiplication $\cdot$ (as per Definition \ref{Ch1:Def:KG-Module}), then for $v \in V$, the multiplication
    \begin{align*}
       \underbrace{\parenth{\sum_{g \in G} \lambda_g e_g}}_{\in KG} \cdot v &:= \sum_{g \in G} \overbrace{\lambda_g \underbrace{\parenth{g \cdot v}}_{\text{$KG$-module multiplication}}}^{\text{$K$-vector space multuplication}}
    \end{align*}
    endows $V$ with a module structure over $K[G]$.
\end{proposition}

It turns out that we can move from modules to representations and vice-versa quite easily.

\begin{proposition}
    \letg\ and let $V$ be a vector space over a field $K$.
    \begin{enumerate}[label = \normalfont \arabic*.]
        \item If $\rho : G \to \GL{V}$ gives a representation of $G$, then $V$ is a $KG$-module with multiplication given by $g \cdot v = \rhoff{g}{v}$ for all $g \in G$ and $v \in V$.
        \item If $V$ is a $KG$-module with multiplication $\cdot$, the map $\rho : G \to \GL{V}$ given by $\rhoff{g}{v} := g \cdot v$ is a representation.
    \end{enumerate}
\end{proposition}

While we do not prove this proposition either, we will mention that this underscores the importance of the language of $KG$-modules in representation theory. As commutative algebra---specifically, the theory of modules---has been well-studied, we can use those tools to understand representations of groups. Table \ref{Ch1:Tab:Module_Rep_Dictionary} relates notions about group modules and representations.

\begin{table}[!h]
    \centering
    \begin{tabular}{c|c}
        \textbf{$KG$-Modules} & \textbf{Representations} \\ \hline
        Simple & Irreducible \\
        Semi-Simple & Completely Irreducible \\
        Submodule & Subrepresentation \\
        Viewing $KG$ as a $KG$-Module & The Regular Representation \\
        Isomorphism & Equivalence of Representations \\
        Dimension (as a $K$-vector space) & Degree
    \end{tabular}
    \caption{\centering A `dictionary' to translate between the language of $KG$-modules and that of representations of $G$ over $K$.}
    \label{Ch1:Tab:Module_Rep_Dictionary}
\end{table}

We illustrate the above equivalence by stating Maschke's Theorem in the language of $KG$-Modules.

\begin{lemma}[Maschke's Theorem, Module Version]
    \letfg, $K$ a field whose characteristic does not divide the order of $G$. Then, any $KG$-Module $V$ is semi-simple.
\end{lemma}

\subsection{Schur's Lemmas}

In this subsection, we explore several versions of an important result by Schur. The main result consists of two cases, the first of which shall (by convention) deal with the non-isomorphic case and the second of which shall deal with the isomorphic case. We will refer to them as Schur's First and Second Lemmas respectively. We begin by stating them in their most general form.

\begin{theorem}[Schur's Lemmas for Rings]
    Let $A$ be a ring and let $S, T$ be simple $A$-modules.
    \begin{enumerate}[label = \normalfont \arabic*., noitemsep]
        \item If $S$ and $T$ are non-isomorphic, then $\Hom_A(S, T) = \set{0}$.
        \item If $S$ and $T$ are isomorphic, then $\Hom_A(S,T)$ is a division ring.
    \end{enumerate}
\end{theorem}
\begin{proof}
    We rely on the fact that for all $\phi \in \Hom_A(S, T)$, $\pker{\phi} \leq S$ and $\pim{\phi} \leq T$.
    \begin{enumerate}
        \item Let $S$ and $T$ be non-isomorphic. Fix $\phi \in \Hom_A(S,T)$. Since $S$ is simple, we must have that $\pker{\phi} \in \set{\set{0}, S}$. If $\pker{\phi} = \set{0}$, then $\pim{\phi} = T$, meaning $S \cong T$, a contradiction.
        \item Let $\phi \in \Hom_A(S,T) \setminus \set{0}$. Then, $\pker{\phi} \neq S$, meaning that $\pker{\phi} = \set{0}$. Then, $\pim{\phi} = T$, making $\phi$ an isomorphism. In particular, this means that $\phi$ admits an inverse, making $\Hom_A(S,T)$ a division ring.
    \end{enumerate}
\end{proof}

It turns out we can do a bit better when dealing with a specific class of rings, namely, algebras over fields.

\begin{theorem}[Schur's Lemmas for Algebras]\label{Ch1:Thm:Schur_Algebra}
    Let $K$ be an algebraically closed field and $A$ a $K$-algebra. Let $S$ and $T$ be simple $A$-modules.
    \begin{enumerate}[label = \normalfont \arabic*., noitemsep]
        \item If $S \not\cong T$, then $\Hom_A(S, T) = \set{0}$.
        \item If $S \cong T$, then $K \cong \Hom_A(S, T)$ via the map $\alpha \mapsto \alpha \cdot \id$.
    \end{enumerate}
\end{theorem}
\begin{proof}
    \hfill
    \begin{enumerate}
        \item As before.
        \item We do not distinguish $S$ and $T$ in this proof.
        
        Fix $\phi \in \Hom_A(S,S)$. Then, $\phi$ can be viewed as an element of $\Mn{n}{K}$, where $n = \pdim{S}$. Since $K$ is algebraically closed, $\phi$ admits an eigenvalue $\lambda \in K$. Now, consider the map $\phi - \lambda \id \in \Hom_A(S,S)$. Clearly, $\pker{\phi - \lambda \id} \neq \set{0}$, since it contains all eigenvectors with eigenvalue $\lambda$. Since $S$ is simple, it must be that $\pker{\phi - \lambda \id} = S$, meaning $\phi - \lambda \id = 0$. In other words, $\phi = \lambda \id$.  % Explain the logic of this in more detail.
    \end{enumerate}
\end{proof}

We also have a converse (of sorts) when working with algebras.

\begin{theorem}[Converse of Schur's Lemma for Algebras]\label{Ch1:Thm:Schur_Algebra_Converse}
    Let $K$ be a field, $A$ a $K$-algebra and $M$ a semi-simple $A$-module. If $\Hom_A(M,M) \cong K$, then $M$ is simple.
\end{theorem}
\begin{proof}
    Assume that $\Hom_A(M, M) \cong K$. As $M$ is semi-simple, $M$ admits a decomposition $M = \bigoplus_{i \in \I} N_i$, where for all $i \in \I$, $N_i$ is a simple $A$-module. We will proceed by proving that $\abs{\I} = 1$ and that $M \cong N_i$ for the unique $i \in \I$.

    Suppose $\abs{\I} > 1$, ie, that there exist distinct $i, j \in \I$. Theorem~\ref{Ch1:Thm:Schur_Algebra} tells us that $\Hom_A(N_i, N_i) \cong \Hom_A(N_j, N_j) \cong K$. Consider the maps $phi_i \in \Hom_A(N_i, N_i)$ and $\phi_j \in \Hom_A(N_j, N_j)$. We know that $\exists \lambda_i, \lambda_j \in K$ such that $\phi_i = \lambda_i \cdot \id_{N_i}$ and $\phi_j = \lambda_j \cdot \id_{N_j}$. Indeed, we can choose $\phi_i$ and $\phi_j$ so that $\lambda_i \neq \lambda_j$. Then, we can construct a map $\phi \in \Hom_A(M, M)$ such that $\phi\vert_{N_i} = \phi_i$ and $\phi\vert_{N_j} = \phi_j$.\footnote{For example, if $M = N_i \+ N_j$, then take $\phi = \phi_i \+ \phi_j$. Else, add the identity maps on the other summands.} By assumption, $\phi$ must be of the form $\lambda \cdot \id_M$ for some $\lambda \in K$. But, this implies that $\lambda = \lambda_i = \lambda_j$, a contradiction. Therefore, $\abs{\I} = 1$ and $M \cong N_i$ for the unique $i \in \I$.
\end{proof}

\begin{boxtheorem}[Schur's Lemmas for Finite Groups, over $\C$] \label{SP:Thm:Schur_fin_G_over_C}
    \letfg\ and let $S$ and $T$ be simple $\C G$ modules that are finite-dimensional (as vector spaces) over $K$, with associated representations $\rho_S : G \to \GL{S}$ and $\rho_T : G \to \GL{T}$. Let $f : S \to T$ be an arbitrary $\C$-linear map. Define the map
    \begin{align}
        \hat{f} := \frac{1}{\abs{G}} \sum_{g \in G} \rho_T(g) \circ f \circ \rho_S\!\parenth{g\inv}
        \label{SP:eq:Def_Hat_Map}
    \end{align}
    Then, $\hat{f}$ is, in fact, a \CGM\ homomorphism. Furthermore, we have the following:
    \begin{enumerate}[label = \normalfont \arabic*.]
        \item If $S \not\cong T$, then $\hat{f}$ is identically zero.
        \item If $S \cong T$, then
        \begin{align*}
            \hat{f} &= \frac{1}{\pdim{S}} \Tr{f} \cdot \id_S
        \end{align*}
    \end{enumerate}
\end{boxtheorem}
\begin{proof}
    We only really need to show that $\hat{f} \in \Hom_{\C G}(S, T)$. The rest follows relatively naturally from Theorem \ref{Ch1:Thm:Schur_Algebra} (Schur's Lemmas for Algebras), as we shall see. So, fix $h \in G$. Observe that
    \begin{align*}
        \rho_T(h) \circ \hat{f} &= \rho_T(h)\!\parenth{\frac{1}{\abs{G}} \sum_{g \in G} \rho_T(g) \circ f \circ \rho_S\!\parenth{g\inv}} \\
        &= \frac{1}{\abs{G}} \sum_{g \in G} \rho_T(hg) \circ f \circ \rho_S\!\parenth{g\inv} \circ \rho_S\!\parenth{h\inv} \circ \rho_S(h) \\
        &= \frac{1}{\abs{G}} \sum_{g \in G} \parenth{\rho_T(hg) \circ f \circ \rho_S\!\parenth{g\inv h\inv}} \circ \rho_S(h) \\
        &= \hat{f} \circ \rho_S(h)
    \end{align*}
    proving that $\hat{f}$ is, indeed, a homomorphism of $\C G$-modules. Then,
    \begin{enumerate}
        \item If $S \not\cong T$, then by Schur's First Lemma for Algebras, $\hat{f}$ is identically $0$.
        \item If $S \not\cong T$, then by Schur's Second Lemma for Algebras, $\hat{f}$ is a scalar multiple of the identity. Write $\hat{f} = \lambda \cdot \id_S$. Clearly, we have that $\Tr{\hat{f}} = \lambda \pdim{S}$. But, by the definition of $\hat{f}$, we have that
        \begin{align*}
            \Tr{\hat{f}} &= \frac{1}{\abs{G}} \sum_{g \in G} \Tr{\rho_T(g) \circ f \circ \rho_S\!\parenth{g\inv}} \\
            &= \frac{1}{\abs{G}} \sum_{g \in G} \Tr{f \circ \rho_T\!\parenth{g} \circ \rho_S\!\parenth{g\inv}} \\
            &= \frac{1}{\abs{G}} \sum_{g \in G} \Tr{f} = \Tr{f}
        \end{align*}
        where we can apply the Trace Theorem to the composition $\rho_T(g) \circ f \circ \rho_S\!\parenth{g\inv}$ to get the second inequality because $S \cong T$, meaning that we can, indeed, change the order of composition and get a `sensible' linear map, as all dimensions agree. Furthermore, since $S \cong T$, we have, in particular, that $\rho_S = \rho_T$, which gives us the third equality.
        
        Seeing as the trace of a linear map is unique, we must have that $\lambda \cdot \pdim{S} = \Tr{f} \iff \lambda = \frac{1}{\pdim{S}} \Tr{f}$ (we know $\pdim{S} \neq 0$ because $S$ is simple). This then tells us that $\hat{f}$ is of the desired form.
    \end{enumerate}
\end{proof}
We can simplify the above a little when $f$ is a \CGM\ homomorphism.
\begin{corollary}\label{SP:Cor:Schur_fin_G_over_C_Morhp_Ver}
    \letfg\ and let $S$ and $T$ be simple $\C G$ modules that are finite-dimensional (as vector spaces) over $K$, with associated representations $\rho_S : G \to \GL{S}$ and $\rho_T : G \to \GL{T}$. Let $f : S \to T$ be a \CGM\ homomorphism. Then,
    \begin{enumerate}[label = \normalfont \arabic*.]
        \item If $S \not\cong T$, then $f$ is identically zero.
        \item If $S \cong T$, then
        \begin{align*}
            f &= \frac{1}{\pdim{S}} \Tr{f} \cdot \id_S
        \end{align*}
    \end{enumerate}
\end{corollary}
\begin{proof}
    We simply show that $f = \hat{f}$, where $\hat{f}$ is as defined in \eqref{SP:eq:Def_Hat_Map}: for all $s \in S$,
    \begin{align*}
        \hat{f} &= \frac{1}{\abs{G}} \sum_{g \in G} \rho_T(g) \circ f \circ \rho_S\!\parenth{g\inv} \\
        &= \frac{1}{\abs{G}} \sum_{g \in G} \rho_T(g) \circ \rho_T\!\parenth{g\inv} \circ f \\
        &= \frac{1}{\abs{G}} \sum_{g \in G} f = f
    \end{align*}
    where the second equality follows from the fact that $f$ is a \CGM\ homomorphism, and therefore, a homomorphism of representations---cf. \eqref{Ch1:Eq:Def_Rep_Morph}. We can now simply apply Theorem~\ref{SP:Thm:Schur_fin_G_over_C} to get the desired result.
\end{proof}

Schur's Lemmas (especially the version thereof pertaining to complex representations of finite groups) will prove to be incredibly useful. In the next subsection, we will give a direct application to the classification of representations of finite abelian groups; however, this is not the last that we will be seeing of them.

\subsection{Representations of Finite Abelian Groups over $\C$}

It is natural to wonder what the purpose was of studying group algebras and modules. It turns out that one of the reasons the correspondence between representations and group modules is so powerful is that it allows the application of Schur's Lemmas to representation theoretic problems. For instance, in the following Lemma, we classify all irreducible representations of finite abelian groups over $\C$.

\begin{lemma} \label{Ch1:Lem:Fin_Ab_over_C}
    Let $G$ be a fininte abelian group. Then, all irreducible $\C G$-modules are of dimension $1$. Equivalently, all irreducible representations of $G$ over $\C$ are of degree $1$.
\end{lemma}
\begin{proof}
    Let $V$ be an irreducible $\C G$-module. Since $G$ is abelian, for all $g,h \in G$ and $v \in V$, $\parenth{gh} \cdot v = \parenth{hg} \cdot v$. Therefore, for some fixed $h \in G$, the following map is $\C G$-linear:
    \begin{align*}
        \phi_h : V \to V : v \mapsto h \cdot v
    \end{align*}
    By Theorem \ref{SP:Thm:Schur_fin_G_over_C}, we know that $\exists \lambda_h \in \C$ such that $\widehat{\phi_h} = \phi_h = \lambda_h \cdot \id_V$. Hence, any subspace of $V$ must be a $\C G$-submodule. Therefore, the only way for $V$ to be irreducible is for it to \textit{admit} no proper, nonzero subspaces in the first place, which would necessitate that $\pdim{V} = 1$.
\end{proof}
\begin{remark}
    It goes without saying that any degree $1$ representation of a group is irreducible. Hence, the above Lemma effectively tells us that irreducibility is equivalent to having degree $1$ when working with finite abelian groups over $\C$.
\end{remark}

We now have a complete classification of irreducible representations of cyclic groups over $\C$.

\begin{corollary}\label{Ch1:Cor:Irr_Cycl}
    Let $G = C_n = \cycl{a}$ be the cyclic group of order $n$. Then, there are precisely $n$ irreducible representations of $G$ over $\C$.
    \begin{proof}
        \letgv. We know, from Lemma \ref{Ch1:Lem:Fin_Ab_over_C}, that $V \cong \C$ and hence, $\GL{V} \cong \C^\times$. Now, let $x := \rho(a)$. It must be that $x^n = 1$, making $x$ an $n$th root of unity. In other words, $\exists 1 \leq k \leq n$ such that $x = e^{\frac{2\pi i}{k}}$. Therefore, there are precisely $n$ possible choices of $x$, each giving a different representation.
    \end{proof}
\end{corollary}

Combining this with the structure theorem for finite abelian groups, we can classify all irreducible representations thereof over $\C$.

\begin{boxtheorem}\label{Ch1:Thm:Irreps_Fin_Ab_Grps_C}
    Let $G$ be a finite abelian group. Then, there are precisely $\abs{G}$ irreducible representations of $G$ over $\C$.
\end{boxtheorem}
\begin{proof}
    Let $G = C_{n_1} \times \cdots \times C_{n_r}$ be the decomposition of $G$ into cyclic groups of orders $n_i$. We know that $G$ has generators $e_i = \parenth{0, \ldots, 0, 1, 0, \ldots, 0}$, with $1$ in the $i$th coordinate. There are precisely $n_i$ choices of where to send $e_i$, namely, the $n_i$th roots of unity. Given that group homomorphisms are uniquely defined by where they map the generators, the degree $1$ representations are those mapping the $e_i$s to each possible $n_i$th root of unity. Hence, there are $n_1 \times \cdots \times n_r = \abs{G}$ irreducible representations of $G$ over $\C$.
\end{proof}
\begin{remark}
    We will see later on that the sum of the squares of the dimensions of all of the irreducible representations of a group will need to equal its order (cf. \eqref{Ch2:Eq:Grp_Order_Sum_Char_Sq}). This is another way to see that there must be exactly $\abs{G}$ irreducible complex representations of a finite abelian group $G$. Specifically, this works because of Lemma~\ref{Ch1:Lem:Fin_Ab_over_C}.
\end{remark}
This tells us a little something about representations of any finite group over $\C$.
\begin{corollary}
    Let $G$ be a finite group. The number of irreducible representations of $G$ over $\C$ is precisely $\abs{G} / \abs{\brac{G, G}}$.
\end{corollary}
\begin{proof}
    We know, from Corollary~\ref{Ch1:Cor:Degree1Lifts} that the irreducible complex representations of $G$ are in bijection with those of its abelianisation $\quotient{G}{\brac{G, G}}$. Given that $G$ is finite, so is its abelianisation. Therefore, by Theorem~\ref{Ch1:Thm:Irreps_Fin_Ab_Grps_C}, the number of such irreducible representations is the cardinality of its abelianisation.
\end{proof}

Lemma~\ref{Ch1:Lem:Fin_Ab_over_C} is also useful in the study of arbitrary complex representations over $\C$ of finite groups.

\begin{proposition}\label{Ch1:Prop:Diag_Roots_of_Ord}
    \letfgv\ over $\C$. For all $g \in G$, there is a basis of $V$ with respect to which $\rho(g)$ has matrix $\diag{\eps_1, \cdots, \eps_n}$, with $\eps_i^{\ord{g}} = 1$ for all $1 \leq i \leq n$.
\end{proposition}
\begin{proof}
    Fix $g \in G$, and consider the representation $\rho' : \cycl{g} \to \GL{V}$ given by $\rho' = \rho\vert_{\cycl{g}}$. Then, $\rho'$ is a representation of a finite abelian group.

    By Maschke's Theorem, $\rho' = \sigma_1 \+ \cdots \+ \sigma_k$ for irreducible subrepresentations $\sigma_1, \ldots, \sigma_m$ of $\cycl{g}$. By Lemma \ref{Ch1:Lem:Fin_Ab_over_C}, we know that $\pdeg{\sigma_i} = 1$ for each $i$, and hence, that $m = n$. Picking $\B$ to be the basis corresponding to this decomposition of $\rho'$, we get that the matrix of $\rho$ with respect to $\B$ is, indeed, of the desired form.
\end{proof}

As it turns out, we can combine the theory developed here with the Structure Theorem for Finite Abelian Groups to get an interesting result.

\begin{lemma}
    Let $G$ be a finite abelian group, expressed as a product $C_{n_1} \times \cdots \times C_{n_r}$ of cyclic groups $C_{n_i}$ of order $n_i > 1$. Then, $G$ has a faithful representation of degree $r$ over $\C$.
\end{lemma}
\begin{proof}
    Consider the space $V = \C^r = \C_1 \+ \cdots \+ \C_r$ (where each $\C_i$ is the one-dimensional subspace spanned by the $i$th element of some chosen $\C$-basis for $V$). Let $C_{n_i} = \cycl{g_i}$ and denote by $e_i$ the corresponding generators $\parenth{1, \ldots, 1, g_i, 1, \ldots, 1}$ of $G$. Define the map
    \begin{align*}
        \rho : G \to \C : e_i \mapsto R_{i}
    \end{align*}
    where $R_i$ is the rotation by $2\pi / n_i$ acting on the subspace $\C_i \cong \C$. In other words, with respect to the isomorphism $\GL{\C_i} \cong \C^{\times}$, the map $R_i$ corresponds to $\pexp{{2\pi}/{n_i}}$.\footnote{To be perfectly precise, $e_i$ is mapped not to $R_i$ but to the image of $R_i$ in the inclusion $\GL{\C_i} \to \GL{V}$ that extends $R_i$ by acting as the identity on components other than $i$ and as $R_i$ on component $i$. We use the word `component' to refer to a one-dimensional direct summand $\C_j$ of $V$.}

    $\rho$ has the following effect on group elements: for two group elements acting on the same component of $V$, $\rho$ maps their product to the composition of their associated rotations, and for elements acting on different components, $\rho$ combines their componentwise actions into a single action across two components. Therefore, $\rho$ is a group homomorphism, and hence, $\Vp$ is a representation of $G$ over $\C$.

    We now show that $\rho$ is injective. If $g \in G$ acts identically on all of $V$, it must, in particular, act identically on each component. But, the action of $g$ on each $\C_j$ is merely the action of the $j$th component of $g$ on $\C_j$. It is easy to see that the componentwise actions of $\rho$ on $V$ are all faithful, meaning that each component of $g$ is the identity in its respective cyclic group. Therefore, $g$ must be the identity in $G$, making $\rho$ a faithful representation.
\end{proof}

Now that we have classified all complex representations of finite abelian groups, we can move on to complex representations of arbitrary finite groups. Unfortunately, this is easier said than done: the machinery we have thus far developed is not enough. In the next chapter, we will explore the theory of \textbf{characters}, which will allow us to understand complex representations of finite groups in much greater detail.