\section{Group Algebras and Modules}

In this section, we study an important class of field algebras, namely, group algebras, and an important class of modules over said algebras, namely, group modules.

\subsection{A Few Definitions}

\begin{boxdefinition}[Group Algebra]
    \letfg\ and let $K$ be a field. The group algebra $KG$ is the $K$-algebra obtained by endowing the free vector space $K[G]$ generated by $G$ (as a set) with the multiplication
    \begin{align*}
        \parenth{\sum_{g \in G} \alpha_g e_g} \cdot \parenth{\sum_{g \in G} \beta_g e_g} &:= \sum_{g \in G} \sum_{h \in G} \alpha_{g} \beta_h e_{gh}
    \end{align*}
\end{boxdefinition}
\begin{remark} \hfill
    \begin{enumerate}[noitemsep]
        \item For ease of notation, we often denote elements $e_g$ of the basis as simply $g$.
        \item It is easy to verify that $KG$ is, indeed, a $K$-algebra, with the multiplicative identity given by $e_1$ (where $1 \in G$ is the identity).
        \item The map $g \mapsto e_g : G \to KG$ gives a trivial embedding of $G$ in $KG$.
    \end{enumerate}
\end{remark}

We have a similar notion of group modules.

\begin{boxdefinition}[Group Module] \label{Ch1:Def:KG-Module}
    \letg\ and let $V$ be a vector space over a field $K$. We say that $V$ is a $KG$-module if we can define a multiplication $g \cdot v$ for some $g \in G$ and $v \in V$ that satisfies the following conditions for all $u, v \in V$, $g, h \in G$ and $\lambda \in K$:
    \begin{enumerate}[noitemsep]
        \item $g \cdot v \in V$
        \item $\parenth{gh} \cdot v = g \cdot \parenth{h \cdot v}$
        \item $1 \cdot v = v$
        \item $g \cdot \parenth{\lambda v} = \lambda \parenth{g \cdot v}$
        \item $g \cdot \parenth{u + v} = g \cdot u + g \cdot v$
    \end{enumerate}
\end{boxdefinition}

Note that a $KG$-module is, indeed, a module over $KG$.

\begin{proposition}
    \letg\ and let $V$ be a vector space over a field $K$. If $V$ is a $KG$-module with multiplication $\cdot$ (as per Definition \ref{Ch1:Def:KG-Module}), then for $v \in V$, the multiplication
    \begin{align*}
        \parenth{\sum_{g \in G} \lambda_g e_g} \cdot v &:= \sum_{g \in G} \lambda_g \parenth{g \cdot v}
    \end{align*}
    endows $V$ with a module structure over $K[G]$.
\end{proposition}

Furthermore, it turns out that we can move from modules to representations and vice-versa quite easily.

\begin{proposition}
    \letg\ and let $V$ be a vector space over a field $K$.
    \begin{enumerate}[label = \normalfont \arabic*.]
        \item If $\rho : G \to \GL{V}$ gives a representation of $G$, then $V$ is a $KG$-module with multiplication given by $g \cdot v = \rhoff{g}{v}$ for all $g \in G$ and $v \in V$.
        \item If $V$ is a $KG$-module with multiplication $\cdot$, the map $\rho : G \to \GL{V}$ given by $\rhoff{g}{v} := g \cdot v$ is a representation.
    \end{enumerate}
\end{proposition}

The proofs of the above propositions are trivial and merely involve manually checking several basic conditions. Hence, we omit them.

We now give a basic `dictionary' of sorts to go back and forth between the language of group modules and that of representations:

\begin{table}[!h]
    \centering
    \begin{tabular}{c|c}
        \textbf{$KG$-Modules} & \textbf{Representations} \\ \hline
        Simple & Irreducible \\
        Semi-Simple & Completely Irreducible \\
        Submodule & Subrepresentation \\
        Viewing $KG$ as a $KG$-Module & The Regular Representation \\
        Isomorphism & Equivalence of Representations \\
        Dimension (as a $K$-vector space) & Degree
    \end{tabular}
\end{table}

We illustrate the above equivalence by stating Maschke's Theorem in the language of $KG$-Modules.

\begin{lemma}[Maschke's Theorem, Module Version]
    \letfg, $K$ a field whose characteristic does not divide the order of $G$. Then, any $KG$-Module $V$ is semi-simple.
\end{lemma}

\subsection{Schur's Lemmas}

In this subsection, we explore several versions of an important result by Schur.

\begin{theorem}[Schur's Lemmas for Rings]
    Let $A$ be a ring and let $S, T$ be simple $A$-modules.
    \begin{enumerate}[label = \normalfont \arabic*., noitemsep]
        \item If $S$ and $T$ are non-isomorphic, then $\Hom_A(S, T) = \set{0}$.
        \item If $S$ and $T$ are isomorphic, then $\Hom_A(S,T)$ is a division ring.
    \end{enumerate}
\end{theorem}
\begin{proof}
    We rely on the fact that for all $\phi \in \Hom_A(S, T)$, $\pker{\phi} \leq S$ and $\pim{\phi} \leq T$.
    \begin{enumerate}
        \item Let $S$ and $T$ be non-isomorphic. Fix $\phi \in \Hom_A(S,T)$. Since $S$ is simple, we must have that $\pker{\phi} \in \set{\set{0}, S}$. If $\pker{\phi} = \set{0}$, then $\pim{\phi} = T$, meaning $S \cong T$, a contradiction.
        \item Let $\phi \in \Hom_A(S,T) \setminus \set{0}$. Then, $\pker{\phi} \neq S$, meaning that $\pker{\phi} = \set{0}$. Then, $\pim{\phi} = T$, making $\phi$ an isomorphism. In particular, this means that $\phi$ admits an inverse, making $\Hom_A(S,T)$ a division ring.
    \end{enumerate}
\end{proof}

\begin{theorem}[Schur's Lemmas for Algebras]
    Let $K$ be an algebraically closed field and $A$ a $K$-algebra. Let $S$ and $T$ be simple $A$-modules.
    \begin{enumerate}[label = \normalfont \arabic*., noitemsep]
        \item If $S \not\cong T$, then $\Hom_A(S, T) = \set{0}$.
        \item If $S \cong T$, then $K \cong \Hom_A(S, T)$ via the map $\alpha \mapsto \alpha \cdot \id$.
    \end{enumerate}
\end{theorem}
\begin{proof}
    \hfill
    \begin{enumerate}
        \item As before.
        \item We do not distinguish $S$ and $T$ in this proof.
        
        Fix $\phi \in \Hom_A(S,S)$. Then, $\phi$ can be viewed as an element of $\Mn{n}{K}$, where $n = \pdim{S}$. Since $K$ is algebraically closed, $\phi$ admits an eigenvalue $\lambda \in K$. Now, consider the map $\phi - \lambda \id \in \Hom_A(S,S)$. Clearly, $\pker{\phi - \lambda \id} \neq \set{0}$, since it contains all eigenvectors with eigenvalue $\lambda$. Since $S$ is simple, it must be that $\pker{\phi - \lambda \id} = S$, meaning $\phi - \lambda \id = 0$. In other words, $\phi = \lambda \id$.  % Explain the logic of this in more detail.
    \end{enumerate}
\end{proof}

\begin{boxtheorem}[Schur's Lemmas for Finite Groups, over $\C$]
    \letfg\ and let $S$ and $T$ be simple $\C G$ modules that are finite-dimensional (as vector spaces) over $K$, with associated representations $\rho_S : G \to \GL{S}$ and $\rho_T : G \to \GL{T}$.
    \begin{enumerate}[label = \normalfont \arabic*.]
        \item If $S \not\cong T$, then for all $\C$-linear maps $f : S \to T$, the map
        \begin{align*}
            \hat{f} := \frac{1}{\abs{G}} \sum_{g \in G} \rho_T(g) \circ f \circ \rho_S\!\parenth{g\inv}
        \end{align*}
        is identically zero.

        \item If $S \cong T$, then for all $\C$-linear maps $f : S \to T$, we have
        \begin{align*}
            \hat{f} &:= \frac{1}{\abs{G}} \sum_{g \in G} \rho_T(g) \circ f \circ \rho_S\!\parenth{g\inv} \\
            &= \frac{1}{\pdim{S}} \Tr{f} \cdot \id_S
        \end{align*}
    \end{enumerate}
\end{boxtheorem}

\begin{proof}
    Let $f : S \to T$ be $\C$-linear. We show that $\hat{f} \in \Hom_{\C G}(S, T)$: for all $h \in G$,
    \begin{align*}
        \rho_T(h) \circ \hat{f} &= \rho_T(h)\!\parenth{\frac{1}{\abs{G}} \sum_{g \in G} \rho_T(g) \circ f \circ \rho_S\!\parenth{g\inv}} \\
        &= \frac{1}{\abs{G}} \sum_{g \in G} \rho_T(hg) \circ f \circ \rho_S\!\parenth{g\inv} \circ \rho_S\!\parenth{h\inv} \circ \rho_S(h) \\
        &= \frac{1}{\abs{G}} \sum_{g \in G} \parenth{\rho_T(hg) \circ f \circ \rho_S\!\parenth{g\inv h\inv}} \circ \circ \rho_S(h) \\
        &= \hat{f} \circ \rho_S(h)
    \end{align*}
    proving that $\hat{f}$ is, indeed, a homomorphism of $\C G$-modules.


\end{proof}
