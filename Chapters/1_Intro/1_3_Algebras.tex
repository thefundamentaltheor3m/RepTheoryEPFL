\section{Group Algebras and Modules}

In this section, we study an important class of field algebras, namely, group algebras, and an important class of modules over said algebras, namely, group modules.

\subsection{Preliminaries}

\begin{boxdefinition}[Group Algebra]
    \letfg\ and let $K$ be a field. The group algebra $KG$ is the $K$-algebra obtained by endowing the free vector space $K[G]$ generated by $G$ (as a set) with the multiplication
    \begin{align*}
        \parenth{\sum_{g \in G} \alpha_g e_g} \cdot \parenth{\sum_{g \in G} \beta_g e_g} &:= \sum_{g \in G} \sum_{h \in G} \alpha_{g} \beta_h e_{gh}
    \end{align*}
\end{boxdefinition}
\begin{remark} \hfill
    \begin{enumerate}[noitemsep]
        \item It is easy to verify that $KG$ is, indeed, a $K$-algebra, with the multiplicative identity given by $e_1$ (where $1 \in G$ is the identity).
        \item The map $g \mapsto e_g : G \to KG$ gives a trivial embedding of $G$ in $KG$.
        \item Going forward, for ease of notation, we will denote each basis element $e_g$ as simply $g$.
    \end{enumerate}
\end{remark}

We have a similar notion of group modules.

\begin{boxdefinition}[Group Module] \label{Ch1:Def:KG-Module}
    \letg\ and let $V$ be a vector space over a field $K$. We say that $V$ is a $KG$-module if we can define a multiplication $g \cdot v$ for some $g \in G$ and $v \in V$ that satisfies the following conditions for all $u, v \in V$, $g, h \in G$ and $\lambda \in K$:
    \begin{enumerate}[noitemsep]
        \item $g \cdot v \in V$
        \item $\parenth{gh} \cdot v = g \cdot \parenth{h \cdot v}$
        \item $1 \cdot v = v$
        \item $g \cdot \parenth{\lambda v} = \lambda \parenth{g \cdot v}$
        \item $g \cdot \parenth{u + v} = g \cdot u + g \cdot v$
    \end{enumerate}
\end{boxdefinition}

Note that a $KG$-module is, indeed, a module over $KG$.

\begin{proposition}
    \letg\ and let $V$ be a vector space over a field $K$. If $V$ is a $KG$-module with multiplication $\cdot$ (as per Definition \ref{Ch1:Def:KG-Module}), then for $v \in V$, the multiplication
    \begin{align*}
        \parenth{\sum_{g \in G} \lambda_g e_g} \cdot v &:= \sum_{g \in G} \lambda_g \parenth{g \cdot v}
    \end{align*}
    endows $V$ with a module structure over $K[G]$.
\end{proposition}

Furthermore, it turns out that we can move from modules to representations and vice-versa quite easily.

\begin{proposition}
    \letg\ and let $V$ be a vector space over a field $K$.
    \begin{enumerate}[label = \normalfont \arabic*.]
        \item If $\rho : G \to \GL{V}$ gives a representation of $G$, then $V$ is a $KG$-module with multiplication given by $g \cdot v = \rhoff{g}{v}$ for all $g \in G$ and $v \in V$.
        \item If $V$ is a $KG$-module with multiplication $\cdot$, the map $\rho : G \to \GL{V}$ given by $\rhoff{g}{v} := g \cdot v$ is a representation.
    \end{enumerate}
\end{proposition}

The proofs of the above propositions are trivial and merely involve manually checking several basic conditions. Hence, we omit them.

We now give a basic `dictionary' of sorts to go back and forth between the language of group modules and that of representations:

\begin{table}[!h]
    \centering
    \begin{tabular}{c|c}
        \textbf{$KG$-Modules} & \textbf{Representations} \\ \hline
        Simple & Irreducible \\
        Semi-Simple & Completely Irreducible \\
        Submodule & Subrepresentation \\
        Viewing $KG$ as a $KG$-Module & The Regular Representation \\
        Isomorphism & Equivalence of Representations \\
        Dimension (as a $K$-vector space) & Degree
    \end{tabular}
\end{table}

We illustrate the above equivalence by stating Maschke's Theorem in the language of $KG$-Modules.

\begin{lemma}[Maschke's Theorem, Module Version]
    \letfg, $K$ a field whose characteristic does not divide the order of $G$. Then, any $KG$-Module $V$ is semi-simple.
\end{lemma}

\subsection{Schur's Lemmas}

In this subsection, we explore several versions of an important result by Schur. We begin by stating it in its most general form.

\begin{theorem}[Schur's Lemmas for Rings]
    Let $A$ be a ring and let $S, T$ be simple $A$-modules.
    \begin{enumerate}[label = \normalfont \arabic*., noitemsep]
        \item If $S$ and $T$ are non-isomorphic, then $\Hom_A(S, T) = \set{0}$.
        \item If $S$ and $T$ are isomorphic, then $\Hom_A(S,T)$ is a division ring.
    \end{enumerate}
\end{theorem}
\begin{proof}
    We rely on the fact that for all $\phi \in \Hom_A(S, T)$, $\pker{\phi} \leq S$ and $\pim{\phi} \leq T$.
    \begin{enumerate}
        \item Let $S$ and $T$ be non-isomorphic. Fix $\phi \in \Hom_A(S,T)$. Since $S$ is simple, we must have that $\pker{\phi} \in \set{\set{0}, S}$. If $\pker{\phi} = \set{0}$, then $\pim{\phi} = T$, meaning $S \cong T$, a contradiction.
        \item Let $\phi \in \Hom_A(S,T) \setminus \set{0}$. Then, $\pker{\phi} \neq S$, meaning that $\pker{\phi} = \set{0}$. Then, $\pim{\phi} = T$, making $\phi$ an isomorphism. In particular, this means that $\phi$ admits an inverse, making $\Hom_A(S,T)$ a division ring.
    \end{enumerate}
\end{proof}

It turns out we can do a bit better when dealing with a specific class of rings, namely, algebras over fields.

\begin{theorem}[Schur's Lemmas for Algebras]
    Let $K$ be an algebraically closed field and $A$ a $K$-algebra. Let $S$ and $T$ be simple $A$-modules.
    \begin{enumerate}[label = \normalfont \arabic*., noitemsep]
        \item If $S \not\cong T$, then $\Hom_A(S, T) = \set{0}$.
        \item If $S \cong T$, then $K \cong \Hom_A(S, T)$ via the map $\alpha \mapsto \alpha \cdot \id$.
    \end{enumerate}
\end{theorem}
\begin{proof}
    \hfill
    \begin{enumerate}
        \item As before.
        \item We do not distinguish $S$ and $T$ in this proof.
        
        Fix $\phi \in \Hom_A(S,S)$. Then, $\phi$ can be viewed as an element of $\Mn{n}{K}$, where $n = \pdim{S}$. Since $K$ is algebraically closed, $\phi$ admits an eigenvalue $\lambda \in K$. Now, consider the map $\phi - \lambda \id \in \Hom_A(S,S)$. Clearly, $\pker{\phi - \lambda \id} \neq \set{0}$, since it contains all eigenvectors with eigenvalue $\lambda$. Since $S$ is simple, it must be that $\pker{\phi - \lambda \id} = S$, meaning $\phi - \lambda \id = 0$. In other words, $\phi = \lambda \id$.  % Explain the logic of this in more detail.
    \end{enumerate}
\end{proof}

We also have a converse when working with algebras.

\begin{theorem}[Converse of Schur's Lemma for Algebras]
    Let $K$ be a field, $A$ a $K$-algebra and $M$ a completely reducible $A$-module. If $\Hom_A(M,M) = K$, then $M$ is simple.
\end{theorem}
\begin{proof}
    \verb|sorry|
\end{proof}

\begin{boxtheorem}[Schur's Lemmas for Finite Groups, over $\C$] \label{SP:Thm:Schur_fin_G_over_C}
    \letfg\ and let $S$ and $T$ be simple $\C G$ modules that are finite-dimensional (as vector spaces) over $K$, with associated representations $\rho_S : G \to \GL{S}$ and $\rho_T : G \to \GL{T}$.
    \begin{enumerate}[label = \normalfont \arabic*.]
        \item If $S \not\cong T$, then for all $\C$-linear maps $f : S \to T$, the map
        \begin{align*}
            \hat{f} := \frac{1}{\abs{G}} \sum_{g \in G} \rho_T(g) \circ f \circ \rho_S\!\parenth{g\inv}
        \end{align*}
        is identically zero.

        \item If $S \cong T$, then for all $\C$-linear maps $f : S \to T$, we have
        \begin{align*}
            \hat{f} &:= \frac{1}{\abs{G}} \sum_{g \in G} \rho_T(g) \circ f \circ \rho_S\!\parenth{g\inv} \\
            &= \frac{1}{\pdim{S}} \Tr{f} \cdot \id_S
        \end{align*}
    \end{enumerate}
\end{boxtheorem}
\begin{proof}
    Let $f : S \to T$ be $\C$-linear. We show that $\hat{f} \in \Hom_{\C G}(S, T)$: for all $h \in G$,
    \begin{align*}
        \rho_T(h) \circ \hat{f} &= \rho_T(h)\!\parenth{\frac{1}{\abs{G}} \sum_{g \in G} \rho_T(g) \circ f \circ \rho_S\!\parenth{g\inv}} \\
        &= \frac{1}{\abs{G}} \sum_{g \in G} \rho_T(hg) \circ f \circ \rho_S\!\parenth{g\inv} \circ \rho_S\!\parenth{h\inv} \circ \rho_S(h) \\
        &= \frac{1}{\abs{G}} \sum_{g \in G} \parenth{\rho_T(hg) \circ f \circ \rho_S\!\parenth{g\inv h\inv}} \circ \circ \rho_S(h) \\
        &= \hat{f} \circ \rho_S(h)
    \end{align*}
    proving that $\hat{f}$ is, indeed, a homomorphism of $\C G$-modules.
\end{proof}

It turns out that Schur's Lemmas are powerful tools in the study of certain classes of representations. We investigate one such class in the next subsection.

\subsection{Representations of Finite Abelian Groups over $\C$}

It is natural to wonder what the purpose was of studying group algebras and modules. It turns out that one of the reasons the correspondence between representations and group modules is so powerful is that it allows the application of Schur's Lemmas to representation theoretic problems. For instance, in the following Lemma, we classify all irreducible representations of finite abelian groups over $\C$.

\begin{lemma} \label{Ch1:Lem:Fin_Ab_over_C}
    Let $G$ be a fininte abelian group. Then, all irreducible $\C G$-modules are of dimension $1$. Equivalently, all irreducible representations of $G$ over $\C$ are of degree $1$.
\end{lemma}
\begin{proof}
    Let $V$ be an irreducible $\C G$-module. Since $G$ is abelian, for all $g,h \in G$ and $v \in V$, $\parenth{gh} \cdot v = \parenth{hg} \cdot v$. Therefore, for some fixed $h \in G$, the following map is $\C G$-linear:
    \begin{align*}
        \phi_h : V \to V : v \mapsto h \cdot v
    \end{align*}
    By Theorem \ref{SP:Thm:Schur_fin_G_over_C}, we know that $\exists \lambda_h \in \C$ such that $\widehat{\phi_h} = \phi_h = \lambda_h \cdot \id_V$. Hence, any subspace of $V$ must be a $\C G$-submodule. But, since $V$ is irreducible, $V$ cannot admit any nonzero, proper $\C G$-submodules unless $V$ is of ($\C$-)dimension $1$.
\end{proof}
We now have a complete classification of irreducible representations of cyclic groups over $\C$.
\begin{corollary}
    Let $G = C_n = \cycl{a}$ be the cyclic group of order $n$. Then, there are precisely $n$ irreducible representations of $G$ over $\C$.
    \begin{proof}
        \letgv. We know, from Lemma \ref{Ch1:Lem:Fin_Ab_over_C}, that $V \cong \C$ and hence, $\GL{V} \cong \C^\times$. Now, let $x := \rho(a)$. It must be that $x^n = 1$, making $x$ an $n$th root of unity. In other words, $\exists 1 \leq k \leq n$ such that $x = e^{\frac{2\pi i}{k}}$. Therefore, there are precisely $n$ possible choices of $x$, each giving a different representation.
    \end{proof}
\end{corollary}

Lemma \ref{Ch1:Lem:Fin_Ab_over_C} is also useful in the study of representations over $\C$ of arbitrary finite groups.

\begin{proposition}
    \letfgv\ over $\C$. For all $g \in G$, there is a basis of $V$ with respect to which $\rho(g)$ has matrix $\diag{\eps_1, \cdots, \eps_n}$, with $\eps_i^{\ord{g}} = 1$ for all $1 \leq i \leq n$.
\end{proposition}
\begin{proof}
    Fix $g \in G$, and consider the representation $\rho' : \cycl{g} \to \GL{V}$ given by $\rho' = \rho\vert_{\cycl{g}}$. Then, $\rho'$ is a representation of a finite abelian group.

    By Maschke's Theorem, $\rho' = \sigma_1 \+ \cdots \+ \sigma_k$ for irreducible subrepresentations $\sigma_1, \ldots, \sigma_m$ of $\cycl{g}$. By Lemma \ref{Ch1:Lem:Fin_Ab_over_C}, we know that $\pdeg{\sigma_i} = 1$ for each $i$, and hence, that $m = n$. Picking $\B$ to be the basis corresponding to this decomposition of $\rho'$, we get that the matrix of $\rho$ with respect to $\B$ is, indeed, of the desired form.
\end{proof}

As it turns out, we can combine the theory developed here with the Structure Theorem for Finite Abelian Groups to get an interesting result.

\begin{lemma}
    Let $G$ be a finite abelian group, expressed as a product $C_{n_1} \times \cdots \times C_{n_r}$ of cyclic groups $C_{n_i}$ of order $n_i > 1$. Then, $G$ has a faithful representation of degree $r$ over $\C$.
\end{lemma}
\begin{proof}
    Consider the space $V = \C^r = \C_1 \+ \cdots \+ \C_r$ (where each $\C_i$ is the one-dimensional subspace spanned by the $i$th element of some chosen $\C$-basis for $V$). Let $C_{n_i} = \cycl{g_i}$ and denote by $e_i$ the corresponding generators $\parenth{1, \ldots, 1, g_i, 1, \ldots, 1}$ of $G$. Define the map
    \begin{align}
        \rho : G \to \C : e_i \mapsto R_{i}
    \end{align}
    where $R_i$ is the rotation by $2\pi / n_i$ acting on the subspace $\C_i \cong \C$. In other words, with respect to the isomorphism $\GL{\C_i} \cong \C^{\times}$, the map $R_i$ corresponds to $\pexp{{2\pi}/{n_i}}$.\footnote{To be perfectly precise, $e_i$ is mapped not to $R_i$ but to the image of $R_i$ in the inclusion $\GL{\C_i} \to \GL{V}$ that extends $R_i$ by acting as the identity on components other than $i$ and as $R_i$ on component $i$. We use the word `component' to refer to a one-dimensional direct summand $\C_j$ of $V$.}

    $\rho$ has the following effect on group elements: for two group elements acting on the same component of $V$, $\rho$ maps their product to the composition of their associated rotations, and for elements acting on different components, $\rho$ combines their componentwise actions into a single action across two components. Therefore, $\rho$ is a group homomorphism, and hence, $\Vp$ is a representation of $G$ over $\C$.

    We now show that $\rho$ is injective. If $g \in G$ acts identically on all of $V$, it must, in particular, act identically on each component. But, the action of $g$ on each $\C_j$ is merely the action of the $j$th component of $g$ on $\C_j$. It is easy to see that the componentwise actions of $\rho$ on $V$ are all faithful, meaning that each component of $g$ is the identity in its respective cyclic group. Therefore, $g$ must be the identity in $G$, making $\rho$ a faithful representation.
\end{proof}
