\section{The Theory of Orthogonal Characters}

\subsection{On Central Functions}

\begin{boxdefinition}[Central Function]
    We say $f : G \to \C$ is central if $\fx = \fof{g\inv x g}$ for all $x,g \in G$.
\end{boxdefinition}

\begin{boxexample}[Examples of Central Functions]
    \hfill
    \begin{enumerate}
        \item The character of a $\C G$-module
        \item The order function $x \mapsto \ord{x} : G \to \N \subset \C$
    \end{enumerate}
\end{boxexample}

\begin{boxnotation}
    We define
    \begin{enumerate}
        \item $\Fof{G, \C} := \set{f : G \to \C}$
        \item $\Fcof{G, \C} := \set{f \in \Fof{G, \C} : f \text{ is central}}$
        \item $\delta_g(x)$ to be the indicator function (for $g, x \in G$).
    \end{enumerate}
\end{boxnotation}

\begin{remark}
    It turns out that $\Fof{G, \C}$ has the natural structure of being a $\C$-vector space of dimension $\abs{G}$, with basis $\set{\delta_g : g \in G}$.
\end{remark}

\begin{definition}
    Define $\cycl{\cdot, \cdot} : \FGC \times \FGC \to \C$ By
    \begin{align}
        \cycl{f_1, f_2} = \frac{1}{\abs{G}} \sum_{g \in G} f_1(g) \overline{f_2(g)}
    \end{align}
    One can show this to be an inner-product on $\FGC$.
\end{definition}

\subsection{The Orthogonality Theorem}

\begin{lemma}
    Let $\rho : G \to \GL{n, \C}$ and $\rho' : G \to \GL{m, \C}$ be irreducible representations of $G$ over $\C$. Fix $j, s \in \set{1, \ldots, m}$ and $r, i \in \set{1, \ldots, n}$.
    \begin{enumerate}[label = \normalfont \arabic*., noitemsep]
        \item If $\rho$ and $\rho'$ are \textit{not} equivalent, then
        \begin{align*}
            \frac{1}{\abs{G}} \sum_{g \in G} {\rho(g)}_{ri} {\rho'\!\parenth{g\inv}}_{js} = 0
        \end{align*}

        \item If $\rho$ and $\rho'$ \textit{are} equivalent, then
        \begin{align*}
            \frac{1}{\abs{G}} \sum_{g \in G} {\rho(g)}_{ri} {\rho'\!\parenth{g\inv}}_{js} =
            \begin{cases}
                \frac{1}{n} & \text{if } i = j \text{ and } r = s \\
                0 & \text{otherwise}
            \end{cases}
        \end{align*}
    \end{enumerate}
    where we use the notation $T_{ij}$ to refer to the $ij$th entry of the matrix of $T$.
\end{lemma}
\begin{proof}
    Let $V = \C^n$ and $W = \C^m$ be the two simple $\C G$-modules corresponding to $\rho$ and $\rho'$ respectively. The idea is to define a linear map that will allow us to use Schur's Lemma.

    For some chosen bases of $V$ and $W$, let $\phi_{ij} : W \to V$ be the $\C$-linear map given by the $n \times m$ matrix with $ij$th entry $1$ and all other entries $0$. Define
    \begin{align*}  % TODO: Make hat look nice
        \hat{\phi_{ij}} := \frac{1}{\abs{G}} \sum_{g \in G} \rho(g) \circ \phi_{ij} \circ \rho'\!\parenth{g\inv}
    \end{align*}
    By Theorem \ref{SP:Thm:Schur_fin_G_over_C}, $\hat{\phi_{ij}}$ is a $\C G$-module homomorphism from $W$ to $V$.
    \begin{enumerate}
        \item If $W \not\cong V$, we have $\hat{\phi_{ij}} = 0$. In particular,
        \begin{align*}
            0 &= \parenth{
                \frac{1}{\abs{G}} \sum_{g \in G} \rho(g) \circ E_{ij} \circ \rho'\!\parenth{g\inv}
            }_{rs} \\
            &= \frac{1}{\abs{G}} \sum_{g \in G} \brac{\rho(g) \circ E_{ij} \circ \rho'\!\parenth{g\inv}}_{rs} \\
            &= \sum_{k=1}^{n} \sum_{l=1}^{m} \frac{1}{\abs{G}} \sum_{g \in G} \brac{\rho(g)}_{rk} \brac{E_{ij}}_{kl} \brac{\rho'\!\parenth{g\inv}}_{ls} \\
            &= \frac{1}{\abs{G}} \sum_{g \in G} \brac{\rho(g)}_{ri} \brac{\rho'\!\parenth{g\inv}}_{js}
        \end{align*}

        \item Similarly, if $W \cong V$, we can view $\phi_{ij}$ as being given by the $n \times n$ matrix $E_{ij}$. Now, by Theorem \ref{SP:Thm:Schur_fin_G_over_C}, we know that
        \begin{align*}
            \hat{\phi_{ij}} &= \frac{1}{n} \Tr{E_{ij}} \cdot \id_V \\
            &= \frac{1}{n} \delta_{ij} \cdot \id_V
        \end{align*}
        We can then show the desired result using a similar computation.
    \end{enumerate}
\end{proof}
\begin{remark}
    As per Dr. Rizzoli, on the exam, it's more important to know the idea of such a proof than the specifics of \textit{which index goes where}.
\end{remark}

\begin{boxtheorem}[Orthogonality Theorem] \label{Ch2:Thm:Orth_Char}
    Let $S, T$ be irreducible $\C G$-modules.
    \begin{enumerate}[label = \normalfont \arabic*., noitemsep]
        \item If $S \not\cong T$, then $\cycl{\chi_S, \chi_T} = 0$.
        \item If $S \cong T$, then $\cycl{\chi_S, \chi_T} = 1$
    \end{enumerate}
    In other words, irreducible characters form an orthogonal system.
\end{boxtheorem}
\begin{proof}
    Let $P : G \to \GL{n, \C}$ and $Q : G \to \GL{m, \C}$ be the representations corresponding to $S$ and $T$. We know that
    \begin{align*}
        \cycl{\chi_S, \chi_T} &=
        \frac{1}{\abs{G}} \sum_{G \in G} \chi_S(g) \chi_T\!\parenth{g\inv} \\
        &= \frac{1}{\abs{G}} \sum_{g \in G} \Tr{P(g)} \Tr(Q\!\parenth{g\inv}) \\
        &= \frac{1}{\abs{G}} \sum_{g \in G} \parenth{\sum_{i = 1}^{n} \brac{P(g)}_{ii}}\parenth{\sum_{j=1}^{n} \brac{Q\!\parenth{g\inv}}_{jj}} \\
        &= \sum_{i=1}^{n} \sum_{j=1}^{m} \frac{1}{\abs{G}} \sum_{g \in G} \brac{P(g)}_{ii} \brac{Q\!\parenth{g\inv}_{jj}}
    \end{align*}
    We can then use the previous lemma to evaluate these sums.
\end{proof}

We have the following important corollary.

\begin{corollary}
    Up to isomorphism, there are finitely many irreducible $\C G$-modules.
\end{corollary}

\subsection{Irreducible Characters}

\begin{boxnotation}[Irreducilbe Characters]
    Denote by $\Irr{G}$ the subset of $\Fcof{G, \C}$ consisting of irreducible characters of $G$.
\end{boxnotation}

The Orthogonality Theorem tells us the following facts about Irreducible Characters.

\begin{proposition}[On the Behaviour of Irreducible Characters] \label{Ch2:Prop:Bhv_Irred_Char}
    \hfill
    \begin{enumerate}[label = \normalfont \arabic*., noitemsep]
        \item $\Irr{G}$ is a linearly independent set. In particular, $\abs{\Irr{G}} \leq \abs{G}$.
        
        \item Let $V = V_1 \+ \cdots \+ V_r$ be a $\C G$ module, with $V_i$ simple for $1 \leq i \leq r$. For any simple $\C G$ module $S$, the number of $V_i$s isomorphic to $S$ is given by $\cycl{\chi_V, \chi_S}$.
        
        \item Let $V, V'$ be $\C G$-modules. Then, $V \cong V' \iff \chi_V = \chi_{V'}$.
        
        \item A \CGM\ $V$ is simple iff $\cycl{\chi_V, \chi_V} = 1$.
    \end{enumerate}
\end{proposition}
\begin{proof}
    \hfill
    \begin{enumerate}[label = \normalfont \arabic*.]
        \item The linear independence follows immediately from the fact that $\Irr{G}$ form an orthonormal system. The inequality follows from the fact that all central functions are class functions: they agree for all conjugate elements of $G$. This means that $\pdim{\Fcof{G, \C}}$ is simply the number of conjugacy classes of $G$. Since $\pdim{\Fcof{G, \C}}$ must be at least $\abs{\Irr{G}}$ and the number of conjugacy classes of $G$ must be at most $\abs{G}$, we have the desired result.
        
        \item We know that $\chi_V = \chi_{V_1} + \cdots + \chi_{V_r}$. So,
        \begin{align*}
            \cycl{\chi_V, \chi_S} &= \cycl{\chi_{V_1} + \cdots + \chi_{V_r}, V_S} \\
            &= \text{sorry}
        \end{align*}

        \item 
        \begin{description}
            \item[$\parenth{\implies}$] Already seen. % TODO: PUT REFERENCE!
            \item[$\parenth{\impliedby}$] % Let $\Irr{G} = \set{\chi_1, \ldots, \chi_r}$.
            The multiplicity\footnote{Ie, the number of times it appears in the direct sum decomposition of $V$ into simple $\C G$-modules} of a simple \CGM\ $S$ in $V$ is given by $\cycl{\chi_V, \chi_S} = \cycl{\chi_{V'}, \chi_S}$. Then, a simple \CGM\ must have the same multiplicity in both $V$ and $V'$. % TODO: REFINE!!
        \end{description}

        \item 
        \begin{description}
            \item[$\parenth{\implies}$] Already seen. % TODO: PUT REFERENCE!
            \item[$\parenth{\impliedby}$] Let $\Irr{G} = \set{\chi_1, \ldots, \chi_r}$, with corresponding simple \CGM s $\set{V_1, \ldots, V_r}$. Denoting by $a_i$ the multiplicity of each $V_i$ in $V$, we have that
            \begin{align*}
                V &= V_1^{\+ a_1} \+ \cdots \+ V_r^{\+ a_r}
            \end{align*}
            where we use the notation $V_i^{\+ a_i}$ to mean $\underbrace{V_i \+ \cdots \+ V_i}_{a_i \text{ times}}$.

            We then have
            \begin{align*}
                \chi_V &= \sum_{i=1}^{r} a_i \chi_i \\
                \implies \cycl{\chi_V, \chi_V} &= \sum_{i=1}^{r} a_i^2
            \end{align*}
            This means only one of the $a_i$s is nonzero, and equal to $1$. % WHAT????
        \end{description}
    \end{enumerate}
\end{proof}

We now have everything we need to prove the following important theorem.

\begin{theorem}
    $\Irr{G}$ is a basis for $\Fcof{G, \C}$.
\end{theorem}
\begin{proof}
    Let $W = \Span{\Irr{G}} \leq \Fcof{G, \C}$, with orthogonal complement $W^\perp$ with respect to the inner-product \verb|insert reference|. Since $V = W \+ W^\perp$, if we can show that $W^\perp = \set{0}$, we would have that $V = W$, proving the desired result.

    Fix $f \in W^\perp$, and consider the element $\hat{f} \in \C G$ given by  % CHANGE NOTATION!
    \begin{align*}
        \hat{f} &= \sum_{g \in G} \overline{f(g)} \cdot g
    \end{align*}

    First, we show that $\hat{f} \in \Zof{\C G}$---ie, that $\hat{f}$ commutes (multiplicatively) with all elements of $\C G$. To show this, it suffices to show that $h\inv \hat{f} h = \hat{f}$ for all $h \in G$. So, fix $h \in G$. Then,
    \begin{align*}
        h\inv \hat{f} h &=
        \sum_{g \in G} \overline{f(g)} \cdot h\inv g h \\
        &= \sum_{g \in G} \overline{\fof{h\inv g h}} \cdot h\inv g h \\ % Can take h\inv g h inside f because f is central. Mention somewhere.
        &= \hat{f}(g)
    \end{align*}
    where the last equality follows from a change of variables in the summation.

    Now, let $S$ be any simple \CGM. One can show that the map  % DOMAIN AND CODOMAIN????
    \begin{align*}
        \phi : S \to S : v \mapsto \hat{f} \cdot v
    \end{align*}
    is a \CGM\ homomorphism. % Include more details
    Then, by Theorem \ref{SP:Thm:Schur_fin_G_over_C}, we have that
    \begin{align*}
        \hat{f} &= \frac{1}{\pdim{S}} \Tr{\hat{f}\vert_S} \cdot \id
    \end{align*}
    We then have that
    \begin{align*}
        \Tr{\hat{f}\vert_S} &= \Tr{\sum_{g \in G} \overline{f(g)} \cdot g\vert_{S}} \\
        &= \sum_{g \in G} \overline{f(g)} \chi_S(g) \\
        &= \abs{G} \cycl{\underbrace{\chi_S}_{\in W}, \underbrace{f}_{\in W^\perp}} = 0
    \end{align*}
    proving that in fact, $\hat{f}\vert_S = 0$.
    
    \verb|sorry| % Use photo on phone to complete proof.
\end{proof}