\section{Character Tables}

In Example~\ref{Ch2:Eg:Cycl_3_Char}, we constructed a table consisting of the character values of each conjugacy class of the cyclic group of order $3$ (in this case, each class consisted of only one element---but this is not always the case). Our approach in Example~\ref{Ch2:Eg:Cycl_3_Char} was direct computation, which, while straightforward, is not always practical---especially when dealing larger groups. In this section, we will introduce a more systematic way of constructing such tables using the orthogonality properties of irreducible characters.

\subsection{The Orthogonality Relations}

For the purposes of this subsection, let $C_1, \ldots, C_k$ be the conjugacy classes of $G$ with representatives $g_1, \ldots, g_k$ respectively. Recall that by the Orbit-Stabiliser Theorem,
\begin{align*}
    \abs{C_i} &= \frac{\abs{G}}{\abs{C_G(g_i)}}
\end{align*}
where $C_G(\cdot)$ refers to the centraliser\footnote{Recall that the centraliser of a group element is the set of all elements that commute with it.} of an element in $G$. Now, let $\Irr{G} = \set{\chi_1, \ldots, \chi_k}$.

\begin{boxdefinition}[Character Table]
    A character table is a $k \times k$ table whose columns are the conjugacy classes (or their representatives) and whose rows are the irreducible characters.
    \begin{table}[H]
        \centering
        \begin{tabular}{c|ccc}
            & $g_1$ & $\cdots$ & $g_k$ \\
            \hline
            $\chi_1$ & $\chiof{1}{g_1}$ & $\cdots$ & $\chiof{1}{g_k}$ \\
            $\vdots$ & $\vdots$ & $\ddots$ & $\vdots$ \\
            $\chi_k$ & $\chiof{k}{g_1}$ & $\cdots$ & $\chiof{k}{g_k}$
        \end{tabular}
    \end{table}
    In other words, it is a table whose $\parenth{i,j}$th entry is $\chi_i(g_j)$.
    % Maybe draw this...
\end{boxdefinition}

The table in Example~\ref{Ch2:Eg:Cycl_3_Char} is an example of a character table.

It turns out that we have a certain orthogonality property as we fix two rows and sum across the columns of a character table.

\begin{boxproposition}[First Orthogonality Relation]\label{SP:1st_Orth_Rel}
    Given $1 \leq r, s \leq k$, we have
    \begin{align*}
        \frac{1}{\abs{G}} \sum_{j=1}^{k} \abs{C_j} \chi_r\!\parenth{g_j} \chi_s\!\parenth{g_j\inv} &= \delta_{rs}
    \end{align*}
\end{boxproposition}
\begin{proof}
    The proof is quite straightforward. It merely involves applying the centrality of characters to the definition of the inner-product and concluding using the Orthogonality Theorem:
    \begin{align*}
        \delta_{rs} &=
        \cycl{\chi_r, \chi_s} \\
        &= \frac{1}{\abs{G}} \sum_{g \in G} \chi_r(g) \chi_s\!\parenth{g\inv} \\
        &= \frac{1}{\abs{G}} \sum_{j=1}^{k} \parenth{\sum_{g \in C_i} \chi_r(g) \chi_s\!\parenth{g\inv}} \\
        &= \frac{1}{\abs{G}} \sum_{j=1}^{k} \abs{C_j} \chi_r\!\parenth{g_j} \chi_s\!\parenth{g_j\inv}
    \end{align*}
    where the last inequality follows from the fact that characters are central (ie, they are equal for all elements of a conjugacy class).
\end{proof}
\begin{remark}
    Note that the result of Proposition~\ref{SP:1st_Orth_Rel} can be equivalently phrased in the following manner: given $1 \leq r, s \leq k$,
    \begin{align}
        \sum_{j=1}^{k} \frac{1}{\abs{C_G(g_j)}} \chi_r\!\parenth{g_j} \chi_s\!\parenth{g_j\inv} &= \delta_{rs}
        \label{SP:Eq:1st_Orth_Rel_Conj_Cl}
    \end{align}
\end{remark}

\begin{comment}
We can use this to prove an important result about the regular representation.

\begin{theorem}\label{Ch2:Thm:Reg_Rep_Col_1}
    Let $\chi_{\reg}$ denote the character of the regular representation of $G$ over $\C$. Then,
    \begin{align*}
        \chi_{\reg} &= \sum_{j=1}^{k} \chiof{j}{1} \chi_j
    \end{align*}
\end{theorem}
\begin{proof}
    
    \verb|sorry|  % From phone
\end{proof}

\begin{corollary} \label{Ch2:Cor:Reg_Rep_Col_1_Grp_Size}
    $\sum_{i=1}^{k} \parenth{\chiof{i}{1}}^2 = \abs{G}$.
\end{corollary}
\begin{proof}
    We evaluate both sides of the expression in Theorem~\ref{Ch2:Thm:Reg_Rep_Col_1} at $1$ to obtain that
    \begin{align*}
        \chiof{\reg}{1} = \sum_{i=1}^{k} \parenth{\chi_i(1)}^2 = \abs{G}        
    \end{align*}
\end{proof}
\end{comment}

Indeed, we have a similar result about fixing two columns and summing down the rows.

\begin{boxproposition}[Second Orthogonality Relation]\label{SP:2nd_Orth_Rel}
    Given $1 \leq r, s \leq k$, we have
    \begin{align*}
        \sum_{i=1}^{k} \chiof{i}{g_r} \chiof{i}{g_s\inv} =
        \begin{cases}
            0 &\text{ if } r \neq s \\
            \abs{C_G(g_r)} &\text{ if } r = s
        \end{cases}
    \end{align*}
\end{boxproposition}
\begin{proof}
    Let $A$ be the matrix $\parenth{A_{ij}}_{1 \leq i, j \leq k}$, where $A_{ij} := \chiof{i}{g_j}$. Similarly, let $B$ be the matrix $\parenth{B_{ij}}_{1 \leq i, j \leq k}$, where $B_{ij} := \frac{\abs{C_i}}{\abs{G}} \chiof{j}{g_i\inv}$. Then, writing $AB = \parenth{AB}_{pq}$, we have
    \begin{align*}
        \parenth{AB}_{pq} &=
        \sum_{l=1}^{k} A_{pl} B_{lq} \\
        &= \sum_{l=1}^{k} \chiof{p}{g_l} \frac{\abs{C_l}}{\abs{G}} \chiof{q}{g_l\inv} \\
        &= \cycl{\chi_p, \chi_q} = \delta_{pq}
    \end{align*}
    This proves that in fact, $AB = I$, the identity matrix. This, in particular, means that $BA = I$ as well. It turns out that setting $\delta_{pq} = \parenth{BA}_{pq}$ gives us the desired result. % TODO: COMPLETE!!!!!!!!!
\end{proof}
\begin{remark}
    Note that the result of Proposition~\ref{SP:2nd_Orth_Rel} can be equivalently phrased in the following manner: given $1 \leq r, s \leq k$,
    \begin{align}
        \sum_{j=1}^{k} \frac{1}{\abs{C_G(g_r)}} \chiof{i}{g_r} \chiof{i}{g_s\inv} &= \delta_{rs}
        \label{SP:Eq:2nd_Orth_Rel_Conj_Cl}
    \end{align}
    \eqref{SP:Eq:2nd_Orth_Rel_Conj_Cl} bears a striking similarity to \eqref{SP:Eq:1st_Orth_Rel_Conj_Cl}. The only difference is that in \eqref{SP:Eq:1st_Orth_Rel_Conj_Cl}, we fix the character and sum over the conjugacy classes (ie, fix the row and sum across the columns), while in \eqref{SP:Eq:2nd_Orth_Rel_Conj_Cl}, we fix the conjugacy class and sum over the characters (ie, fix the column and sum across the rows).
\end{remark}

In the next subsection, we will demonstrate how we can use these orthogonality relations, in combination with other results---most notably, \eqref{Ch2:Eq:Grp_Order_Sum_Char_Sq}---to construct character tables.

\subsection{Character Tables of Symmetric Groups}

\begin{boxexample}[$S_3$]
    $S_3$ has precisely three conjugacy classes $C_1$, $C_2$ and $C_3$ with representatives $g_1 = 1$, $g_2 = \parenth{1 2}$ and $g_3 = \parenth{1 2 3}$. We then have the following character table for $S_3$:
    \begin{table}[H]
        \centering
        \begin{tabular}{c|ccc}
            & $1$ & $\parenth{12}$ & $\parenth{123}$ \\
            \hline
            $\chi_1$ & $1$ & $1$ & $1$ \\
            $\chi_2$ & $1$ & $-1$ & $1$ \\
            $\chi_3$ & $2$ & $0$ & $-1$
        \end{tabular}
    \end{table}
    Here, $\chi_1$ corresponds to the trivial representation (of degree $1$), $\chi_2$ to the sign representation (of degree $1$), and $\chi_3$ to the subrepresentation $W = \Span{e_1 - e_2, e_1 - e_3}$ (of degree $2$) of the permutation representation on $\C^3$. Note that all of these representations are irreducible, the first two because their degrees are $1$, and the third because we can show $\cycl{\chi_W, \chi_W}$ to be equal to $1$ (cf. the fourth point of Proposition \ref{Ch2:Prop:Bhv_Irred_Char}).
\end{boxexample}


\begin{boxexample}[$S_4$]
    Observe that the conjugacy classes $C_i$, $i = 1, \ldots, 5$, of $S_4$ correspond precisely to the various possible cycle shapes. They therefore have representatives $g_1 = 1$, $g_2 = \parenth{12}$, $g_3 = \parenth{123}$, $g_4 = \parenth{12}\parenth{34}$, and $g_5 = \parenth{1234}$. The conjugacy classes have sizes $1$, $6$, $8$, $3$ and $6$ respectively.

    We then have the following character table for $S_4$:
    \begin{table}[H]
        \centering
        \begin{tabular}{c|ccccc}
            & $1$ & $\parenth{12}$ & $\parenth{123}$ & $(12)(34)$ & $(1234)$ \\
            \hline
            $\chi_1$ & $1$ & $1$ & $1$ & $1$ & $1$ \\
            $\chi_2$ & $1$ & $-1$ & $1$ & $1$ & $-1$ \\
            $\chi_3$ & $3$ & $1$ & $0$ & $-1$ & $-1$ \\
            $\chi_4$ & $2$ & $0$ & $-1$ & $2$ & $0$ \\
            $\chi_5$ & $a$ & $b$ & $c$ & $d$ & $e$
        \end{tabular}
    \end{table}
    where the characters $\chi_i$ are as follows:
    \begin{enumerate}
        \item $\chi_1$ is the \underline{trivial character}.
        
        \item $\chi_2$ is the \underline{sign character}\footnote{ie, that corresponding to the sign representation}.
        
        \item $\chi_3$ is the \underline{deleted permutation character} given by the subrepresentation of the permutation representation\footnote{ie, that corresponding to its action on $\C^4$ by permuting the standard basis} of $S_4$ corresponding to $\Span{e_1 - e_2, e_2 - e_3, e_3 - e_4}$, an $S_4$-invariant subspace of $\C^4$.
        
        \item $\chi_4$ is given by the following construction. Consider
        \begin{align*}
            N = \set{1, (12)(34), (13)(24), (14)(23)} \nsg S_4
        \end{align*}
        One can show that $\quotient{S_4}{N} \cong S_3$. Let $\pi : S_4 \surj S_3$ be the associated quotient map. Then, if $\rho : S_3 \to \GL{2, \C}$ is a $2$-dimensional irreducible representation of $S_3$, then the map $\rho' := \rho \circ \pi : S_4 \to \GL{2, \C}$ is also irreducible. We take $\chi_4$ to be the associated character.

        \item $\chi_5$ is the \underline{regular character}. We can actually solve for $a,b,c,d,e$ using the Orthogonality Relations. From \eqref{Ch2:Eq:Grp_Order_Sum_Char_Sq}, we know that
        \begin{align*}
            1 + 1 + 9 + 4 + a^2 = 24 = \abs{S_4}
        \end{align*}
        from which we can conclude that $a = 3$. Then, using the First Orthonality Relation (Proposition \ref{SP:1st_Orth_Rel}), we can see that
        \begin{align*}
            1 - 1 + 3 + 0 + ab = 0
        \end{align*}
        from which we can conclude that $b = -1$.

        \verb|sorry|
    \end{enumerate}
\end{boxexample}

Note that the number of irreducible characters of degree $1$ of any group $G$ is $\abs{G} / \abs{G'}$, where $G'$ is the derived subgroup of $G$. Indeed, there is a one-to-one correspondence between degree $1$ irreducible characters of $G$ and those of $\quotient{G}{G'}$.

It turns out that we can glean even more information about groups from their characters, for which we will need the properties of the algebraic integers.