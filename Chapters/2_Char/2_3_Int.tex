\section{Integrality}

\subsection{The Algebraic Integers}

First, we recall the notion of integrality over an integral domain.

\begin{boxdefinition}[Integrality]
    Let $R$ be an integral domain, and $S \supset R$ an extension. We say that $s \in S$ is integral over $R$ if one of the following equivalent conditions is satisfied:
    \begin{itemize}
        \item $s$ is a root of a monic polynomial in $R[X]$.
        \item The minimal polynomial of $s$ over $\Frac{R}$ is actually in $R[X]$.
    \end{itemize}
\end{boxdefinition}
We now define what it means for a complex number to be an algebraic integer.
\begin{boxdefinition}[Algebraic Integer]
    We say that a number $\alpha \in \C$ is an algebraic integer if it is integral over $\Z$.
\end{boxdefinition}

\begin{boxnotation}
    Given a conjugacy class $C$ of $G$, we define
    \begin{align}
        \hat{C} := \sum_{g \in C} g
    \end{align}
    to be an element of $\C G$.
\end{boxnotation}

\begin{lemma}
    Let $g \in G$ and let $C = g^G$ % Conjugacy class of g?
    Let $S$ be a simple \CGM. Then, for all $s \in S$, we have an action by scalar multiplication
    \begin{align*}
        \hat{C} \cdot s = \lambda s
    \end{align*}
    where $\displaystyle \lambda = \frac{\abs{C}}{\abs{C_G(g)}} \frac{\chi(g)}{\chi_s(1)} = \abs{C} \frac{\chi_s(g)}{\chi_s(1)}$.
\end{lemma}
\begin{proof}
    Define a function $\phi : S \to S : s \mapsto \hat{C} \cdot s$. Since $\hat{C} \in \Zof{\C G}$, % EXERCISE!
    we have that $\forall x \in G$, $x \cdot \phi(s) = \phi(x \cdot s)$. This makes $\phi$ a \CGM\ homomorphism. Then, by Schur's Lemmas, we know that $\phi = \lambda \id$. This means that... % finish
\end{proof}

\begin{lemma}
    Let $r = \sum_{g \in G} \alpha_g g$ for some $\alpha_g \in \Z$. Suppose that $\exists \lambda, v \in \C G \setminus \set{0}$ such that $rv = \lambda v$. Then, $\lambda$ is an algebraic integer.
\end{lemma}
\begin{proof}
    Let $G = \set{g_1, \ldots, g_n}$. For all $1 \leq i \leq n$,
    \begin{align*}
        r g_i &= \sum_{j=1}^{n} \alpha_{ij} g_j
    \end{align*}
    The key observation here is that $\alpha_{ij} \in \Z$ for all $1 \leq i,j \leq n$. % WHY????
    Then, if $rv = \lambda v$, we have that $\lambda$ is an eigenvalue of the matrix $A := \parenth{\alpha_{ij}}_{1 \leq i,j \leq n} \in \mat{n}{n}{\Z}$. This makes $\lambda$ a root of the characteristic polynomial of $A$ (over $\Z$), which is monic and of degree $n$.
\end{proof}

What we can gather from lemmas $1$ and $2$ is that for any $\chi \in \Irr{G}$ and $g \in G$, the quantity $\lambda = \frac{\abs{G}}{\abs{C_G(g)}} \frac{\chi(g)}{\chi(1)}$ is an algebraic integer. This leads us to the following important connection between algebraic integers and irreducible representations.

\begin{theorem} \label{Ch2:Thm:Char_div_Ord_Grp}
    For all $\chi \in \Irr{G}$, $\chi(1) \vert \abs{G}$.
\end{theorem}
\begin{proof}
    \verb|Check phone|
\end{proof}

\subsection{The $n$th Roots of Unity}

\begin{lemma} \label{Ch3:Lem:Int_Sum_Coprime_Roots_Unity}
    Let $w$ be an $n$th root of unity in $\C$. Then,
    \begin{align*}
        \sum_{1 \leq i \leq m, \parenth{i,n} = 1} w^i \in \Z
    \end{align*}
\end{lemma}
\begin{proof}
    EXERCISE. Hint: induction on $n$.
\end{proof}

\begin{proposition}
    Let $g \in G$ be of order $n$. Suppose that $g^i$ is conjugate to $g$ for all $1 \leq i \leq n$ such that $\parenth{i,n} = 1$. Then, for all characters $\chi$ of $G$, $\chi(g)$ is an integer.
\end{proposition}
\begin{proof}
    Let $\chi$ be a character of $G$, with associated representation $\Vp$ of degree $m$ over $\C$. We know that $\chi(g) = \sum_{i=1}^{m} w_i$, where each $w_i$ is an $n$th root of unity (cf. Proposition \ref{Ch2:Prop:Bhv_Char}). Indeed, the idempotent linear map $\rho(g)$ is diagonalisable, with matrix $\diag{w_1, \ldots, w_m}$ with respect to an appropriate basis. Hence, we have that $\rhoof{g^i} = \diag{w_1^i, \ldots, w_m^i}$, meaning that $\chi\!\parenth{g^i} = \sum_{j=1}^{m} w_j^i$. But this is nothing but $\sum_{1 \leq i \leq m, \parenth{i,n} = 1} w^i$, by assumption. Hence, $\chi(g)$ is an integer, by Lemma \ref{Ch3:Lem:Int_Sum_Coprime_Roots_Unity}.
\end{proof}

% \subsection{Congruence Conditions}

\begin{lemma} \label{Ch2:Lem:p_and_p'_parts}
    Fix $g \in G$, and let $p$ be a prime number. Then, $\exists! x, y \in G$ such that
    \begin{enumerate}[label= \normalfont \arabic*., noitemsep]
        \item $g = xy = yx$
        \item $\ord{x} = p^k$ for some $k \in \N$
        \item $\parenth{\ord{y}, p} = 1$
    \end{enumerate}
\end{lemma}
\begin{proof}
    Let $\ord{g} = u p^v$, where $\parenth{u,v = 1}$. By Bézout's Lemma, we know that $\exists a, b \in \Z$ such that $au + bp^v = 1$. % WHY?
    Then, we can define $x = g^{au}$ and $y = g^{bp^v}$. This satisfies all the conditions:
    \begin{enumerate}
        \item $g = xy = g^{au} g^{bp^v} = g^{bp^v} g^{au} = yx$.
        \item $x^{p^v} = 1$. % SO WHAT?
        \item $y^v = 1$.    % WHY?
    \end{enumerate}
    One can also show $x$ and $y$ to be the only elements of $G$ satisfying these conditions. %% SHOW!!!! 
\end{proof}

\begin{boxdefinition}[The $p$- $p'$-parts of $g$]
    For any $g \in G$, call $x$ and $y$ from Lemma \ref{Ch2:Lem:p_and_p'_parts} the $p$- and $p'$-parts of $g$ respectively.
\end{boxdefinition}

For the remainder of this subsection, let $n$ = $\abs{G}$, and let $\zeta = e^{\frac{2\pi i}{n}}$ be a primitive $n$th root of unity. Define $\Z[\zeta]$ to be the subring of $\C$ generated by $\Z$ and $\zeta$. Let $p$ be a prime number and $p\Z[\zeta]$ be the principal ideal of $\Z[\zeta]$ generated by $p$. By the Correspondence Theorem, the ideals of $\Z[\zeta]$ containing $p\Z[\zeta]$ are in bijection with the ideals of $\quotient{\Z[\zeta]}{p\Z[\zeta]}$. Since the latter object is finite, it contains finitely many ideals, meaning that only finitely many of the ideals of $\Z[\zeta]$ contain $p\Z[\zeta]$. We can then look at the maximal (proper) ideal amongst these, which is a maximal ideal of $\Z[\zeta]$. Denote it by $P$. Indeed, we can show that $P \cap \Z = p\Z$.

\begin{theorem}
    Let $g \in G$, and let $y$ be the $p'$-part of $g$ for some $p$ prime. For all characters $\chi$ of $G$, $\chi(g) - \chi(y)$ lies in the maximal ideal $P$.
\end{theorem}
\begin{proof}
    Let $m = \ord{g} = u p^v$. Let $a,b \in \Z$ be such that $au + b p^v = 1$ (as in the proof of Lemma \ref{Ch2:Lem:p_and_p'_parts}). We have that $y = g^{bp^v}$. We know that $\chi(g), \chi(y) \in \Z[\zeta]$.

    Now, let $w$ be an $m$th root of unity. Since $m \vert n$ (by Lagrange's Theorem), we know that $w \in \Z[\zeta]$. We have that $w = w^{au + bp^v}$. Raising both sides to the $p^v$th power, we get that $w^{p^v} = w^{aup^v} w^{bp^{2v}}$. Since $w^{up^v} = w^m = 1$, we have that $w^{p^v} = w^{bp^{2v}}$.

    Now, consider the binomial expansion of $\parenth{w - w^{bp^v}}^{p^v}$. Some blah blah, use fact that $P$ is a prime ideal (since it is maximal). % FINISH
\end{proof}

This theorem has several important consequences.

\begin{corollary}
    Let $g \in G$ and $y$ be the $p'$-part of $g$ for some prime $p$. Let $\chi$ be a character of $G$.
    \begin{enumerate}[label = \normalfont \arabic*., noitemsep]
        \item If $\chi(g), \chi(y) \in \Z$, then $\chi(g) \equiv \chi(y) \pmod{p}$.
        \item If $\ord{g} = p^k$ for some $k \in \N$, then $\chi(g) \equiv \chi(1) \pmod{p}$.
    \end{enumerate}
\end{corollary}

% Move the following to the end of the bit on orthogonality
\begin{lemma}
    Let $g, h \in G$. Then, $g$ is conjugate to $h$ iff for all characters $\chi$ of $G$, $\chi(g) = \chi(h)$.
\end{lemma}
\begin{proof}
    \hfill
    \begin{description}
        \item[$\parenth{\implies}$] Seen Before
        \item[$\parenth{\impliedby}$] As we know, $\Irr{G}$ is an orthonormal basis for $\FcGC$. Let $f$ be the indicator function on $g^G$. Then, $f(g) = 1 \implies f(h) = 1 \implies h \in g^G$.
    \end{description}
\end{proof}
\begin{corollary}
    Let $g \in G$. Then, $g$ is conjugate to $g^{-1}$ iff for all characters $\chi$ of $G$, $\chi(g) \in \R$.
\end{corollary}

\begin{boxexample}[Constructing the Character Table of $G = \PSL{2, 7}$]
    First, note that $\abs{G} = 168$ and that $G$ has $6$ conjugacy classes. Denote them by $C_1, \ldots, C_6$ with representatives $g_1, \ldots, g_6$, with $g_1 = 1$. It turns out we have exactly $6$ irreducible characters of $G$. \\ % HOW DO WE KNOW THIS?

    We begin filling the table as follows.
    \begin{table}[H]
        \centering
        \begin{tabular}{|c|cccccc|}
            $C_G(g_i)$ & $168$ & $8$ & $4$ & $3$ & $7$ & $7$ \\
            $\ord{g_i}$ & $1$ & $2$ & $4$ & $3$ & $7$ & $7$ \\
            $g_i$ & $g_1$ & $g_2$ & $g_3$ & $g_4$ & $g_5$ & $g_6$ \\
            \hline
            $\chi_1$ & $1$ & $1$ & $1$ & $1$ & $1$ & $1$ \\
            $\chi_2$ & & & & & & \\
            $\chi_3$ & & & & & & \\
            $\chi_4$ & & & & & & \\
            $\chi_5$ & & & & & & \\
            $\chi_6$ & & & & & & \\
        \end{tabular}
    \end{table}
    We can then fill in the first column by applying Corollary \ref{Ch2:Cor:Reg_Rep_Col_1_Grp_Size}.
    \begin{table}[H]
        \centering
        \begin{tabular}{|c|cccccc|}
            $C_G(g_i)$ & $168$ & $8$ & $4$ & $3$ & $7$ & $7$ \\
            $\ord{g_i}$ & $1$ & $2$ & $4$ & $3$ & $7$ & $7$ \\
            $g_i$ & $g_1$ & $g_2$ & $g_3$ & $g_4$ & $g_5$ & $g_6$ \\
            \hline
            $\chi_1$ & $1$ & $1$ & $1$ & $1$ & $1$ & $1$ \\
            $\chi_2$ & $7$ & & & & & \\
            $\chi_3$ & $8$ & & & & & \\
            $\chi_4$ & $3$ & & & & & \\
            $\chi_5$ & $3$ & & & & & \\
            $\chi_6$ & $6$ & & & & & \\
        \end{tabular}
    \end{table}
    Observe that since the orders are unique for the first four columns, they contain \underline{all} elements of their respective orders. We can then fill in the third column by applying the Column Orthogonality Relation and the first Lemma from today. % ADD REFERENCE! %% MOVE FILLED OUT COLUMN 1 TO THE RIGHT! IT SHOULD BE THE THIRD ONE BUT WE ACCIDENTALLY FILLED IT IN AS THE SECOND.
    \begin{table}[H]
        \centering
        \begin{tabular}{|c|cccccc|}
            $C_G(g_i)$ & $168$ & $8$ & $4$ & $3$ & $7$ & $7$ \\
            $\ord{g_i}$ & $1$ & $2$ & $4$ & $3$ & $7$ & $7$ \\
            $g_i$ & $g_1$ & $g_2$ & $g_3$ & $g_4$ & $g_5$ & $g_6$ \\
            \hline
            $\chi_1$ & $1$ & $1$ & $1$ & $1$ & $1$ & $1$ \\
            $\chi_2$ & $7$ & & $-1$ & & & \\
            $\chi_3$ & $8$ & & $0$ & & & \\
            $\chi_4$ & $3$ & & $1$ & & & \\
            $\chi_5$ & $3$ & & $1$ & & & \\
            $\chi_6$ & $6$ & & $0$ & & & \\
        \end{tabular}
    \end{table}
    We can then fill in the fourth column using the second point of Corollary 2.3.11. % ADD REFERENCE!
    \begin{table}[H]
        \centering
        \begin{tabular}{|c|cccccc|}
            $C_G(g_i)$ & $168$ & $8$ & $4$ & $3$ & $7$ & $7$ \\
            $\ord{g_i}$ & $1$ & $2$ & $4$ & $3$ & $7$ & $7$ \\
            $g_i$ & $g_1$ & $g_2$ & $g_3$ & $g_4$ & $g_5$ & $g_6$ \\
            \hline
            $\chi_1$ & $1$ & $1$ & $1$ & $1$ & $1$ & $1$ \\
            $\chi_2$ & $7$ & & $-1$ & $1$ & & \\
            $\chi_3$ & $8$ & & $0$ & $-1$ & & \\
            $\chi_4$ & $3$ & & $1$ & $0$ & & \\
            $\chi_5$ & $3$ & & $1$ & $0$ & & \\
            $\chi_6$ & $6$ & & $0$ & $0$ & & \\
        \end{tabular}
    \end{table}
    We now have four possibilities for the second column. % CHECK PHONE. SECOND COL WAS \pm EVERYWHERE, AND 2 AND 0 COULD HAVE BOTH BEEN SWAPPED.
    \begin{table}[H]
        \centering
        \begin{subtable}{0.45\linewidth}
            \begin{tabular}{|c|cccccc|}
                $C_G(g_i)$ & $168$ & $8$ & $4$ & $3$ & $7$ & $7$ \\
                $\ord{g_i}$ & $1$ & $2$ & $4$ & $3$ & $7$ & $7$ \\
                $g_i$ & $g_1$ & $g_2$ & $g_3$ & $g_4$ & $g_5$ & $g_6$ \\
                \hline
                $\chi_1$ & $1$ & $1$ & $1$ & $1$ & $1$ & $1$ \\
                $\chi_2$ & $7$ & & $-1$ & $1$ & & \\
                $\chi_3$ & $8$ & & $0$ & $-1$ & & \\
                $\chi_4$ & $3$ & & $1$ & $0$ & & \\
                $\chi_5$ & $3$ & & $1$ & $0$ & & \\
                $\chi_6$ & $6$ & & $0$ & $0$ & & \\
            \end{tabular}
        \end{subtable}
        \hspace{0.05\linewidth}
        \begin{subtable}{0.45\linewidth}
            \begin{tabular}{|c|cccccc|}
                $C_G(g_i)$ & $168$ & $8$ & $4$ & $3$ & $7$ & $7$ \\
                $\ord{g_i}$ & $1$ & $2$ & $4$ & $3$ & $7$ & $7$ \\
                $g_i$ & $g_1$ & $g_2$ & $g_3$ & $g_4$ & $g_5$ & $g_6$ \\
                \hline
                $\chi_1$ & $1$ & $1$ & $1$ & $1$ & $1$ & $1$ \\
                $\chi_2$ & $7$ & & $-1$ & $1$ & & \\
                $\chi_3$ & $8$ & & $0$ & $-1$ & & \\
                $\chi_4$ & $3$ & & $1$ & $0$ & & \\
                $\chi_5$ & $3$ & & $1$ & $0$ & & \\
                $\chi_6$ & $6$ & & $0$ & $0$ & & \\
            \end{tabular}
        \end{subtable}
    \end{table}
    For the last two columns of the table, we use row orthogonality, in combination with the fact that $g_5 = g_6\inv$.
    \begin{table}[H]
        \centering
        \begin{tabular}{|c|cccccc|}
            $C_G(g_i)$ & $168$ & $8$ & $4$ & $3$ & $7$ & $7$ \\
            $\ord{g_i}$ & $1$ & $2$ & $4$ & $3$ & $7$ & $7$ \\
            $g_i$ & $g_1$ & $g_2$ & $g_3$ & $g_4$ & $g_5$ & $g_6$ \\
            \hline
            $\chi_1$ & $1$ & $1$ & $1$ & $1$ & $1$ & $1$ \\
            $\chi_2$ & $7$ & $-1$ & $1$ & & $0$ & $0$ \\
            $\chi_3$ & $8$ & $0$ & $-1$ & & $1$ & $1$ \\
            $\chi_4$ & $3$ & $1$ & $0$ & & $z$ & $\bar{z}$ \\
            $\chi_5$ & $3$ & $1$ & $0$ & & $\bar{z}$ & $z$ \\
            $\chi_6$ & $6$ & $0$ & $0$ & & $t$ & $\bar{t}$ \\
        \end{tabular}
    \end{table}
    % CHECK PHONE FOR REST!

    FINAL COMMENTS. Also use the results on $p$ groups, since $G$ is a $7$-group.
\end{boxexample}