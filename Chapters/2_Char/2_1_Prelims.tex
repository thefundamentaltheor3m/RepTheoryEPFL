\section{Preliminaries}

\subsection{Important Definitions and Properties}

\begin{boxdefinition}[Character]
    Let $V$ be a $\C G$-module. The character $\chi_v$ is the function $\chi_V : G \to \C$ given by $\chi_v(g) = \Tr{\rho(g)}$, where $\rho$ is the representation associated to $V$.
\end{boxdefinition}

\begin{remark}
    Since the trace is independent of our choice of basis, the definition makes sense.
\end{remark}

\begin{definition}[Irreducibility]
    We say a character $\chi_V$ is irreducible if the associated representation $\Vp$ is irreducible over $\C$.
\end{definition}
\begin{definition}[Degree]
    We define the degree of a character to be that of its associated representation.
\end{definition}
\begin{definition}[Trivial Character]
    We define the trivial character to be that associated with the trivial representation.
\end{definition}

\begin{proposition}[Behaviour of Characters]
    Let $V$, $W$ be $\C G$-modules, and let $g \in G$ be arbitrary. Then,
    \begin{enumerate}[label = \normalfont \arabic*., noitemsep]
        \item $\chi_{V \+ W} = \chi_V + \chi_W$
        \item $V \cong W \implies \chi_V = \chi_W$
        \item $\pdim{V} = \chi_V(1)$
        \item $\chi_v(g)$ is a sum of $d$th roots of unity, where $d = \ord{g}$.
        \item $\abs{\chi_V(g)} \leq \pdim{V}$, with equality iff $g \in \pker{\rho}$, where $\rho$ is the associated representation.
        \item $\chi_V\!\parenth{g\inv} = \overline{\chi_V(g)}$
    \end{enumerate}
\end{proposition}
\begin{proof}
    We do not give complete proofs here, just sketches.
    \begin{enumerate}
        \item This follows from the fact that the trace of a direct sum is the sum of the traces.
        \item This follows from the invariance of the trace under change of basis.
    \end{enumerate}
    \verb|sorry|
\end{proof}

\subsection{Character Tables}

A character table is exactly what it sounds like: a table consisting of the elements of a group, their images in a representation, and their associated characters. In this subsection, we investigate character tables by going through specific examples.

\begin{example}[The Dihedral Group of Order $8$]
    Let $G = D_8$, the dihedral group of order $8$. Consider the presentation
    \begin{align*}
        G = \cycl{a, b \mid a^4 = b^2 = 1, b\inv a b = a\inv}
    \end{align*}
    Let $\rho : G \to \GL{2, \C}$ be a representation of $G$ over $\C$ given by
    \begin{align*}
        \rho(a) = \begin{bmatrix}
            0 & 1 \\ -1 & 0
        \end{bmatrix}
        \quad\quad \text{ and } \quad\quad
        \rho(b) = \begin{bmatrix}
            1 & 0 \\ 0 & -1
        \end{bmatrix}
    \end{align*}
    We then have the following character table for $\rho$:
    \begin{table}[H]
        \centering
        \begin{tabular}{|c|cccccccc|}
            \hline
            $g$ & $1$ & $a$ & $a^2$ & $a^3$ & $b$ & $ab$ & $a^2 b$ & $a^3 b$ \\
            $\rho(g)$ &
            $\begin{bmatrix} 1 & 0 \\ 0 & 1 \end{bmatrix}$
            &
            $\begin{bmatrix} 0 & 1 \\ -1 & 0 \end{bmatrix}$
            &
            $\begin{bmatrix} -1 & 0 \\ 0 & -1 \end{bmatrix}$
            &
            $\begin{bmatrix} 0 & -1 \\ 1 & 0 \end{bmatrix}$
            &
            $\begin{bmatrix} 1 & 0 \\ 0 & -1 \end{bmatrix}$
            &
            $\begin{bmatrix} 0 & -1 \\ -1 & 0 \end{bmatrix}$
            &
            $\begin{bmatrix} -1 & 0 \\ 0 & 1 \end{bmatrix}$
            &
            $\begin{bmatrix} 0 & 1 \\ 1 & 0 \end{bmatrix}$
            \\
            $\chi_V(g)$ & $2$ & $0$ & $-2$ & $0$ & $0$ & $0$ & $0$ & $0$ \\
            \hline
        \end{tabular}
    \end{table}
\end{example}

\begin{example}[The Cyclic Group of Order $3$]
    Let $G = C_3 = \cycl{a}$ be the cyclic group of order $3$. Let $\rho_1, \rho_2, \rho_3$ be irreducible representations of $G$, with corresponding (irreducible) characters $\chi_1, \chi_2, \chi_3$.
\end{example}
