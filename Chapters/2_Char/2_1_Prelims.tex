\section{Preliminaries}

\subsection{Important Definitions and Properties}

\begin{boxdefinition}[Character]
    Let $V$ be a $\C G$-module. The character $\chi_v$ is the function $\chi_V : G \to \C$ given by $\chi_v(g) = \Tr{\rho(g)}$, where $\rho$ is the representation associated to $V$.
\end{boxdefinition}

\begin{remark}
    Since the trace is independent of our choice of basis, the definition makes sense.
\end{remark}

\begin{definition}[Irreducibility]
    We say a character $\chi_V$ is irreducible if the associated representation $\Vp$ is irreducible over $\C$.
\end{definition}
\begin{definition}[Degree]
    We define the degree of a character to be that of its associated representation.
\end{definition}
\begin{definition}[Trivial Character]
    We define the trivial character to be that associated with the trivial representation.
\end{definition}

\begin{proposition}[Behaviour of Characters]
    Let $V$, $W$ be $\C G$-modules, and let $g \in G$ be arbitrary. Then,
    \begin{enumerate}[label = \normalfont \arabic*., noitemsep]
        \item $\chi_{V \+ W} = \chi_V + \chi_W$
        \item $V \cong W \implies \chi_V = \chi_W$
        \item $\pdim{V} = \chi_V(1)$
        \item $\chi_v(g)$ is a sum of $d$th roots of unity, where $d = \ord{g}$.
        \item $\abs{\chi_V(g)} \leq \pdim{V}$
        \item $\chi_V\!\parenth{g\inv} = \overline{\xi_V(g)}$
    \end{enumerate}
\end{proposition}
\begin{proof}
    We do not give complete proofs here, just sketches.
    \begin{enumerate}
        \item This follows from the fact that the trace of a direct sum is the sum of the traces.
        \item This follows from the invariance of the trace under change of basis.
    \end{enumerate}
    \verb|sorry|
\end{proof}

\subsection{Character Tables}

\begin{boxexample}[Elementary Examples of Character Tables] \hfill
    \begin{enumerate}
        \item Let $G = D_8$, the dihedral group of order $8$. Consider the presentation
        \begin{align*}
            G = \cycl{a, b \mid a^4 = b^2 = 1, b\inv a b = a\inv}
        \end{align*}
        Let $\rho : G \to \GL{2, \C}$ be a representation of $G$ over $\C$ given by
        \begin{align*}
            \rho(a) = \begin{bmatrix}
                0 & 1 \\ -1 & 0
            \end{bmatrix}
            \quad\quad \text{ and } \quad\quad
            \rho(b) = \begin{bmatrix}
                1 & 0 \\ 0 & -1
            \end{bmatrix}
        \end{align*}
        The following table, known as a \textit{character table}, lists the characters under $\rho$ of each element of $g$.
        \begin{table}[H]
            \centering
            \begin{tabular}{|c||c|c|c|c|c|c|c|c|}
                \hline
                $g$ & $1$ & $a$ & $a^2$ & $a^3$ & $b$ & $ab$ & $a^2 b$ & $a^3 b$ \\
                $\rho(g)$ &
                $\begin{bmatrix} 1 & 0 \\ 0 & 1 \end{bmatrix}$
                &
                $\begin{bmatrix} 0 & 1 \\ -1 & 0 \end{bmatrix}$
                &
                $\begin{bmatrix} -1 & 0 \\ 0 & -1 \end{bmatrix}$
                &
                $\begin{bmatrix} 0 & -1 \\ 1 & 0 \end{bmatrix}$
                &
                $\begin{bmatrix} 1 & 0 \\ 0 & -1 \end{bmatrix}$
                &
                $\begin{bmatrix} 0 & -1 \\ -1 & 0 \end{bmatrix}$
                &
                $\begin{bmatrix} -1 & 0 \\ 0 & 1 \end{bmatrix}$
                &
                $\begin{bmatrix} 0 & 1 \\ 1 & 0 \end{bmatrix}$
                \\
                $\chi_V(g)$ & $2$ & $0$ & $-2$ & $0$ & $0$ & $0$ & $0$ & $0$ \\
                \hline
            \end{tabular}
        \end{table}

        \item Let $G = C_3 = \cycl{a}$ be the cyclic group of order $3$. Let $\rho_1, \rho_2, \rho_3$ be irreducible representations of $G$, with corresponding (irreducible) characters $\chi_1, \chi_2, \chi_3$.
    \end{enumerate}
\end{boxexample}

\subsection{Central Functions}

\begin{boxdefinition}[Central Function]
    We say $f : G \to \C$ is central if $\fx = \fof{g\inv x g}$ for all $x,g \in G$.
\end{boxdefinition}

\begin{boxexample}[Examples of Central Functions]
    \hfill
    \begin{enumerate}
        \item The character of a $\C G$-module
        \item The order function $x \mapsto \ord{x} : G \to \N \subset \C$
    \end{enumerate}
\end{boxexample}

\begin{boxnotation}
    We define
    \begin{enumerate}
        \item $\Fof{G, \C} := \set{f : G \to \C}$
        \item $\Fcof{G, \C} := \set{f \in \Fof{G, \C} : f \text{ is central}}$
        \item $\delta_g(x)$ to be the indicator function (for $g, x \in G$).
    \end{enumerate}
\end{boxnotation}

\begin{remark}
    It turns out that $\Fof{G, \C}$ has the natural structure of being a $\C$-vector space of dimension $\abs{G}$, with basis $\set{\delta_g : g \in G}$.
\end{remark}

\begin{definition}
    Define $\cycl{\cdot, \cdot} : \FGC \times \FGC \to \C$ By
    \begin{align}
        \cycl{f_1, f_2} = \frac{1}{\abs{G}} \sum_{g \in G} f_1(g) \overline{f_2(g)}
    \end{align}
    One can show this to be an inner-product on $\FGC$.
\end{definition}

\subsection{Orthogonality of Characters}

\begin{lemma}
    Let $\rho : G \to \GL{n, \C}$ and $\rho' : G \to \GL{m, \C}$ be irreducible representations of $G$ over $\C$. Fix $j, s \in \set{1, \ldots, m}$ and $r, i \in \set{1, \ldots, n}$.
    \begin{enumerate}[label = \normalfont \arabic*., noitemsep]
        \item If $\rho$ and $\rho'$ are \textit{not} equivalent, then
        \begin{align*}
            \frac{1}{\abs{G}} \sum_{g \in G} {\rho(g)}_{ri} {\rho'\!\parenth{g\inv}}_{js} = 0
        \end{align*}

        \item If $\rho$ and $\rho'$ \textit{are} equivalent, then
        \begin{align*}
            \frac{1}{\abs{G}} \sum_{g \in G} {\rho(g)}_{ri} {\rho'\!\parenth{g\inv}}_{js} =
            \begin{cases}
                \frac{1}{n} & \text{if } i = j \text{ and } r = s \\
                0 & \text{otherwise}
            \end{cases}
        \end{align*}
    \end{enumerate}
    where we use the notation $T_{ij}$ to refer to the $ij$th entry of the matrix of $T$.
\end{lemma}
\begin{proof}
    Let $V = \C^n$ and $W = \C^m$ be the two simple $\C G$-modules corresponding to $\rho$ and $\rho'$ respectively. The idea is to define a linear map that will allow us to use Schur's Lemma.

    For some chosen bases of $V$ and $W$, let $\phi_{ij} : W \to V$ be the $\C$-linear map given by the $n \times m$ matrix with $ij$th entry $1$ and all other entries $0$. Define
    \begin{align*}  % TODO: Make hat look nice
        \hat{\phi_{ij}} := \frac{1}{\abs{G}} \sum_{g \in G} \rho(g) \circ \phi_{ij} \circ \rho'\!\parenth{g\inv}
    \end{align*}
    By Theorem \ref{SP:Thm:Schur_fin_G_over_C}, $\hat{\phi_{ij}}$ is a $\C G$-module homomorphism from $W$ to $V$.
    \begin{enumerate}
        \item If $W \not\cong V$, we have $\hat{\phi_{ij}} = 0$. In particular,
        \begin{align*}
            0 &= \parenth{
                \frac{1}{\abs{G}} \sum_{g \in G} \rho(g) \circ E_{ij} \circ \rho'\!\parenth{g\inv}
            }_{rs} \\
            &= \frac{1}{\abs{G}} \sum_{g \in G} \brac{\rho(g) \circ E_{ij} \circ \rho'\!\parenth{g\inv}}_{rs} \\
            &= \sum_{k=1}^{n} \sum_{l=1}^{m} \frac{1}{\abs{G}} \sum_{g \in G} \brac{\rho(g)}_{rk} \brac{E_{ij}}_{kl} \brac{\rho'\!\parenth{g\inv}}_{ls} \\
            &= \frac{1}{\abs{G}} \sum_{g \in G} \brac{\rho(g)}_{ri} \brac{\rho'\!\parenth{g\inv}}_{js}
        \end{align*}

        \item Similarly, if $W \cong V$, we can view $\phi_{ij}$ as being given by the $n \times n$ matrix $E_{ij}$. Now, by Theorem \ref{SP:Thm:Schur_fin_G_over_C}, we know that
        \begin{align*}
            \hat{\phi_{ij}} &= \frac{1}{n} \Tr{E_{ij}} \cdot \id_V \\
            &= \frac{1}{n} \delta_{ij} \cdot \id_V
        \end{align*}
        We can then show the desired result using a similar computation.
    \end{enumerate}
\end{proof}
\begin{remark}
    As per Dr. Rizzoli, on the exam, it's more important to know the idea of such a proof than the specifics of \textit{which index goes where}.
\end{remark}

\begin{boxtheorem}[Orthogonality Theorem] \label{Ch2:Thm:Orth_Char}
    Let $S, T$ be irreducible $\C G$-modules.
    \begin{enumerate}[label = \normalfont \arabic*., noitemsep]
        \item If $S \not\cong T$, then $\cycl{\chi_S, \chi_T} = 0$.
        \item If $S \cong T$, then $\cycl{\chi_S, \chi_T} = 1$
    \end{enumerate}
    In other words, irreducible characters form an orthogonal system.
\end{boxtheorem}
\begin{proof}
    Let $P : G \to \GL{n, \C}$ and $Q : G \to \GL{m, \C}$ be the representations corresponding to $S$ and $T$. We know that
    \begin{align*}
        \cycl{\chi_S, \chi_T} &=
        \frac{1}{\abs{G}} \sum_{G \in G} \chi_S(g) \chi_T\!\parenth{g\inv} \\
        &= \frac{1}{\abs{G}} \sum_{g \in G} \Tr{P(g)} \Tr(Q\!\parenth{g\inv}) \\
        &= \frac{1}{\abs{G}} \sum_{g \in G} \parenth{\sum_{i = 1}^{n} \brac{P(g)}_{ii}}\parenth{\sum_{j=1}^{n} \brac{Q\!\parenth{g\inv}}_{jj}} \\
        &= \sum_{i=1}^{n} \sum_{j=1}^{m} \frac{1}{\abs{G}} \sum_{g \in G} \brac{P(g)}_{ii} \brac{Q\!\parenth{g\inv}_{jj}}
    \end{align*}
    We can then use the previous lemma to evaluate these sums.
\end{proof}

We have the following important corollary.

\begin{corollary}
    Up to isomorphism, there are finitely many irreducible $\C G$-modules.
\end{corollary}