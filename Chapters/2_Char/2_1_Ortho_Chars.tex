\section{The Theory of Irreducible Characters}

In this section, we give an overview of the theory of irreducilbe characters. Our approach focuses extensively on the notion of orthogonality with respect to an inner product we shall soon define. An interesting result we will go on to prove is that irreducilbe characters play an important role in a broader class of functions on groups, telling us that representation theory can be used to study not only groups themselves but also functions thereof.

\subsection{On Central Functions}

As we all know, a group can contain several elements that all behave similarly---take, for instance, similar matrices in any general linear group. This is why we have the notion of \textit{conjugacy classes}, which allow us to study the elements of a group in terms of their actions or behaviours.

The purpose of character theory is to understand the structural and behavioural properties of a group, without focusing on syntactic particularities. This is why we define the notion of a \textit{class function}.

\begin{definition}[Class Function]
    Let $X$ be any set. A function $f : G \to X$ is said to be a class function if $\fx = \fof{g\inv x g}$ for all $x,g \in G$---in other words, if it is constant on all conjugacy classes of $G$.
\end{definition}

In this section, we will primarily be focusing on complex representations. Therefore, the following definition is useful.

\begin{boxdefinition}[Central Function]
    We say $f : G \to \C$ is central if it is a class function. In other words, central functions are precisely $\C$-valued class functions.
\end{boxdefinition}

We are already familiar with several examples of central functions.

\begin{boxexample}[Examples of Central Functions] \label{Ch2:Eg:Central_Funcs}
    The following are all central:
    \begin{enumerate}
        \item The order function $h \mapsto \ord{h} : H \to \N \subset \C$ for any group $H$
        \item The determinant function $A \mapsto \pdet{A} : \GL{n, \C} \to \C$ for any $n \in \N$
        \item The trace function $A \mapsto \Tr{A} : \GL{n, \C} \to \C$ for any $n \in \N$
    \end{enumerate}
\end{boxexample}

The last two examples are, in particular, compatible with representations of $G$ over $\C$. This will prove to be important when we define the character of a representation. Before doing so, however, we will need to outline the structure of the inner-product space of central functions on $G$. First, some notation.

\begin{boxnotation}
    We define
    \begin{enumerate}
        \item $\FGC := \set{f : G \to \C}$
        \item $\FcGC := \set{f \in \FGC : f \text{ is central}}$
        \item $\delta_g(x)$ to be the indicator function (for $g, x \in G$).
    \end{enumerate}
\end{boxnotation}

We have the following natural result.

\begin{proposition}[$\C$-Vector Space Structure of $\FGC$ and $\FcGC$]\label{Ch2:Prop:FGC_FcGC_Fin_Dim} \hfill
    \begin{enumerate}[label = \normalfont \arabic*., noitemsep]
        \item $\FGC$ is a $\C$-vector space of dimension $\abs{G}$, with basis $\set{\delta_g : g \in G}$.
        \item $\FcGC$ is a subspace of $\FGC$ of dimension equal to the number of conjugacy classes of $G$, with basis $\set{\delta_g : g \text{ uniquely represents a conjugacy class of $G$}}$.
    \end{enumerate}
\end{proposition}
\begin{proof}
    The model we choose for this proof is that of $k^n$ being isomorphic to the set of functions from $\set{1, \ldots, n}$ to $k$ (for any field $k$). Under this model, the indicator functions on $\set{1, \ldots, n}$ correspond to the standard basis of $k^n$. % We do not argue too rigorously here, as the proofs of both points are rather straightforward.
    \begin{enumerate}
        \item Since $G$ is finite, this is immediate from the model described above.
        \item That $\FcGC$ is a subspace follows from the fact that the sum of two central functions is central, as is any scalar multiple of a central function. Furthermore, since central functions are uniquely determined by their values on each conjugacy class, $\pdim{\FcGC}$ is the number of conjugacy classes of $G$.
        
        I also offer a more formal argument here for the dimensionality, as I found it instructive to think about it this way. Let $C_1, \ldots, C_r$ be the conjugacy classes of $G$. Write $C_i = \set{g_{i_1}, g_{i_2}, \ldots, g_{i_{l_i}}}$. Let $w_i := \sum_{j=1}^{l_i} \delta_{g_{i_j}}$. It is easy to see that the $w_i$ are linearly independent: any linear combination of all the $w_i$ is, in particular, a linear combination of $\delta_{g}$ as $g$ ranges over $G$, with each $\delta_g$ appearing exactly once. Furthermore, the $w_i$ span $\FcGC$: their span is clearly contained in $\FcGC$, and every element of $\FcGC$ must be constant on any given conjugacy class, meaning that the coordinates of conjugate components must be the same.
    \end{enumerate}
\end{proof}

It turns out that there is also a natural inner-product on $\FGC$.

\begin{boxdefinition}[Standard Inner-Product on $\FGC$]
    Define the function $\cycl{\cdot, \cdot} : \FGC \times \FGC \to \C$ by
    \begin{align}
        \cycl{f_1, f_2} = \frac{1}{\abs{G}} \sum_{g \in G} f_1(g) \overline{f_2(g)}
        \label{SP:eq:Char_Inner_Product}
    \end{align}
    It is easy enough to show that the $\cycl{\cdot, \cdot}$ is, indeed, an inner-product on $\FGC$. We will also use it as an inner-product on $\FcGC$.
\end{boxdefinition}

An obvious orthogonal basis for $\FcGC$ is that of the indicator functions of representatives of distinct conjugacy classes. However, in this paradigm, the set $S$ of conjugacy classes of $G$ merely acts as an \textit{index} set for the standard basis of $\C^{\abs{S}}$. Ie, we can infer nothing about $S$ beyond its size, a quantity that does not tell us about the actual \textit{group} structure of $G$ (for instance, for any prime $p$, the decidedly non-isomorphic abelian groups $\quotient{\Z}{p^2\Z}$ and $\quotient{\Z}{p\Z} \+ \quotient{\Z}{p\Z}$ both have exactly $p^2$ conjugacy classes). It turns out that we can find a much better orthogonal (indeed, ortho\textit{normal}) basis for $\FcGC$ in the world of \textit{characters}, a class of central functions closely related to the group structure of $G$.

\subsection{Introduction to Character Theory}

In this subsection, we introduce a central function that is compatible with the notion of a representation, namely, the \textit{character}. As we saw in Example \ref{Ch2:Eg:Central_Funcs}, both the trace and the determinant would be good options, but the convention is to define the character in terms of the trace. A heuristic reason is that this way, given characters associated to two representations, their sum will give the character of the direct sum, and their product that of the tensor product of the underlying representations. There are deeper reasons too, which will become clearer as we progress.

\begin{boxdefinition}[Character]
    Let $V$ be a $\C G$-module. The character of $V$ is the map $\chi_V : G \to \C$ given by $\chi_V(g) = \Tr{\rho(g)}$, where $\rho$ is the representation associated to $V$.
\end{boxdefinition}

Immediately, we are able to ``import'' the following definitions from Chapter~\ref{Ch1:CH}.

\begin{definition}[Irreducibility]
    We say a character $\chi_V$ is irreducible if the associated representation $\Vp$ is irreducible over $\C$.
\end{definition}

\begin{definition}[Degree]
    We define the degree of a character to be that of its associated representation.
\end{definition}

\begin{definition}[Trivial Character]
    We define the trivial character to be that associated with the trivial representation.
\end{definition}

Characters have several important properties, which we list below. We will make extensive use of these properties for the remainder of this chapter.

\begin{boxproposition}[On the Behaviour of Characters]\label{Ch2:Prop:Bhv_Char}
    Let $V$ be a \CGM, with associated representation $\rho$.
    \begin{enumerate}[label = \normalfont \arabic*., noitemsep]
        \item For all \CGM s $W$, $\chi_{V \+ W} = \chi_V + \chi_W$
        \item  For all \CGM s $W$, $V \cong W \implies \chi_V = \chi_W$
        \item $\pdim{V} = \chi_V(1)$
        \item For all $g \in G$, $\chi_V(g)$ is a sum of $d$th roots of unity, where $d = \ord{g}$.
        \item For all $g \in G$, $\abs{\chi_V(g)} \leq \pdim{V}$, with equality iff $g \in \pker{\rho}$.
        \item For all $g \in G$, $\chi_V\!\parenth{g\inv} = \overline{\chi_V(g)}$
    \end{enumerate}
\end{boxproposition}
\begin{proof}
    We just give sketches here, not complete proofs.
    \begin{enumerate}[noitemsep]
        \item This follows from the fact that the trace of a direct sum is the sum of the traces.
        \item This follows from the invariance of the trace under change of basis.
        \item This follows from the fact that the trace of the identity is the dimension of the space.
        \item For any $g \in G$ of order $d$, $\rho(g)$ is of order $d$. Hence, its minimal polynomial divides $X^d - 1 \in \C[X]$, which has distinct roots in $\C$. This means that the eigenvalues of $\rho(g)$ are $d$th roots of unity. Putting $\rho(g)$ in Jordan Normal Form, we see that its trace is a sum of $d$th roots of unity.\footnote{It is not necessarily a sum of \textit{all} $d$th roots of unity, or even \textit{distinct} $d$th roots of unity.}
        \item \verb|sorry|
        \item \verb|sorry|
    \end{enumerate}
\end{proof}

We now give a few examples of characters of representations.

\begin{boxexample}[The Dihedral Group of Order $8$]
    Let $G = D_8$, the dihedral group of order $8$. Consider the presentation
    \begin{align*}
        G = \cycl{a, b \mid a^4 = b^2 = 1, b\inv a b = a\inv}
    \end{align*}
    Let $\rho : G \to \GL{2, \C}$ be a representation of $G$ over $\C$ given by
    \begin{align*}
        \rho(a) = \begin{bmatrix}
            0 & 1 \\ -1 & 0
        \end{bmatrix}
        \quad\quad \text{ and } \quad\quad
        \rho(b) = \begin{bmatrix}
            1 & 0 \\ 0 & -1
        \end{bmatrix}
    \end{align*}
    The associated character $\chi_V$ then takes on the following values:
    \begin{table}[H]
        \centering
        \hspace{-0.3cm}
        \begin{tabular}{C{1.1cm}|C{1.1cm} C{1.38cm} C{1.67cm} C{1.38cm} C{1.38cm} C{1.67cm} C{1.38cm} C{1.1cm}}
            $g$ & $1$ & $a$ & $a^2$ & $a^3$ & $b$ & $ab$ & $a^2 b$ & $a^3 b$ \\
            \hline \\[-11pt]
            $\rho(g)$ & \footnotesize 
            $\begin{bmatrix} 1 & 0 \\ 0 & 1 \end{bmatrix}$
            & \footnotesize
            $\begin{bmatrix} 0 & 1 \\ -1 & 0 \end{bmatrix}$
            & \footnotesize
            $\begin{bmatrix} -1 & 0 \\ 0 & -1 \end{bmatrix}$
            & \footnotesize
            $\begin{bmatrix} 0 & -1 \\ 1 & 0 \end{bmatrix}$
            & \footnotesize
            $\begin{bmatrix} 1 & 0 \\ 0 & -1 \end{bmatrix}$
            & \footnotesize
            $\begin{bmatrix} 0 & -1 \\ -1 & 0 \end{bmatrix}$
            & \footnotesize
            $\begin{bmatrix} -1 & 0 \\ 0 & 1 \end{bmatrix}$
            & \footnotesize
            $\begin{bmatrix} 0 & 1 \\ 1 & 0 \end{bmatrix}$
            \\
            $\chi_V(g)$ & $2$ & $0$ & $-2$ & $0$ & $0$ & $0$ & $0$ & $0$
        \end{tabular}
    \end{table}
\end{boxexample}

One can show the conjugacy classes of $D_8$ to be precisely $\set{1}$, $\set{a, a^3}$, $\set{a^2}$, $\set{b, a^2 b}$, and $\set{ab, a^3 b}$. From the table above, it is clear that conjugate elements do, indeed, have the same character value. Therefore, for the sake of brevity, one typically only lists the values of a character corresponding to the representatives of its conjugacy classes.

\begin{boxexample}[The Cyclic Group of Order $3$]\label{Ch2:Eg:Cycl_3_Char}
    Let $G = C_3 = \cycl{a}$ be the cyclic group of order $3$. By Corollary \ref{Ch1:Cor:Irr_Cycl}, we know that $G$ admits exactly three irreducible representations over $\C$, all of degree $1$, each corresponding to choice of $3$rd root of unity to which to send $a$. Denote these by $\rho_1$, $\rho_2$ and $\rho_3$, so that $\rho_j(a) = e^{\frac{2\pi i \parenth{j-1}}{3}}$ for $j = 1, 2, 3$. Denote their corresponding (irreducible) characters $\chi_1$, $\chi_2$ and $\chi_3$. Since the trace map is simply the constant map on a space of dimension $1$, we have the following values for the characters:
    \begin{table}[H]
        \centering
        \begin{tabular}{c|ccc}
            $g$ & $1$ & $a$ & $a^2$ \\
            \hline
            $\chi_1(g)$ & $1$ & $1$ & $1$ \\
            $\chi_2(g)$ & $1$ & $e^{\frac{2\pi i}{3}}$ & $e^{\frac{4\pi i}{3}}$ \\
            $\chi_3(g)$ & $1$ & $e^{\frac{4\pi i}{3}}$ & $e^{\frac{2\pi i}{3}}$
        \end{tabular}
    \end{table}
\end{boxexample}

In the above example, we see that the characters of the irreducible representations of $G$ form an orthonormal system with respect to the standard inner-product \eqref{SP:eq:Char_Inner_Product}---for instance, $\chi_1 \neq \chi_2$, and clearly,
\begin{align*}
    \cycl{\chi_1, \chi_2} &= \frac{1}{3} \sum_{g \in G} \chi_1(g) \overline{\chi_2(g)} \\
    &= \frac{1}{3} \parenth{1 + \bar{e^{\frac{2\pi i}{3}}} + \bar{e^{\frac{4\pi i}{3}}}} \\
    &= \frac{1}{3} \parenth{1 + e^{\frac{4\pi i}{3}} + e^{\frac{2\pi i}{3}}} = 0
\end{align*}
whereas $\cycl{\chi_1, \chi_1} = 1$. In similar fashion, one can check the remaining $7$ cases to show that $\cycl{\chi_i, \chi_j} = \delta_{ij}$ for $i,j = 1,2,3$. It turns out that this is merely one instance of a more general orthogonality phenomenon. We will understand this better in the next subsection.

\subsection{The Orthogonality Theorem}

In this subsection, we dive into the heart of Character Theory, proving one of the most fundamental results in the field: the Orthogonality Theorem. This theorem generalises the orthogonality that we observed in Example \ref{Ch2:Eg:Cycl_3_Char}, and has several important consequences, most notably the fact that the irreducible characters of a group form an orthonormal basis for the space of class functions on that group, a result we will see in the next subsection.

First, we state and prove a useful lemma that helps us understand the \textit{matrices} of irreducible representations (taken, as per convention, with respect to the standard basis of $\C^n$). The point of this is that matrices are more computation-friendly.

\begin{lemma}\label{Ch2:Lem:Pre_Ortho_Thm_Sums}
    Let $\rho : G \to \GL{n, \C}$ and $\rho' : G \to \GL{m, \C}$ be irreducible representations of $G$ over $\C$. Fix $j, s \in \set{1, \ldots, m}$ and $r, i \in \set{1, \ldots, n}$.
    \begin{enumerate}[label = \normalfont \arabic*., noitemsep]
        \item If $\rho$ and $\rho'$ are \textit{not} equivalent, then
        \begin{align*}
            \frac{1}{\abs{G}} \sum_{g \in G} \brac{\rho(g)}_{ri} \brac{\rho'\!\parenth{g\inv}}_{js} = 0
        \end{align*}

        \item If $\rho$ and $\rho'$ \textit{are} equivalent, then
        \begin{align*}
            \frac{1}{\abs{G}} \sum_{g \in G} \brac{\rho(g)}_{ri} \brac{\rho'\!\parenth{g\inv}}_{js} =
            \begin{cases}
                \frac{1}{n} & \text{if } i = j \text{ and } r = s \\
                0 & \text{otherwise}
            \end{cases}
        \end{align*}
    \end{enumerate}
    where we use the notation $\brac{T}_{ij}$ to refer to the $ij$th entry of the matrix of any $T \in \GL{n, \C}$ with respect to the standard basis of $\C^n$.
\end{lemma}
\begin{proof}
    Let $V = \C^n$ and $W = \C^m$ be the two simple $\C G$-modules corresponding to $\rho$ and $\rho'$ respectively. The idea is to define a linear map that will allow us to use Schur's Lemmas---specifically, those pertaining to finite groups over $\C$ (see Theorem \ref{SP:Thm:Schur_fin_G_over_C}).

    Let $\phi_{ij} : W \to V$ be the $\C$-linear map given by the $n \times m$ matrix (with respect to the standard bases of $V$ and $W$) with $ij$th entry $1$ and all other entries $0$. In other words, let $\phi_{ij}$ denote the $ij$th element of the standard basis of the space of linear maps from $V$ to $W$. Define
    \begin{align*}  % TODO: Make hat look nice
        \hat{\phi_{ij}} := \frac{1}{\abs{G}} \sum_{g \in G} \rho(g) \circ \phi_{ij} \circ \rho'\!\parenth{g\inv}
    \end{align*}
    By Theorem \ref{SP:Thm:Schur_fin_G_over_C}, $\hat{\phi_{ij}}$ is a $\C G$-module homomorphism from $W$ to $V$. Indeed, we have that
    \begin{align*}
        \brac{\hat{\phi_{ij}}}_{rs} &= \brac{
            \frac{1}{\abs{G}} \sum_{g \in G} \rho(g) \circ \phi_{ij} \circ \rho'\!\parenth{g\inv}
        }_{rs} \\
        &= \frac{1}{\abs{G}} \sum_{g \in G} \brac{\rho(g) \circ \phi_{ij} \circ \rho'\!\parenth{g\inv}}_{rs} \\
        &= \sum_{k=1}^{n} \sum_{l=1}^{m} \frac{1}{\abs{G}} \sum_{g \in G} \brac{\rho(g)}_{rk} \brac{\phi_{ij}}_{kl} \brac{\rho'\!\parenth{g\inv}}_{ls} \\
        &= \frac{1}{\abs{G}} \sum_{g \in G} \brac{\rho(g)}_{ri} \brac{\rho'\!\parenth{g\inv}}_{js}
    \end{align*}
    \begin{enumerate}
        \item If $W \not\cong V$, we have $\hat{\phi_{ij}} = 0$. In particular,
        \begin{align*}
            \frac{1}{\abs{G}} \sum_{g \in G} \brac{\rho(g)}_{ri} \brac{\rho'\!\parenth{g\inv}}_{js} = \brac{\hat{\phi_{ij}}}_{rs} = 0
        \end{align*}

        \item Similarly, if $W \cong V$, we have $m = n$. Now, by Theorem \ref{SP:Thm:Schur_fin_G_over_C}, we know that
        \begin{align*}
            \hat{\phi_{ij}} = \frac{1}{n} \Tr{\phi_{ij}} \cdot \id_V = \frac{1}{n} \delta_{ij} \cdot \id_V
        \end{align*}
        We then have that
        \begin{align*}
            \brac{\rho(g)}_{ri} \brac{\rho'\!\parenth{g\inv}}_{js} = \brac{\hat{\phi_{ij}}}_{rs} = \begin{cases}
                \frac{1}{n} & \text{if } i = j \text{ and } r = s \\
                0 & \text{otherwise}
            \end{cases}
        \end{align*}
    \end{enumerate}
\end{proof}
\begin{remark}
    As per Dr. Rizzoli, on the exam, it's more important to know the idea of such a proof than the specifics of \textit{which index goes where}.
\end{remark}

It turns out that by simply unfolding a few definitions and playing with the order of summations, we can use the above lemma to prove the Orthogonality Theorem.

\begin{boxtheorem}[Orthogonality Theorem] \label{Ch2:Thm:Orth_Char}
    Let $S, T$ be irreducible $\C G$-modules.
    \begin{enumerate}[label = \normalfont \arabic*., noitemsep]
        \item If $S \not\cong T$, then $\cycl{\chi_S, \chi_T} = 0$.
        \item If $S \cong T$, then $\cycl{\chi_S, \chi_T} = 1$
    \end{enumerate}
    In other words, irreducible characters form an orthogonal system.
\end{boxtheorem}
\begin{proof}
    Let $P : G \to \GL{n, \C}$ and $Q : G \to \GL{m, \C}$ be the representations corresponding to $S$ and $T$. We know that
    \begin{align*}
        \cycl{\chi_S, \chi_T} &=
        \frac{1}{\abs{G}} \sum_{G \in G} \chi_S(g) \chi_T\!\parenth{g\inv} \\
        &= \frac{1}{\abs{G}} \sum_{g \in G} \Tr{P(g)} \Tr{Q\!\parenth{g\inv}} \\
        &= \frac{1}{\abs{G}} \sum_{g \in G} \parenth{\sum_{i = 1}^{n} \brac{P(g)}_{ii}}\parenth{\sum_{j=1}^{n} \brac{Q\!\parenth{g\inv}}_{jj}} \\
        &= \sum_{i=1}^{n} \sum_{j=1}^{m} \frac{1}{\abs{G}} \sum_{g \in G} \brac{P(g)}_{ii} \brac{Q\!\parenth{g\inv}}_{jj}
    \end{align*}
    We can now use Lemma \ref{Ch2:Lem:Pre_Ortho_Thm_Sums} to evaluate this sum.
    \begin{enumerate}
        \item If $S \not\cong T$, then
        \begin{align*}
            \frac{1}{\abs{G}} \sum_{g \in G} \brac{P(g)}_{ii} \brac{Q\!\parenth{g\inv}}_{jj} = 0
        \end{align*}
        for all $1 \leq i \leq n$ and $1 \leq j \leq m$. Hence,
        \begin{align*}
            \cycl{\chi_S, \chi_T} = \sum_{i=1}^{n} \sum_{j=1}^{m} 0 = 0
        \end{align*}

        \item If $S \cong T$, then first of all, $n = m$. Indeed,
        \begin{align*}
            \frac{1}{\abs{G}} \sum_{g \in G} \brac{P(g)}_{ii} \brac{Q\!\parenth{g\inv}}_{jj} =
            \begin{cases}
                \frac{1}{n} & \text{ if } i = j \\
                0 & \text{ otherwise}
            \end{cases}
        \end{align*}
        for all $1 \leq i, j \leq n$. Hence,
        \begin{align*}
            \cycl{\chi_S, \chi_T} = \sum_{i=1}^{n} \sum_{j=1}^{n} \frac{1}{n} \delta_{ij} = \sum_{i=1}^{n} \frac{1}{n} = 1
        \end{align*}
    \end{enumerate}
\end{proof}

We have the following important corollary.

\begin{corollary}
    Up to isomorphism, there are finitely many irreducible $\C G$-modules.
\end{corollary}
\begin{proof}
    Suppose there exist infinitely many, pairwise nonisomorphic \CGM s. Then, the (infinite) subset of $\FGC$ consisting of all of their characters would be pairwise orthogonal (by Theorem \ref{Ch2:Thm:Orth_Char}). Since orthogonality implies linear independence, this set would need to be contained in a basis of $\FGC$. However, by Proposition \ref{Ch2:Prop:FGC_FcGC_Fin_Dim}, such a basis would need to be finite, leading to a clear contradiction. Hence, there can only exist finitely many \CGM s (up to isomorphism).
\end{proof}

\subsection{Understanding Irreducible Characters}

In this subsection, we will explore the properties of irreducible characters in more detail. In particular, we will see that they form a basis for the space of class functions on $G$. This is a very important result, which illustrates the power of character theory in understanding the properties of a group. We will begin by introducing some notation and stating a few basic properties of irreducible characters that will be useful in the proof of the main result.

\begin{boxnotation}[Irreducible Characters]
    Denote by $\Irr{G}$ the subset of $\Fcof{G, \C}$ consisting of irreducible characters of $G$.
\end{boxnotation}

The Orthogonality Theorem tells us the following facts about Irreducible Characters.

\begin{boxproposition}[On the Behaviour of Irreducible Characters] \label{Ch2:Prop:Bhv_Irred_Char}
    \hfill
    \begin{enumerate}[label = \normalfont \arabic*., noitemsep]
        \item $\Irr{G}$ is a linearly independent set. In particular, $\abs{\Irr{G}} \leq \abs{G}$.
        
        \item Let $V = V_1 \+ \cdots \+ V_r$ be a $\C G$ module, with $V_i$ simple for $1 \leq i \leq r$. For any simple $\C G$ module $S$, the number of $V_i$s isomorphic to $S$, known as the \emph{multiplicity} of $S$, is given by $\cycl{\chi_V, \chi_S}$.
        
        \item Let $V, V'$ be simple $\C G$-modules. Then, $V \cong V' \iff \chi_V = \chi_{V'}$.
        
        \item A \CGM\ $V$ is simple iff $\cycl{\chi_V, \chi_V} = 1$.
    \end{enumerate}
\end{boxproposition}
\begin{proof}
    \hfill
    \begin{enumerate}[label = \normalfont \arabic*.]
        \item The linear independence follows immediately from the fact that $\Irr{G}$ form an orthonormal system. The inequality follows from the fact that all central functions are class functions: they agree for all conjugate elements of $G$. This means that $\pdim{\Fcof{G, \C}}$ is simply the number of conjugacy classes of $G$. Since $\pdim{\Fcof{G, \C}}$ must be at least $\abs{\Irr{G}}$ and the number of conjugacy classes of $G$ must be at most $\abs{G}$, we have the desired result.
        
        \item We know, from Proposition \ref{Ch2:Prop:Bhv_Char}, that $\chi_V = \chi_{V_1} + \cdots + \chi_{V_r}$. So,
        \begin{align*}
            \cycl{\chi_V, \chi_S} &= \cycl{\chi_{V_1} + \cdots + \chi_{V_r}, \chi_S} \\
            &= \sum_{i=1}^{r} \cycl{\chi_{V_i}, \chi_S}
        \end{align*}
        Observe that for $1 \leq i \leq r$, $\cycl{\chi_{V_i}, \chi_S} = 1$ if $V_i \cong S$ and $0$ otherwise. Hence, the only nonzero terms in the summation are given by those $i$ for which $V_i \cong S$. Since all of those terms are $1$, the sum must evaluate to the multiplicity of $S$.

        \item 
        \begin{description}
            \item[$\parenth{\implies}$] This is true regardless of the simplicity of $V$ and $V'$ (cf. Proposition \ref{Ch2:Prop:Bhv_Char}).
            \item[$\parenth{\impliedby}$] This turns out to be a straightforward application of the previous part. Let $W = V \+ V'$. Then, the multiplicity of $V$ is given by
            \begin{align*}
                \cycl{\chi_W, \chi_V} = \cycl{\chi_V + \chi_{V'}, \chi_V} = \cycl{\chi_V, \chi_V} + \cycl{\chi_{V'}, \chi_V} = 2\cycl{\chi_V, \chi_V} = 2 
            \end{align*}
            This means that there are two terms in the decomposition $W = V \+ V'$ that are isomorphic to $V$, which is only possible if $V \cong V'$.
        \end{description}

        \item 
        \begin{description}
            \item[$\parenth{\implies}$] This follows immediately from the Orthogonality Theorem.
            \item[$\parenth{\impliedby}$] Let $\Irr{G} = \set{\chi_1, \ldots, \chi_r}$, with corresponding simple \CGM s $\set{V_1, \ldots, V_r}$. Denoting by $a_i$ the multiplicity of each $V_i$ in $V$, we have that
            \begin{align*}
                V &= V_1^{\+ a_1} \+ \cdots \+ V_r^{\+ a_r}
            \end{align*}
            where we use the notation $V_i^{\+ a_i}$ to mean $\underbrace{V_i \+ \cdots \+ V_i}_{a_i \text{ times}}$, allowing $a_i = 0$.

            We then have
            \begin{align*}
                \chi_V &= \sum_{i=1}^{r} a_i \chi_i \\
                \implies \cycl{\chi_V, \chi_V} &= \sum_{i=1}^{r} a_i^2
            \end{align*}
            by the Orthogonality Theorem. Since $\cycl{\chi_V, \chi_V} = 1$, it must be that only one of the $a_i$s is nonzero, and equal to $1$. Hence, $V = V_i$ for this $i$, making $V$ simple.
        \end{description}
    \end{enumerate}
\end{proof}

We now have everything we need to prove the following important theorem.

\begin{boxtheorem}
    $\Irr{G}$ is a basis for $\Fcof{G, \C}$.
\end{boxtheorem}
\begin{proof}
    As $\Irr{G}$ is linearly independent, we only need to show that it spans $\FcGC$. To that end, let $W = \Span{\Irr{G}} \leq \Fcof{G, \C}$, with orthogonal complement $W^\perp$ with respect to the inner-product \eqref{SP:eq:Char_Inner_Product}. Since $V = W \+ W^\perp$, if we can show that $W^\perp = \set{0}$, we would have that $V = W$, proving the desired result.

    Fix $f \in W^\perp$, and consider the element $\hat{f} \in \C G$ given by  % CHANGE NOTATION!
    \begin{align*}
        \hat{f} &= \sum_{g \in G} \overline{f(g)} \cdot g
    \end{align*}

    First, we show that $\hat{f} \in \Zof{\C G}$---ie, that $\hat{f}$ commutes (multiplicatively) with all elements of $\C G$. To show this, it suffices to show that $h\inv \hat{f} h = \hat{f}$ for all $h \in G$ (ie, that $\hat{f}$ commutes with all elements of $G$). To that end, fix $h \in G$. Then,
    \begin{align*}
        h\inv \hat{f} h &=
        \sum_{g \in G} \overline{f(g)} \cdot h\inv g h \\
        &= \sum_{g \in G} \overline{\fof{h\inv g h}} \cdot h\inv g h \\ % Can take h\inv g h inside f because f is central. Mention somewhere.
        &= \hat{f}
    \end{align*}
    where the first equality follows from the fact that $\bar{f(g)} \in \C$, thereby commuting with all elements of $g$ in $\C G$; the second equality from the fact that $f$ is central (recall that $W, W^{\perp} \leq \FcGC$); and last equality from a change of variables in the summation, replacing $g$ with $hgh\inv$.

    Now, let $S$ be any simple \CGM. One can show that the map
    \begin{align*}
        \phi : S \to S : v \mapsto \hat{f} \cdot v
    \end{align*}
    is a \CGM\ homomorphism. % Include more details
    Now, in the notation of \eqref{SP:eq:Def_Hat_Map}, consider the map $\hat{\phi} : S \to S$. Since $\hat{f} \in \Zof{\C G}$, we have that for all $v \in V$,
    \begin{align*}
        \hat{\phi}(v) &= \frac{1}{\abs{G}} \sum_{g \in G} g \cdot \phiof{g\inv \cdot v} \\
        &= \frac{1}{\abs{G}} \sum_{g \in G} g\hat{f}g\inv \cdot v \\
        &= \frac{1}{\abs{G}} \sum_{g \in G} \hat{f} \cdot v \\
        &= \hat{f} \cdot v = \phi(v)
    \end{align*}
    proving that $\hat{\phi} = \phi$. Then, by Theorem \ref{SP:Thm:Schur_fin_G_over_C}, we have that % USE FACT THAT \hat{f} IS CENTRAL TO SHOW THAT \hat{\phi} = \phi (OR SOMETHING LIKE THAT).
    \begin{align*}
        \phi = \hat{\phi} = \frac{\Tr{\phi}}{\pdim{S}} \cdot \id
    \end{align*}
    Indeed,
    \begin{align*}
        \Tr{\phi} &= \pdim{S} \cdot \Tr{\sum_{g \in G} \overline{f(g)} \cdot g\vert_{S}} \\
        &= \sum_{g \in G} \overline{f(g)} \chi_S(g) \\
        &= \abs{G} \cycl{\underbrace{\chi_S}_{\in W}, \underbrace{f}_{\in W^\perp}} = 0
    \end{align*}
    proving that in fact, $\hat{f}\vert_S = 0$.
    
    \verb|sorry| % What an absolute mess this proof is. Terrible notation. TODO: FIX IT!!!!
\end{proof}