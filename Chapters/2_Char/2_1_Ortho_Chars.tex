\section{The Theory of Irreducible Characters}

In this section, we give an overview of the theory of irreducilbe characters. Our approach focuses extensively on the notion of orthogonality with respect to an inner product we shall soon define. An interesting result we will go on to prove is that irreducilbe characters play an important role in a broader class of functions on groups, telling us that representation theory can be used to study not only groups themselves but also functions thereof.

\subsection{On Central Functions}

As we all know, a group can contain several elements that all behave similarly---take, for instance, similar matrices in any general linear group. This is why we have the notion of \textit{conjugacy classes}, which allow us to study the elements of a group in terms of their actions or behaviours.

The purpose of character theory is to understand the structural and behavioural properties of a group, without focusing on syntactic particularities. This is why we define the notion of a \textit{class function}.

\begin{definition}[Class Function]
    Let $X$ be any set. A function $f : G \to X$ is said to be a class function if $\fx = \fof{g\inv x g}$ for all $x,g \in G$---in other words, if it is constant on all conjugacy classes of $G$.
\end{definition}

In this section, we will primarily be focusing on complex representations. Therefore, the following definition is useful.

\begin{boxdefinition}[Central Function]
    We say $f : G \to \C$ is central if it is a class function. In other words, central functions are precisely $\C$-valued class functions.
\end{boxdefinition}

We are already familiar with several examples of central functions.

\begin{boxexample}[Examples of Central Functions] \label{Ch2:Eg:Central_Funcs}
    The following are all central:
    \begin{enumerate}
        \item The order function $h \mapsto \ord{h} : H \to \N \subset \C$ for any group $H$
        \item The determinant function $A \mapsto \pdet{A} : \GL{n, \C} \to \C$ for any $n \in \N$
        \item The trace function $A \mapsto \Tr{A} : \GL{n, \C} \to \C$ for any $n \in \N$
    \end{enumerate}
\end{boxexample}

The last two examples are, in particular, compatible with representations of $G$ over $\C$. This will prove to be important when we define the character of a representation. Before doing so, however, we will need to outline the structure of the inner-product space of central functions on $G$. First, some notation.

\begin{boxnotation}
    We define
    \begin{enumerate}
        \item $\FGC := \set{f : G \to \C}$
        \item $\FcGC := \set{f \in \FGC : f \text{ is central}}$
        \item $\delta_g(x)$ to be the indicator function (for $g, x \in G$).
    \end{enumerate}
\end{boxnotation}

We have the following natural result.

\begin{proposition}[$\C$-Vector Space Structure of $\FGC$ and $\FcGC$] \hfill
    \begin{enumerate}[label = \normalfont \arabic*., noitemsep]
        \item $\FGC$ is a $\C$-vector space of dimension $\abs{G}$, with basis $\set{\delta_g : g \in G}$.
        \item $\FcGC$ is a subspace of $\FGC$ of dimension equal to the number of conjugacy classes of $G$, with basis $\set{\delta_g : g \text{ uniquely represents a conjugacy class of $G$}}$.
    \end{enumerate}
\end{proposition}
\begin{proof}
    The model we choose for this proof is that of $k^n$ being isomorphic to the set of functions from $\set{1, \ldots, n}$ to $k$ (for any field $k$). Under this model, the indicator functions on $\set{1, \ldots, n}$ correspond to the standard basis of $k^n$. % We do not argue too rigorously here, as the proofs of both points are rather straightforward.
    \begin{enumerate}
        \item Since $G$ is finite, this is immediate from the model described above.
        \item That $\FcGC$ is a subspace follows from the fact that the sum of two central functions is central, as is any scalar multiple of a central function. Furthermore, since central functions are uniquely determined by their values on each conjugacy class, $\pdim{\FcGC}$ is the number of conjugacy classes of $G$.
        
        I also offer a more formal argument here for the dimensionality, as I found it instructive to think about it this way. Let $C_1, \ldots, C_r$ be the conjugacy classes of $G$. Write $C_i = \set{g_{i_1}, g_{i_2}, \ldots, g_{i_{l_i}}}$. Let $w_i := \sum_{j=1}^{l_i} \delta_{g_{i_j}}$. It is easy to see that the $w_i$ are linearly independent: any linear combination of all the $w_i$ is, in particular, a linear combination of $\delta_{g}$ as $g$ ranges over $G$, with each $\delta_g$ appearing exactly once. Furthermore, the $w_i$ span $\FcGC$: their span is clearly contained in $\FcGC$, and every element of $\FcGC$ must be constant on any given conjugacy class, meaning that the coordinates of conjugate components must be the same.
    \end{enumerate}
\end{proof}

It turns out that there is also a natural inner-product on $\FGC$.

\begin{definition}
    Define $\cycl{\cdot, \cdot} : \FGC \times \FGC \to \C$ By
    \begin{align}
        \cycl{f_1, f_2} = \frac{1}{\abs{G}} \sum_{g \in G} f_1(g) \overline{f_2(g)}
    \end{align}
\end{definition}

It is easy enough to show that the $\cycl{\cdot, \cdot}$ is, indeed, an inner-product on $\FGC$.

\subsection{Introduction to Character Theory}

In this subsection, we introduce a central function that is compatible with the notion of a representation---namely, the \textit{character}. As we saw in Example \ref{Ch2:Eg:Central_Funcs}, both the trace and the determinant would be good options, but the convention is to define the character in terms of the trace. A heuristic reason is that this way, given characters associated to two representations, their sum will give the character of the direct sum, and their product that of the tensor product of the underlying representations. There are deeper reasons too, which will become clearer as we progress.

\begin{boxdefinition}[Character]
    Let $V$ be a $\C G$-module. The character of $V$ is the map $\chi_V : G \to \C$ given by $\chi_V(g) = \Tr{\rho(g)}$, where $\rho$ is the representation associated to $V$.
\end{boxdefinition}

Immediately, we are able to ``import'' the following definitions from Chapter \ref{Ch1:CH}.

\begin{definition}[Irreducibility]
    We say a character $\chi_V$ is irreducible if the associated representation $\Vp$ is irreducible over $\C$.
\end{definition}
\begin{definition}[Degree]
    We define the degree of a character to be that of its associated representation.
\end{definition}
\begin{definition}[Trivial Character]
    We define the trivial character to be that associated with the trivial representation.
\end{definition}

Characters have several important properties, which we list below. We will make extensive use of these properties for the remainder of this chapter.

\begin{proposition}%[Behaviour of Characters]
    Let $V$ be a $\C G$-modules, with associated representation $\rho$.
    \begin{enumerate}[label = \normalfont \arabic*., noitemsep]
        \item For all \CGM s $W$, $\chi_{V \+ W} = \chi_V + \chi_W$
        \item  For all \CGM s $W$, $V \cong W \implies \chi_V = \chi_W$
        \item $\pdim{V} = \chi_V(1)$
        \item For all $g \in G$, $\chi_V(g)$ is a sum of $d$th roots of unity, where $d = \ord{g}$.
        \item For all $g \in G$, $\abs{\chi_V(g)} \leq \pdim{V}$, with equality iff $g \in \pker{\rho}$.
        \item For all $g \in G$, $\chi_V\!\parenth{g\inv} = \overline{\chi_V(g)}$
    \end{enumerate}
\end{proposition}
\begin{proof}
    We just give sketches here, not complete proofs.
    \begin{enumerate}[noitemsep]
        \item This follows from the fact that the trace of a direct sum is the sum of the traces.
        \item This follows from the invariance of the trace under change of basis.
        \item This follows from the fact that the trace of the identity is the dimension of the space.
        \item For any $g \in G$ of order $d$, $\rho(g)$ is of order $d$. Hence, its minimal polynomial divides $X^d - 1 \in \C[X]$, which has distinct roots in $\C$. This means that the eigenvalues of $\rho(g)$ are $d$th roots of unity. Putting $\rho(g)$ in Jordan Normal Form, we see that its trace is a sum of $d$th roots of unity.\footnote{It is not necessarily a sum of \textit{all} $d$th roots of unity, or even \textit{distinct} $d$th roots of unity.}
        \item \verb|sorry|
        \item \verb|sorry|
    \end{enumerate}
\end{proof}

We now give a few examples of characters of representations.

\begin{example}[The Dihedral Group of Order $8$]
    Let $G = D_8$, the dihedral group of order $8$. Consider the presentation
    \begin{align*}
        G = \cycl{a, b \mid a^4 = b^2 = 1, b\inv a b = a\inv}
    \end{align*}
    Let $\rho : G \to \GL{2, \C}$ be a representation of $G$ over $\C$ given by
    \begin{align*}
        \rho(a) = \begin{bmatrix}
            0 & 1 \\ -1 & 0
        \end{bmatrix}
        \quad\quad \text{ and } \quad\quad
        \rho(b) = \begin{bmatrix}
            1 & 0 \\ 0 & -1
        \end{bmatrix}
    \end{align*}
    The associated character $\chi_V$ then takes on the following values:
    \begin{table}[H]
        \centering
        \begin{tabular}{|c|C{1.2cm} C{1.55cm} C{1.85cm} C{1.55cm} C{1.55cm} C{1.85cm} C{1.55cm} C{1.2cm}|}
            \hline
            $g$ & $1$ & $a$ & $a^2$ & $a^3$ & $b$ & $ab$ & $a^2 b$ & $a^3 b$ \\
            $\rho(g)$ &
            $\begin{bmatrix} 1 & 0 \\ 0 & 1 \end{bmatrix}$
            &
            $\begin{bmatrix} 0 & 1 \\ -1 & 0 \end{bmatrix}$
            &
            $\begin{bmatrix} -1 & 0 \\ 0 & -1 \end{bmatrix}$
            &
            $\begin{bmatrix} 0 & -1 \\ 1 & 0 \end{bmatrix}$
            &
            $\begin{bmatrix} 1 & 0 \\ 0 & -1 \end{bmatrix}$
            &
            $\begin{bmatrix} 0 & -1 \\ -1 & 0 \end{bmatrix}$
            &
            $\begin{bmatrix} -1 & 0 \\ 0 & 1 \end{bmatrix}$
            &
            $\begin{bmatrix} 0 & 1 \\ 1 & 0 \end{bmatrix}$
            \\
            $\chi_V(g)$ & $2$ & $0$ & $-2$ & $0$ & $0$ & $0$ & $0$ & $0$ \\
            \hline
        \end{tabular}
    \end{table}
\end{example}

\begin{example}[The Cyclic Group of Order $3$]
    Let $G = C_3 = \cycl{a}$ be the cyclic group of order $3$. Let $\rho_1, \rho_2, \rho_3$ be irreducible representations of $G$, with corresponding (irreducible) characters $\chi_1, \chi_2, \chi_3$.
\end{example}

\subsection{The Orthogonality Theorem}

\begin{lemma}
    Let $\rho : G \to \GL{n, \C}$ and $\rho' : G \to \GL{m, \C}$ be irreducible representations of $G$ over $\C$. Fix $j, s \in \set{1, \ldots, m}$ and $r, i \in \set{1, \ldots, n}$.
    \begin{enumerate}[label = \normalfont \arabic*., noitemsep]
        \item If $\rho$ and $\rho'$ are \textit{not} equivalent, then
        \begin{align*}
            \frac{1}{\abs{G}} \sum_{g \in G} {\rho(g)}_{ri} {\rho'\!\parenth{g\inv}}_{js} = 0
        \end{align*}

        \item If $\rho$ and $\rho'$ \textit{are} equivalent, then
        \begin{align*}
            \frac{1}{\abs{G}} \sum_{g \in G} {\rho(g)}_{ri} {\rho'\!\parenth{g\inv}}_{js} =
            \begin{cases}
                \frac{1}{n} & \text{if } i = j \text{ and } r = s \\
                0 & \text{otherwise}
            \end{cases}
        \end{align*}
    \end{enumerate}
    where we use the notation $T_{ij}$ to refer to the $ij$th entry of the matrix of $T$.
\end{lemma}
\begin{proof}
    Let $V = \C^n$ and $W = \C^m$ be the two simple $\C G$-modules corresponding to $\rho$ and $\rho'$ respectively. The idea is to define a linear map that will allow us to use Schur's Lemma.

    For some chosen bases of $V$ and $W$, let $\phi_{ij} : W \to V$ be the $\C$-linear map given by the $n \times m$ matrix with $ij$th entry $1$ and all other entries $0$. Define
    \begin{align*}  % TODO: Make hat look nice
        \hat{\phi_{ij}} := \frac{1}{\abs{G}} \sum_{g \in G} \rho(g) \circ \phi_{ij} \circ \rho'\!\parenth{g\inv}
    \end{align*}
    By Theorem \ref{SP:Thm:Schur_fin_G_over_C}, $\hat{\phi_{ij}}$ is a $\C G$-module homomorphism from $W$ to $V$.
    \begin{enumerate}
        \item If $W \not\cong V$, we have $\hat{\phi_{ij}} = 0$. In particular,
        \begin{align*}
            0 &= \parenth{
                \frac{1}{\abs{G}} \sum_{g \in G} \rho(g) \circ E_{ij} \circ \rho'\!\parenth{g\inv}
            }_{rs} \\
            &= \frac{1}{\abs{G}} \sum_{g \in G} \brac{\rho(g) \circ E_{ij} \circ \rho'\!\parenth{g\inv}}_{rs} \\
            &= \sum_{k=1}^{n} \sum_{l=1}^{m} \frac{1}{\abs{G}} \sum_{g \in G} \brac{\rho(g)}_{rk} \brac{E_{ij}}_{kl} \brac{\rho'\!\parenth{g\inv}}_{ls} \\
            &= \frac{1}{\abs{G}} \sum_{g \in G} \brac{\rho(g)}_{ri} \brac{\rho'\!\parenth{g\inv}}_{js}
        \end{align*}

        \item Similarly, if $W \cong V$, we can view $\phi_{ij}$ as being given by the $n \times n$ matrix $E_{ij}$. Now, by Theorem \ref{SP:Thm:Schur_fin_G_over_C}, we know that
        \begin{align*}
            \hat{\phi_{ij}} &= \frac{1}{n} \Tr{E_{ij}} \cdot \id_V \\
            &= \frac{1}{n} \delta_{ij} \cdot \id_V
        \end{align*}
        We can then show the desired result using a similar computation.
    \end{enumerate}
\end{proof}
\begin{remark}
    As per Dr. Rizzoli, on the exam, it's more important to know the idea of such a proof than the specifics of \textit{which index goes where}.
\end{remark}

\begin{boxtheorem}[Orthogonality Theorem] \label{Ch2:Thm:Orth_Char}
    Let $S, T$ be irreducible $\C G$-modules.
    \begin{enumerate}[label = \normalfont \arabic*., noitemsep]
        \item If $S \not\cong T$, then $\cycl{\chi_S, \chi_T} = 0$.
        \item If $S \cong T$, then $\cycl{\chi_S, \chi_T} = 1$
    \end{enumerate}
    In other words, irreducible characters form an orthogonal system.
\end{boxtheorem}
\begin{proof}
    Let $P : G \to \GL{n, \C}$ and $Q : G \to \GL{m, \C}$ be the representations corresponding to $S$ and $T$. We know that
    \begin{align*}
        \cycl{\chi_S, \chi_T} &=
        \frac{1}{\abs{G}} \sum_{G \in G} \chi_S(g) \chi_T\!\parenth{g\inv} \\
        &= \frac{1}{\abs{G}} \sum_{g \in G} \Tr{P(g)} \Tr(Q\!\parenth{g\inv}) \\
        &= \frac{1}{\abs{G}} \sum_{g \in G} \parenth{\sum_{i = 1}^{n} \brac{P(g)}_{ii}}\parenth{\sum_{j=1}^{n} \brac{Q\!\parenth{g\inv}}_{jj}} \\
        &= \sum_{i=1}^{n} \sum_{j=1}^{m} \frac{1}{\abs{G}} \sum_{g \in G} \brac{P(g)}_{ii} \brac{Q\!\parenth{g\inv}_{jj}}
    \end{align*}
    We can then use the previous lemma to evaluate these sums.
\end{proof}

We have the following important corollary.

\begin{corollary}
    Up to isomorphism, there are finitely many irreducible $\C G$-modules.
\end{corollary}

\subsection{Irreducible Characters}

\begin{boxnotation}[Irreducilbe Characters]
    Denote by $\Irr{G}$ the subset of $\Fcof{G, \C}$ consisting of irreducible characters of $G$.
\end{boxnotation}

The Orthogonality Theorem tells us the following facts about Irreducible Characters.

\begin{proposition}[On the Behaviour of Irreducible Characters] \label{Ch2:Prop:Bhv_Irred_Char}
    \hfill
    \begin{enumerate}[label = \normalfont \arabic*., noitemsep]
        \item $\Irr{G}$ is a linearly independent set. In particular, $\abs{\Irr{G}} \leq \abs{G}$.
        
        \item Let $V = V_1 \+ \cdots \+ V_r$ be a $\C G$ module, with $V_i$ simple for $1 \leq i \leq r$. For any simple $\C G$ module $S$, the number of $V_i$s isomorphic to $S$ is given by $\cycl{\chi_V, \chi_S}$.
        
        \item Let $V, V'$ be $\C G$-modules. Then, $V \cong V' \iff \chi_V = \chi_{V'}$.
        
        \item A \CGM\ $V$ is simple iff $\cycl{\chi_V, \chi_V} = 1$.
    \end{enumerate}
\end{proposition}
\begin{proof}
    \hfill
    \begin{enumerate}[label = \normalfont \arabic*.]
        \item The linear independence follows immediately from the fact that $\Irr{G}$ form an orthonormal system. The inequality follows from the fact that all central functions are class functions: they agree for all conjugate elements of $G$. This means that $\pdim{\Fcof{G, \C}}$ is simply the number of conjugacy classes of $G$. Since $\pdim{\Fcof{G, \C}}$ must be at least $\abs{\Irr{G}}$ and the number of conjugacy classes of $G$ must be at most $\abs{G}$, we have the desired result.
        
        \item We know that $\chi_V = \chi_{V_1} + \cdots + \chi_{V_r}$. So,
        \begin{align*}
            \cycl{\chi_V, \chi_S} &= \cycl{\chi_{V_1} + \cdots + \chi_{V_r}, V_S} \\
            &= \text{sorry}
        \end{align*}

        \item 
        \begin{description}
            \item[$\parenth{\implies}$] Already seen. % TODO: PUT REFERENCE!
            \item[$\parenth{\impliedby}$] % Let $\Irr{G} = \set{\chi_1, \ldots, \chi_r}$.
            The multiplicity\footnote{Ie, the number of times it appears in the direct sum decomposition of $V$ into simple $\C G$-modules} of a simple \CGM\ $S$ in $V$ is given by $\cycl{\chi_V, \chi_S} = \cycl{\chi_{V'}, \chi_S}$. Then, a simple \CGM\ must have the same multiplicity in both $V$ and $V'$. % TODO: REFINE!!
        \end{description}

        \item 
        \begin{description}
            \item[$\parenth{\implies}$] Already seen. % TODO: PUT REFERENCE!
            \item[$\parenth{\impliedby}$] Let $\Irr{G} = \set{\chi_1, \ldots, \chi_r}$, with corresponding simple \CGM s $\set{V_1, \ldots, V_r}$. Denoting by $a_i$ the multiplicity of each $V_i$ in $V$, we have that
            \begin{align*}
                V &= V_1^{\+ a_1} \+ \cdots \+ V_r^{\+ a_r}
            \end{align*}
            where we use the notation $V_i^{\+ a_i}$ to mean $\underbrace{V_i \+ \cdots \+ V_i}_{a_i \text{ times}}$.

            We then have
            \begin{align*}
                \chi_V &= \sum_{i=1}^{r} a_i \chi_i \\
                \implies \cycl{\chi_V, \chi_V} &= \sum_{i=1}^{r} a_i^2
            \end{align*}
            This means only one of the $a_i$s is nonzero, and equal to $1$. % WHAT????
        \end{description}
    \end{enumerate}
\end{proof}

We now have everything we need to prove the following important theorem.

\begin{theorem}
    $\Irr{G}$ is a basis for $\Fcof{G, \C}$.
\end{theorem}
\begin{proof}
    Let $W = \Span{\Irr{G}} \leq \Fcof{G, \C}$, with orthogonal complement $W^\perp$ with respect to the inner-product \verb|insert reference|. Since $V = W \+ W^\perp$, if we can show that $W^\perp = \set{0}$, we would have that $V = W$, proving the desired result.

    Fix $f \in W^\perp$, and consider the element $\hat{f} \in \C G$ given by  % CHANGE NOTATION!
    \begin{align*}
        \hat{f} &= \sum_{g \in G} \overline{f(g)} \cdot g
    \end{align*}

    First, we show that $\hat{f} \in \Zof{\C G}$---ie, that $\hat{f}$ commutes (multiplicatively) with all elements of $\C G$. To show this, it suffices to show that $h\inv \hat{f} h = \hat{f}$ for all $h \in G$. So, fix $h \in G$. Then,
    \begin{align*}
        h\inv \hat{f} h &=
        \sum_{g \in G} \overline{f(g)} \cdot h\inv g h \\
        &= \sum_{g \in G} \overline{\fof{h\inv g h}} \cdot h\inv g h \\ % Can take h\inv g h inside f because f is central. Mention somewhere.
        &= \hat{f}(g)
    \end{align*}
    where the last equality follows from a change of variables in the summation.

    Now, let $S$ be any simple \CGM. One can show that the map  % DOMAIN AND CODOMAIN????
    \begin{align*}
        \phi : S \to S : v \mapsto \hat{f} \cdot v
    \end{align*}
    is a \CGM\ homomorphism. % Include more details
    Then, by Theorem \ref{SP:Thm:Schur_fin_G_over_C}, we have that
    \begin{align*}
        \hat{f} &= \frac{1}{\pdim{S}} \Tr{\hat{f}\vert_S} \cdot \id
    \end{align*}
    We then have that
    \begin{align*}
        \Tr{\hat{f}\vert_S} &= \Tr{\sum_{g \in G} \overline{f(g)} \cdot g\vert_{S}} \\
        &= \sum_{g \in G} \overline{f(g)} \chi_S(g) \\
        &= \abs{G} \cycl{\underbrace{\chi_S}_{\in W}, \underbrace{f}_{\in W^\perp}} = 0
    \end{align*}
    proving that in fact, $\hat{f}\vert_S = 0$.
    
    \verb|sorry| % Use photo on phone to complete proof.
\end{proof}