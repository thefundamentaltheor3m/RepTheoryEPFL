\chapter{Character Theory}
\thispagestyle{empty}

\begin{wrapfigure}[2]{r}{0.45\linewidth}
    \vspace{-15em}
    \centering
    \includegraphics[width=0.8\linewidth]{./Chapters/2_Char/IMG_Char_Thy.jpg}
    \caption{\centering \small The thumbnail of a YouTube video titled ``What is Character Theory? $\vert$ Let's Talk Theory" by Dapper Mr. Tom. The video has nothing to do with mathematics.}
\end{wrapfigure}

In this chapter, we study an important type of functions from groups to fields known as characters. As we shall see, characters encode several useful properties of a group, and have been used extensively to prove several results about finite groups, (the representations of) which are the main object of study in this course. One of the reasons characters are useful to understand representation structures is that they are \textit{class functions}. That is, they encode information not about an individual element of a group but about its conjugacy class, making them good indicators of \textit{structural} and \textit{behavioural} properties. In particular, the character of a representation is independent of the choice of basis of the associated vector space.

\begin{boxnotation}
    Throughout this chapter, we denote by $G$ an arbitrary finite group.
\end{boxnotation}

% \section{Preliminaries}

\subsection{Important Definitions and Properties}

\begin{boxdefinition}[Character]
    Let $V$ be a $\C G$-module. The character $\chi_v$ is the function $\chi_V : G \to \C$ given by $\chi_v(g) = \Tr{\rho(g)}$, where $\rho$ is the representation associated to $V$.
\end{boxdefinition}

\begin{remark}
    Since the trace is independent of our choice of basis, the definition makes sense.
\end{remark}

\begin{definition}[Irreducibility]
    We say a character $\chi_V$ is irreducible if the associated representation $\Vp$ is irreducible over $\C$.
\end{definition}
\begin{definition}[Degree]
    We define the degree of a character to be that of its associated representation.
\end{definition}
\begin{definition}[Trivial Character]
    We define the trivial character to be that associated with the trivial representation.
\end{definition}

\begin{proposition}[Behaviour of Characters]
    Let $V$, $W$ be $\C G$-modules, and let $g \in G$ be arbitrary. Then,
    \begin{enumerate}[label = \normalfont \arabic*., noitemsep]
        \item $\chi_{V \+ W} = \chi_V + \chi_W$
        \item $V \cong W \implies \chi_V = \chi_W$
        \item $\pdim{V} = \chi_V(1)$
        \item $\chi_v(g)$ is a sum of $d$th roots of unity, where $d = \ord{g}$.
        \item $\abs{\chi_V(g)} \leq \pdim{V}$
        \item $\chi_V\!\parenth{g\inv} = \overline{\xi_V(g)}$
    \end{enumerate}
\end{proposition}
\begin{proof}
    We do not give complete proofs here, just sketches.
    \begin{enumerate}
        \item This follows from the fact that the trace of a direct sum is the sum of the traces.
        \item This follows from the invariance of the trace under change of basis.
    \end{enumerate}
    \verb|sorry|
\end{proof}

\subsection{Character Tables}

A character table is exactly what it sounds like: a table consisting of the elements of a group, their images in a representation, and their associated characters. In this subsection, we investigate character tables by going through specific examples.

\begin{example}[The Dihedral Group of Order $8$]
    Let $G = D_8$, the dihedral group of order $8$. Consider the presentation
    \begin{align*}
        G = \cycl{a, b \mid a^4 = b^2 = 1, b\inv a b = a\inv}
    \end{align*}
    Let $\rho : G \to \GL{2, \C}$ be a representation of $G$ over $\C$ given by
    \begin{align*}
        \rho(a) = \begin{bmatrix}
            0 & 1 \\ -1 & 0
        \end{bmatrix}
        \quad\quad \text{ and } \quad\quad
        \rho(b) = \begin{bmatrix}
            1 & 0 \\ 0 & -1
        \end{bmatrix}
    \end{align*}
    We then have the following character table for $\rho$:
    \begin{table}[H]
        \centering
        \begin{tabular}{|c|cccccccc|}
            \hline
            $g$ & $1$ & $a$ & $a^2$ & $a^3$ & $b$ & $ab$ & $a^2 b$ & $a^3 b$ \\
            $\rho(g)$ &
            $\begin{bmatrix} 1 & 0 \\ 0 & 1 \end{bmatrix}$
            &
            $\begin{bmatrix} 0 & 1 \\ -1 & 0 \end{bmatrix}$
            &
            $\begin{bmatrix} -1 & 0 \\ 0 & -1 \end{bmatrix}$
            &
            $\begin{bmatrix} 0 & -1 \\ 1 & 0 \end{bmatrix}$
            &
            $\begin{bmatrix} 1 & 0 \\ 0 & -1 \end{bmatrix}$
            &
            $\begin{bmatrix} 0 & -1 \\ -1 & 0 \end{bmatrix}$
            &
            $\begin{bmatrix} -1 & 0 \\ 0 & 1 \end{bmatrix}$
            &
            $\begin{bmatrix} 0 & 1 \\ 1 & 0 \end{bmatrix}$
            \\
            $\chi_V(g)$ & $2$ & $0$ & $-2$ & $0$ & $0$ & $0$ & $0$ & $0$ \\
            \hline
        \end{tabular}
    \end{table}
\end{example}

Let $G = C_3 = \cycl{a}$ be the cyclic group of order $3$. Let $\rho_1, \rho_2, \rho_3$ be irreducible representations of $G$, with corresponding (irreducible) characters $\chi_1, \chi_2, \chi_3$.

\section{The Theory of Irreducible Characters}

In this section, we give an overview of the theory of irreducilbe characters. Our approach focuses extensively on the notion of orthogonality with respect to an inner product we shall soon define. An interesting result we will go on to prove is that irreducilbe characters play an important role in a broader class of functions on groups, telling us that representation theory can be used to study not only groups themselves but also functions thereof.

\subsection{On Central Functions}

As we all know, a group can contain several elements that all behave similarly---take, for instance, similar matrices in any general linear group. This is why we have the notion of \textit{conjugacy classes}, which allow us to study the elements of a group in terms of their actions or behaviours.

The purpose of character theory is to understand the structural and behavioural properties of a group, without focusing on syntactic particularities. This is why we define the notion of a \textit{class function}.

\begin{definition}[Class Function]
    Let $X$ be any set. A function $f : G \to X$ is said to be a class function if $\fx = \fof{g\inv x g}$ for all $x,g \in G$---in other words, if it is constant on all conjugacy classes of $G$.
\end{definition}

In this section, we will primarily be focusing on complex representations. Therefore, the following definition is useful.

\begin{boxdefinition}[Central Function]
    We say $f : G \to \C$ is central if it is a class function. In other words, central functions are precisely $\C$-valued class functions.
\end{boxdefinition}

We are already familiar with several examples of central functions.

\begin{boxexample}[Examples of Central Functions] \label{Ch2:Eg:Central_Funcs}
    The following are all central:
    \begin{enumerate}
        \item The order function $h \mapsto \ord{h} : H \to \N \subset \C$ for any group $H$
        \item The determinant function $A \mapsto \pdet{A} : \GL{n, \C} \to \C$ for any $n \in \N$
        \item The trace function $A \mapsto \Tr{A} : \GL{n, \C} \to \C$ for any $n \in \N$
    \end{enumerate}
\end{boxexample}

The last two examples are, in particular, compatible with representations of $G$ over $\C$. This will prove to be important when we define the character of a representation. Before doing so, however, we will need to outline the structure of the inner-product space of central functions on $G$. First, some notation.

\begin{boxnotation}
    We define
    \begin{enumerate}
        \item $\FGC := \set{f : G \to \C}$
        \item $\FcGC := \set{f \in \FGC : f \text{ is central}}$
        \item $\delta_g(x)$ to be the indicator function (for $g, x \in G$).
    \end{enumerate}
\end{boxnotation}

We have the following natural result.

\begin{proposition}[$\C$-Vector Space Structure of $\FGC$ and $\FcGC$] \hfill
    \begin{enumerate}[label = \normalfont \arabic*., noitemsep]
        \item $\FGC$ is a $\C$-vector space of dimension $\abs{G}$, with basis $\set{\delta_g : g \in G}$.
        \item $\FcGC$ is a subspace of $\FGC$ of dimension equal to the number of conjugacy classes of $G$, with basis $\set{\delta_g : g \text{ uniquely represents a conjugacy class of $G$}}$.
    \end{enumerate}
\end{proposition}
\begin{proof}
    The model we choose for this proof is that of $k^n$ being isomorphic to the set of functions from $\set{1, \ldots, n}$ to $k$ (for any field $k$). Under this model, the indicator functions on $\set{1, \ldots, n}$ correspond to the standard basis of $k^n$. % We do not argue too rigorously here, as the proofs of both points are rather straightforward.
    \begin{enumerate}
        \item Since $G$ is finite, this is immediate from the model described above.
        \item That $\FcGC$ is a subspace follows from the fact that the sum of two central functions is central, as is any scalar multiple of a central function. Furthermore, since central functions are uniquely determined by their values on each conjugacy class, $\pdim{\FcGC}$ is the number of conjugacy classes of $G$.
        
        I also offer a more formal argument here for the dimensionality, as I found it instructive to think about it this way. Let $C_1, \ldots, C_r$ be the conjugacy classes of $G$. Write $C_i = \set{g_{i_1}, g_{i_2}, \ldots, g_{i_{l_i}}}$. Let $w_i := \sum_{j=1}^{l_i} \delta_{g_{i_j}}$. It is easy to see that the $w_i$ are linearly independent: any linear combination of all the $w_i$ is, in particular, a linear combination of $\delta_{g}$ as $g$ ranges over $G$, with each $\delta_g$ appearing exactly once. Furthermore, the $w_i$ span $\FcGC$: their span is clearly contained in $\FcGC$, and every element of $\FcGC$ must be constant on any given conjugacy class, meaning that the coordinates of conjugate components must be the same.
    \end{enumerate}
\end{proof}

It turns out that there is also a natural inner-product on $\FGC$.

\begin{definition}
    Define $\cycl{\cdot, \cdot} : \FGC \times \FGC \to \C$ By
    \begin{align}
        \cycl{f_1, f_2} = \frac{1}{\abs{G}} \sum_{g \in G} f_1(g) \overline{f_2(g)}
    \end{align}
\end{definition}

It is easy enough to show that the $\cycl{\cdot, \cdot}$ is, indeed, an inner-product on $\FGC$.

\subsection{Introduction to Character Theory}

In this subsection, we introduce a central function that is compatible with the notion of a representation---namely, the \textit{character}. As we saw in Example \ref{Ch2:Eg:Central_Funcs}, both the trace and the determinant would be good options, but the convention is to define the character in terms of the trace. A heuristic reason is that this way, given characters associated to two representations, their sum will give the character of the direct sum, and their product that of the tensor product of the underlying representations. There are deeper reasons too, which will become clearer as we progress.

\begin{boxdefinition}[Character]
    Let $V$ be a $\C G$-module. The character of $V$ is the map $\chi_V : G \to \C$ given by $\chi_V(g) = \Tr{\rho(g)}$, where $\rho$ is the representation associated to $V$.
\end{boxdefinition}

Immediately, we are able to ``import'' the following definitions from Chapter \ref{Ch1:CH}.

\begin{definition}[Irreducibility]
    We say a character $\chi_V$ is irreducible if the associated representation $\Vp$ is irreducible over $\C$.
\end{definition}
\begin{definition}[Degree]
    We define the degree of a character to be that of its associated representation.
\end{definition}
\begin{definition}[Trivial Character]
    We define the trivial character to be that associated with the trivial representation.
\end{definition}

Characters have several important properties, which we list below. We will make extensive use of these properties for the remainder of this chapter.

\begin{proposition}%[Behaviour of Characters]
    Let $V$ be a $\C G$-modules, with associated representation $\rho$.
    \begin{enumerate}[label = \normalfont \arabic*., noitemsep]
        \item For all \CGM s $W$, $\chi_{V \+ W} = \chi_V + \chi_W$
        \item  For all \CGM s $W$, $V \cong W \implies \chi_V = \chi_W$
        \item $\pdim{V} = \chi_V(1)$
        \item For all $g \in G$, $\chi_V(g)$ is a sum of $d$th roots of unity, where $d = \ord{g}$.
        \item For all $g \in G$, $\abs{\chi_V(g)} \leq \pdim{V}$, with equality iff $g \in \pker{\rho}$.
        \item For all $g \in G$, $\chi_V\!\parenth{g\inv} = \overline{\chi_V(g)}$
    \end{enumerate}
\end{proposition}
\begin{proof}
    We just give sketches here, not complete proofs.
    \begin{enumerate}[noitemsep]
        \item This follows from the fact that the trace of a direct sum is the sum of the traces.
        \item This follows from the invariance of the trace under change of basis.
        \item This follows from the fact that the trace of the identity is the dimension of the space.
        \item For any $g \in G$ of order $d$, $\rho(g)$ is of order $d$. Hence, its minimal polynomial divides $X^d - 1 \in \C[X]$, which has distinct roots in $\C$. This means that the eigenvalues of $\rho(g)$ are $d$th roots of unity. Putting $\rho(g)$ in Jordan Normal Form, we see that its trace is a sum of $d$th roots of unity.\footnote{It is not necessarily a sum of \textit{all} $d$th roots of unity, or even \textit{distinct} $d$th roots of unity.}
        \item \verb|sorry|
        \item \verb|sorry|
    \end{enumerate}
\end{proof}

We now give a few examples of characters of representations.

\begin{example}[The Dihedral Group of Order $8$]
    Let $G = D_8$, the dihedral group of order $8$. Consider the presentation
    \begin{align*}
        G = \cycl{a, b \mid a^4 = b^2 = 1, b\inv a b = a\inv}
    \end{align*}
    Let $\rho : G \to \GL{2, \C}$ be a representation of $G$ over $\C$ given by
    \begin{align*}
        \rho(a) = \begin{bmatrix}
            0 & 1 \\ -1 & 0
        \end{bmatrix}
        \quad\quad \text{ and } \quad\quad
        \rho(b) = \begin{bmatrix}
            1 & 0 \\ 0 & -1
        \end{bmatrix}
    \end{align*}
    The associated character $\chi_V$ then takes on the following values:
    \begin{table}[H]
        \centering
        \begin{tabular}{|c|C{1.2cm} C{1.55cm} C{1.85cm} C{1.55cm} C{1.55cm} C{1.85cm} C{1.55cm} C{1.2cm}|}
            \hline
            $g$ & $1$ & $a$ & $a^2$ & $a^3$ & $b$ & $ab$ & $a^2 b$ & $a^3 b$ \\
            $\rho(g)$ &
            $\begin{bmatrix} 1 & 0 \\ 0 & 1 \end{bmatrix}$
            &
            $\begin{bmatrix} 0 & 1 \\ -1 & 0 \end{bmatrix}$
            &
            $\begin{bmatrix} -1 & 0 \\ 0 & -1 \end{bmatrix}$
            &
            $\begin{bmatrix} 0 & -1 \\ 1 & 0 \end{bmatrix}$
            &
            $\begin{bmatrix} 1 & 0 \\ 0 & -1 \end{bmatrix}$
            &
            $\begin{bmatrix} 0 & -1 \\ -1 & 0 \end{bmatrix}$
            &
            $\begin{bmatrix} -1 & 0 \\ 0 & 1 \end{bmatrix}$
            &
            $\begin{bmatrix} 0 & 1 \\ 1 & 0 \end{bmatrix}$
            \\
            $\chi_V(g)$ & $2$ & $0$ & $-2$ & $0$ & $0$ & $0$ & $0$ & $0$ \\
            \hline
        \end{tabular}
    \end{table}
\end{example}

\begin{example}[The Cyclic Group of Order $3$]
    Let $G = C_3 = \cycl{a}$ be the cyclic group of order $3$. Let $\rho_1, \rho_2, \rho_3$ be irreducible representations of $G$, with corresponding (irreducible) characters $\chi_1, \chi_2, \chi_3$.
\end{example}

\subsection{The Orthogonality Theorem}

\begin{lemma}
    Let $\rho : G \to \GL{n, \C}$ and $\rho' : G \to \GL{m, \C}$ be irreducible representations of $G$ over $\C$. Fix $j, s \in \set{1, \ldots, m}$ and $r, i \in \set{1, \ldots, n}$.
    \begin{enumerate}[label = \normalfont \arabic*., noitemsep]
        \item If $\rho$ and $\rho'$ are \textit{not} equivalent, then
        \begin{align*}
            \frac{1}{\abs{G}} \sum_{g \in G} {\rho(g)}_{ri} {\rho'\!\parenth{g\inv}}_{js} = 0
        \end{align*}

        \item If $\rho$ and $\rho'$ \textit{are} equivalent, then
        \begin{align*}
            \frac{1}{\abs{G}} \sum_{g \in G} {\rho(g)}_{ri} {\rho'\!\parenth{g\inv}}_{js} =
            \begin{cases}
                \frac{1}{n} & \text{if } i = j \text{ and } r = s \\
                0 & \text{otherwise}
            \end{cases}
        \end{align*}
    \end{enumerate}
    where we use the notation $T_{ij}$ to refer to the $ij$th entry of the matrix of $T$.
\end{lemma}
\begin{proof}
    Let $V = \C^n$ and $W = \C^m$ be the two simple $\C G$-modules corresponding to $\rho$ and $\rho'$ respectively. The idea is to define a linear map that will allow us to use Schur's Lemma.

    For some chosen bases of $V$ and $W$, let $\phi_{ij} : W \to V$ be the $\C$-linear map given by the $n \times m$ matrix with $ij$th entry $1$ and all other entries $0$. Define
    \begin{align*}  % TODO: Make hat look nice
        \hat{\phi_{ij}} := \frac{1}{\abs{G}} \sum_{g \in G} \rho(g) \circ \phi_{ij} \circ \rho'\!\parenth{g\inv}
    \end{align*}
    By Theorem \ref{SP:Thm:Schur_fin_G_over_C}, $\hat{\phi_{ij}}$ is a $\C G$-module homomorphism from $W$ to $V$.
    \begin{enumerate}
        \item If $W \not\cong V$, we have $\hat{\phi_{ij}} = 0$. In particular,
        \begin{align*}
            0 &= \parenth{
                \frac{1}{\abs{G}} \sum_{g \in G} \rho(g) \circ E_{ij} \circ \rho'\!\parenth{g\inv}
            }_{rs} \\
            &= \frac{1}{\abs{G}} \sum_{g \in G} \brac{\rho(g) \circ E_{ij} \circ \rho'\!\parenth{g\inv}}_{rs} \\
            &= \sum_{k=1}^{n} \sum_{l=1}^{m} \frac{1}{\abs{G}} \sum_{g \in G} \brac{\rho(g)}_{rk} \brac{E_{ij}}_{kl} \brac{\rho'\!\parenth{g\inv}}_{ls} \\
            &= \frac{1}{\abs{G}} \sum_{g \in G} \brac{\rho(g)}_{ri} \brac{\rho'\!\parenth{g\inv}}_{js}
        \end{align*}

        \item Similarly, if $W \cong V$, we can view $\phi_{ij}$ as being given by the $n \times n$ matrix $E_{ij}$. Now, by Theorem \ref{SP:Thm:Schur_fin_G_over_C}, we know that
        \begin{align*}
            \hat{\phi_{ij}} &= \frac{1}{n} \Tr{E_{ij}} \cdot \id_V \\
            &= \frac{1}{n} \delta_{ij} \cdot \id_V
        \end{align*}
        We can then show the desired result using a similar computation.
    \end{enumerate}
\end{proof}
\begin{remark}
    As per Dr. Rizzoli, on the exam, it's more important to know the idea of such a proof than the specifics of \textit{which index goes where}.
\end{remark}

\begin{boxtheorem}[Orthogonality Theorem] \label{Ch2:Thm:Orth_Char}
    Let $S, T$ be irreducible $\C G$-modules.
    \begin{enumerate}[label = \normalfont \arabic*., noitemsep]
        \item If $S \not\cong T$, then $\cycl{\chi_S, \chi_T} = 0$.
        \item If $S \cong T$, then $\cycl{\chi_S, \chi_T} = 1$
    \end{enumerate}
    In other words, irreducible characters form an orthogonal system.
\end{boxtheorem}
\begin{proof}
    Let $P : G \to \GL{n, \C}$ and $Q : G \to \GL{m, \C}$ be the representations corresponding to $S$ and $T$. We know that
    \begin{align*}
        \cycl{\chi_S, \chi_T} &=
        \frac{1}{\abs{G}} \sum_{G \in G} \chi_S(g) \chi_T\!\parenth{g\inv} \\
        &= \frac{1}{\abs{G}} \sum_{g \in G} \Tr{P(g)} \Tr(Q\!\parenth{g\inv}) \\
        &= \frac{1}{\abs{G}} \sum_{g \in G} \parenth{\sum_{i = 1}^{n} \brac{P(g)}_{ii}}\parenth{\sum_{j=1}^{n} \brac{Q\!\parenth{g\inv}}_{jj}} \\
        &= \sum_{i=1}^{n} \sum_{j=1}^{m} \frac{1}{\abs{G}} \sum_{g \in G} \brac{P(g)}_{ii} \brac{Q\!\parenth{g\inv}_{jj}}
    \end{align*}
    We can then use the previous lemma to evaluate these sums.
\end{proof}

We have the following important corollary.

\begin{corollary}
    Up to isomorphism, there are finitely many irreducible $\C G$-modules.
\end{corollary}

\subsection{Irreducible Characters}

\begin{boxnotation}[Irreducilbe Characters]
    Denote by $\Irr{G}$ the subset of $\Fcof{G, \C}$ consisting of irreducible characters of $G$.
\end{boxnotation}

The Orthogonality Theorem tells us the following facts about Irreducible Characters.

\begin{proposition}[On the Behaviour of Irreducible Characters] \label{Ch2:Prop:Bhv_Irred_Char}
    \hfill
    \begin{enumerate}[label = \normalfont \arabic*., noitemsep]
        \item $\Irr{G}$ is a linearly independent set. In particular, $\abs{\Irr{G}} \leq \abs{G}$.
        
        \item Let $V = V_1 \+ \cdots \+ V_r$ be a $\C G$ module, with $V_i$ simple for $1 \leq i \leq r$. For any simple $\C G$ module $S$, the number of $V_i$s isomorphic to $S$ is given by $\cycl{\chi_V, \chi_S}$.
        
        \item Let $V, V'$ be $\C G$-modules. Then, $V \cong V' \iff \chi_V = \chi_{V'}$.
        
        \item A \CGM\ $V$ is simple iff $\cycl{\chi_V, \chi_V} = 1$.
    \end{enumerate}
\end{proposition}
\begin{proof}
    \hfill
    \begin{enumerate}[label = \normalfont \arabic*.]
        \item The linear independence follows immediately from the fact that $\Irr{G}$ form an orthonormal system. The inequality follows from the fact that all central functions are class functions: they agree for all conjugate elements of $G$. This means that $\pdim{\Fcof{G, \C}}$ is simply the number of conjugacy classes of $G$. Since $\pdim{\Fcof{G, \C}}$ must be at least $\abs{\Irr{G}}$ and the number of conjugacy classes of $G$ must be at most $\abs{G}$, we have the desired result.
        
        \item We know that $\chi_V = \chi_{V_1} + \cdots + \chi_{V_r}$. So,
        \begin{align*}
            \cycl{\chi_V, \chi_S} &= \cycl{\chi_{V_1} + \cdots + \chi_{V_r}, V_S} \\
            &= \text{sorry}
        \end{align*}

        \item 
        \begin{description}
            \item[$\parenth{\implies}$] Already seen. % TODO: PUT REFERENCE!
            \item[$\parenth{\impliedby}$] % Let $\Irr{G} = \set{\chi_1, \ldots, \chi_r}$.
            The multiplicity\footnote{Ie, the number of times it appears in the direct sum decomposition of $V$ into simple $\C G$-modules} of a simple \CGM\ $S$ in $V$ is given by $\cycl{\chi_V, \chi_S} = \cycl{\chi_{V'}, \chi_S}$. Then, a simple \CGM\ must have the same multiplicity in both $V$ and $V'$. % TODO: REFINE!!
        \end{description}

        \item 
        \begin{description}
            \item[$\parenth{\implies}$] Already seen. % TODO: PUT REFERENCE!
            \item[$\parenth{\impliedby}$] Let $\Irr{G} = \set{\chi_1, \ldots, \chi_r}$, with corresponding simple \CGM s $\set{V_1, \ldots, V_r}$. Denoting by $a_i$ the multiplicity of each $V_i$ in $V$, we have that
            \begin{align*}
                V &= V_1^{\+ a_1} \+ \cdots \+ V_r^{\+ a_r}
            \end{align*}
            where we use the notation $V_i^{\+ a_i}$ to mean $\underbrace{V_i \+ \cdots \+ V_i}_{a_i \text{ times}}$.

            We then have
            \begin{align*}
                \chi_V &= \sum_{i=1}^{r} a_i \chi_i \\
                \implies \cycl{\chi_V, \chi_V} &= \sum_{i=1}^{r} a_i^2
            \end{align*}
            This means only one of the $a_i$s is nonzero, and equal to $1$. % WHAT????
        \end{description}
    \end{enumerate}
\end{proof}

We now have everything we need to prove the following important theorem.

\begin{theorem}
    $\Irr{G}$ is a basis for $\Fcof{G, \C}$.
\end{theorem}
\begin{proof}
    Let $W = \Span{\Irr{G}} \leq \Fcof{G, \C}$, with orthogonal complement $W^\perp$ with respect to the inner-product \verb|insert reference|. Since $V = W \+ W^\perp$, if we can show that $W^\perp = \set{0}$, we would have that $V = W$, proving the desired result.

    Fix $f \in W^\perp$, and consider the element $\hat{f} \in \C G$ given by  % CHANGE NOTATION!
    \begin{align*}
        \hat{f} &= \sum_{g \in G} \overline{f(g)} \cdot g
    \end{align*}

    First, we show that $\hat{f} \in \Zof{\C G}$---ie, that $\hat{f}$ commutes (multiplicatively) with all elements of $\C G$. To show this, it suffices to show that $h\inv \hat{f} h = \hat{f}$ for all $h \in G$. So, fix $h \in G$. Then,
    \begin{align*}
        h\inv \hat{f} h &=
        \sum_{g \in G} \overline{f(g)} \cdot h\inv g h \\
        &= \sum_{g \in G} \overline{\fof{h\inv g h}} \cdot h\inv g h \\ % Can take h\inv g h inside f because f is central. Mention somewhere.
        &= \hat{f}(g)
    \end{align*}
    where the last equality follows from a change of variables in the summation.

    Now, let $S$ be any simple \CGM. One can show that the map  % DOMAIN AND CODOMAIN????
    \begin{align*}
        \phi : S \to S : v \mapsto \hat{f} \cdot v
    \end{align*}
    is a \CGM\ homomorphism. % Include more details
    Then, by Theorem \ref{SP:Thm:Schur_fin_G_over_C}, we have that
    \begin{align*}
        \hat{f} &= \frac{1}{\pdim{S}} \Tr{\hat{f}\vert_S} \cdot \id
    \end{align*}
    We then have that
    \begin{align*}
        \Tr{\hat{f}\vert_S} &= \Tr{\sum_{g \in G} \overline{f(g)} \cdot g\vert_{S}} \\
        &= \sum_{g \in G} \overline{f(g)} \chi_S(g) \\
        &= \abs{G} \cycl{\underbrace{\chi_S}_{\in W}, \underbrace{f}_{\in W^\perp}} = 0
    \end{align*}
    proving that in fact, $\hat{f}\vert_S = 0$.
    
    \verb|sorry| % Use photo on phone to complete proof.
\end{proof}
\section{Character Tables}

In Example~\ref{Ch2:Eg:Cycl_3_Char}, we constructed a table consisting of the character values of each conjugacy class of the cyclic group of order $3$ (in this case, each class consisted of only one element---but this is not always the case). Our approach in Example~\ref{Ch2:Eg:Cycl_3_Char} was direct computation, which, while straightforward, is not always practical---especially when dealing larger groups. In this section, we will introduce a more systematic way of constructing such tables using the orthogonality properties of irreducible characters.

\subsection{The Orthogonality Relations}

For the purposes of this subsection, let $C_1, \ldots, C_k$ be the conjugacy classes of $G$ with representatives $g_1, \ldots, g_k$ respectively. Recall that by the Orbit-Stabiliser Theorem,
\begin{align*}
    \abs{C_i} &= \frac{\abs{G}}{\abs{C_G(g_i)}}
\end{align*}
where $C_G(\cdot)$ refers to the centraliser\footnote{Recall that the centraliser of a group element is the set of all elements that commute with it.} of an element in $G$. Now, let $\Irr{G} = \set{\chi_1, \ldots, \chi_k}$.

\begin{boxdefinition}[Character Table]
    A character table is a $k \times k$ table whose columns are the conjugacy classes (or their representatives) and whose rows are the irreducible characters.
    \begin{table}[H]
        \centering
        \begin{tabular}{c|ccc}
            & $g_1$ & $\cdots$ & $g_k$ \\
            \hline
            $\chi_1$ & $\chiof{1}{g_1}$ & $\cdots$ & $\chiof{1}{g_k}$ \\
            $\vdots$ & $\vdots$ & $\ddots$ & $\vdots$ \\
            $\chi_k$ & $\chiof{k}{g_1}$ & $\cdots$ & $\chiof{k}{g_k}$
        \end{tabular}
    \end{table}
    In other words, it is a table whose $\parenth{i,j}$th entry is $\chi_i(g_j)$.
    % Maybe draw this...
\end{boxdefinition}

The table in Example~\ref{Ch2:Eg:Cycl_3_Char} is an example of a character table.

It turns out that we have a certain orthogonality property as we go across the columns of a character table.

\begin{boxproposition}[First Orthogonality Relation]\label{SP:1st_Orth_Rel}
    Given $1 \leq r, s \leq k$, we have
    \begin{align*}
        \frac{1}{\abs{G}} \sum_{j=1}^{k} \abs{C_j} \chi_r\!\parenth{g_j} \chi_s\!\parenth{g_j\inv} &= \delta_{rs}
    \end{align*}
\end{boxproposition}
\begin{proof}
    The proof is quite straightforward. It merely involves applying the centrality of characters to the definition of the inner-product and concluding using the Orthogonality Theorem:
    \begin{align*}
        \delta_{rs} &=
        \cycl{\chi_r, \chi_s} \\
        &= \frac{1}{\abs{G}} \sum_{g \in G} \chi_r(g) \chi_s\!\parenth{g\inv} \\
        &= \frac{1}{\abs{G}} \sum_{j=1}^{k} \parenth{\sum_{g \in C_i} \chi_r(g) \chi_s\!\parenth{g\inv}} \\
        &= \frac{1}{\abs{G}} \sum_{j=1}^{k} \abs{C_j} \chi_r\!\parenth{g_j} \chi_s\!\parenth{g_j\inv}
    \end{align*}
    where the last inequality follows from the fact that characters are central (ie, they are equal for all elements of a conjugacy class).
\end{proof}
\begin{remark}
    Note that the result of Proposition~\ref{SP:1st_Orth_Rel} can be equivalently phrased in the following manner: given $1 \leq r, s \leq k$,
    \begin{align}
        \sum_{j=1}^{k} \frac{1}{\abs{C_G(g_j)}} \chi_r\!\parenth{g_j} \chi_s\!\parenth{g_j\inv} &= \delta_{rs}
        \label{SP:Eq:1st_Orth_Rel_Conj_Cl}
    \end{align}
\end{remark}

\begin{comment}
We can use this to prove an important result about the regular representation.

\begin{theorem}\label{Ch2:Thm:Reg_Rep_Col_1}
    Let $\chi_{\reg}$ denote the character of the regular representation of $G$ over $\C$. Then,
    \begin{align*}
        \chi_{\reg} &= \sum_{j=1}^{k} \chiof{j}{1} \chi_j
    \end{align*}
\end{theorem}
\begin{proof}
    
    \verb|sorry|  % From phone
\end{proof}

\begin{corollary} \label{Ch2:Cor:Reg_Rep_Col_1_Grp_Size}
    $\sum_{i=1}^{k} \parenth{\chiof{i}{1}}^2 = \abs{G}$.
\end{corollary}
\begin{proof}
    We evaluate both sides of the expression in Theorem~\ref{Ch2:Thm:Reg_Rep_Col_1} at $1$ to obtain that
    \begin{align*}
        \chiof{\reg}{1} = \sum_{i=1}^{k} \parenth{\chi_i(1)}^2 = \abs{G}        
    \end{align*}
\end{proof}
\end{comment}

Indeed, we have a similar result about going down the rows.

\begin{boxproposition}[Second Orthogonality Relation]\label{SP:2nd_Orth_Rel}
    Given $1 \leq r, s \leq k$, we have
    \begin{align*}
        \sum_{i=1}^{k} \chiof{i}{g_r} \chiof{i}{g_s\inv} =
        \begin{cases}
            0 &\text{ if } r \neq s \\
            \abs{C_G(g_r)} &\text{ if } r = s
        \end{cases}
    \end{align*}
\end{boxproposition}
\begin{proof}
    Let $A$ be the matrix $\parenth{A_{ij}}_{ij}$, where $A_{ij} := \chiof{i}{g_j}$. Similarly, let $B$ be the matrix $\parenth{B_{ij}}_{ij}$, where $B_{ij} := \frac{\abs{C_i}}{\abs{G}} \chiof{j}{g_i\inv}$. Then, writing $AB = \parenth{AB}_{pq}$, we have
    \begin{align*}
        \parenth{AB}_{pq} &=
        \sum_{l=1}^{k} A_{pl} B_{lq} \\
        &= \sum_{l=1}^{k} \chiof{p}{g_l} \frac{\abs{C_l}}{\abs{G}} \chiof{q}{g_l\inv} \\
        &= \cycl{\chi_p, \chi_q} = \delta_{pq}
    \end{align*}
    This proves that in fact, $AB = I$, the identity matrix. This, in particular, means that $BA = I$ as well. It turns out that setting $\delta_{pq} = \parenth{BA}_{pq}$ gives us the desired result. % TODO: COMPLETE!!!!!!!!!
\end{proof}
\begin{remark}
    Note that the result of Proposition~\ref{SP:2nd_Orth_Rel} can be equivalently phrased in the following manner: given $1 \leq r, s \leq k$,
    \begin{align}
        \sum_{j=1}^{k} \frac{1}{\abs{C_G(g_r)}} \chiof{i}{g_r} \chiof{i}{g_s\inv} &= \delta_{rs}
        \label{SP:Eq:2nd_Orth_Rel_Conj_Cl}
    \end{align}
    \eqref{SP:Eq:2nd_Orth_Rel_Conj_Cl} bears a striking similarity to \eqref{SP:Eq:1st_Orth_Rel_Conj_Cl}. The only difference is that in \eqref{SP:Eq:1st_Orth_Rel_Conj_Cl}, we fix the character and sum over the conjugacy classes (ie, fix the row and sum across the columns), while in \eqref{SP:Eq:2nd_Orth_Rel_Conj_Cl}, we fix the conjugacy class and sum over the characters (ie, fix the column and sum across the rows).
\end{remark}

In the next subsection, we will demonstrate how we can use these orthogonality relations, in combination with other results---most notably, \eqref{Ch2:Eq:Grp_Order_Sum_Char_Sq}---to construct character tables.

\subsection{Character Tables of Symmetric Groups}

\begin{boxexample}[$S_3$]
    $S_3$ has precisely three conjugacy classes $C_1$, $C_2$ and $C_3$ with representatives $g_1 = 1$, $g_2 = \parenth{1 2}$ and $g_3 = \parenth{1 2 3}$. We then have the following character table for $S_3$:
    \begin{table}[H]
        \centering
        \begin{tabular}{c|ccc}
            & $1$ & $\parenth{12}$ & $\parenth{123}$ \\
            \hline
            $\chi_1$ & $1$ & $1$ & $1$ \\
            $\chi_2$ & $1$ & $-1$ & $1$ \\
            $\chi_3$ & $2$ & $0$ & $-1$
        \end{tabular}
    \end{table}
    Here, $\chi_1$ corresponds to the trivial representation (of degree $1$), $\chi_2$ to the sign representation (of degree $1$), and $\chi_3$ to the subrepresentation $W = \Span{e_1 - e_2, e_1 - e_3}$ (of degree $2$) of the permutation representation on $\C^3$. Note that all of these representations are irreducible, the first two because their degrees are $1$, and the third because we can show $\cycl{\chi_W, \chi_W}$ to be equal to $1$ (cf. the fourth point of Proposition \ref{Ch2:Prop:Bhv_Irred_Char}).
\end{boxexample}


\begin{boxexample}[$S_4$]
    Observe that the conjugacy classes $C_i$, $i = 1, \ldots, 5$, of $S_4$ correspond precisely to the various possible cycle shapes. They therefore have representatives $g_1 = 1$, $g_2 = \parenth{12}$, $g_3 = \parenth{123}$, $g_4 = \parenth{12}\parenth{34}$, and $g_5 = \parenth{1234}$. The conjugacy classes have sizes $1$, $6$, $8$, $3$ and $6$ respectively.

    We then have the following character table for $S_4$:
    \begin{table}[H]
        \centering
        \begin{tabular}{c|ccccc}
            & $1$ & $\parenth{12}$ & $\parenth{123}$ & $(12)(34)$ & $(1234)$ \\
            \hline
            $\chi_1$ & $1$ & $1$ & $1$ & $1$ & $1$ \\
            $\chi_2$ & $1$ & $-1$ & $1$ & $1$ & $-1$ \\
            $\chi_3$ & $3$ & $1$ & $0$ & $-1$ & $-1$ \\
            $\chi_4$ & $2$ & $0$ & $-1$ & $2$ & $0$ \\
            $\chi_5$ & $a$ & $b$ & $c$ & $d$ & $e$
        \end{tabular}
    \end{table}
    where the characters $\chi_i$ are as follows:
    \begin{enumerate}
        \item $\chi_1$ is the \underline{trivial character}.
        
        \item $\chi_2$ is the \underline{sign character}\footnote{ie, that corresponding to the sign representation}.
        
        \item $\chi_3$ is the \underline{deleted permutation character} given by the subrepresentation of the permutation representation\footnote{ie, that corresponding to its action on $\C^4$ by permuting the standard basis} of $S_4$ corresponding to $\Span{e_1 - e_2, e_2 - e_3, e_3 - e_4}$, an $S_4$-invariant subspace of $\C^4$.
        
        \item $\chi_4$ is given by the following construction. Consider
        \begin{align*}
            N = \set{1, (12)(34), (13)(24), (14)(23)} \nsg S_4
        \end{align*}
        One can show that $\quotient{S_4}{N} \cong S_3$. Let $\pi : S_4 \surj S_3$ be the associated quotient map. Then, if $\rho : S_3 \to \GL{2, \C}$ is a $2$-dimensional irreducible representation of $S_3$, then the map $\rho' := \rho \circ \pi : S_4 \to \GL{2, \C}$ is also irreducible. We take $\chi_4$ to be the associated character.

        \item $\chi_5$ is the \underline{regular character}. We can actually solve for $a,b,c,d,e$ using the Orthogonality Relations. From \eqref{Ch2:Eq:Grp_Order_Sum_Char_Sq}, we know that
        \begin{align*}
            1 + 1 + 9 + 4 + a^2 = 24 = \abs{S_4}
        \end{align*}
        from which we can conclude that $a = 3$. Then, using the First Orthonality Relation (Proposition \ref{SP:1st_Orth_Rel}), we can see that
        \begin{align*}
            1 - 1 + 3 + 0 + ab = 0
        \end{align*}
        from which we can conclude that $b = -1$.

        \verb|sorry|
    \end{enumerate}
\end{boxexample}

Note that the number of irreducible characters of degree $1$ of any group $G$ is $\abs{G} / \abs{G'}$, where $G'$ is the derived subgroup of $G$. Indeed, there is a one-to-one correspondence between degree $1$ irreducible characters of $G$ and those of $\quotient{G}{G'}$.

It turns out that we can glean even more information about groups from their characters, for which we will need the properties of the algebraic integers.
\section{Integrality}

\subsection{The Algebraic Integers}

First, we recall the notion of integrality over an integral domain.

\begin{boxdefinition}[Integrality]
    Let $R$ be an integral domain, and $S \supset R$ an extension. We say that $s \in S$ is integral over $R$ if one of the following equivalent conditions is satisfied:
    \begin{itemize}
        \item $s$ is a root of a monic polynomial in $R[X]$.
        \item The minimal polynomial of $s$ over $\Frac{R}$ is actually in $R[X]$.
    \end{itemize}
\end{boxdefinition}
We now define what it means for a complex number to be an algebraic integer.
\begin{boxdefinition}[Algebraic Integer]
    We say that a number $\alpha \in \C$ is an algebraic integer if it is integral over $\Z$.
\end{boxdefinition}

\begin{boxnotation}
    Given a conjugacy class $C$ of $G$, we define
    \begin{align}
        \hat{C} := \sum_{g \in C} g
    \end{align}
    to be an element of $\C G$.
\end{boxnotation}

\begin{lemma}
    Let $g \in G$ and let $C = g^G$ % Conjugacy class of g?
    Let $S$ be a simple \CGM. Then, for all $s \in S$, we have an action by scalar multiplication
    \begin{align*}
        \hat{C} \cdot s = \lambda s
    \end{align*}
    where $\displaystyle \lambda = \frac{\abs{C}}{\abs{C_G(g)}} \frac{\chi(g)}{\chi_s(1)} = \abs{C} \frac{\chi_s(g)}{\chi_s(1)}$.
\end{lemma}
\begin{proof}
    Define a function $\phi : S \to S : s \mapsto \hat{C} \cdot s$. Since $\hat{C} \in \Zof{\C G}$, % EXERCISE!
    we have that $\forall x \in G$, $x \cdot \phi(s) = \phi(x \cdot s)$. This makes $\phi$ a \CGM\ homomorphism. Then, by Schur's Lemmas, we know that $\phi = \lambda \id$. This means that... % finish
\end{proof}

\begin{lemma}
    Let $r = \sum_{g \in G} \alpha_g g$ for some $\alpha_g \in \Z$. Suppose that $\exists \lambda, v \in \C G \setminus \set{0}$ such that $rv = \lambda v$. Then, $\lambda$ is an algebraic integer.
\end{lemma}
\begin{proof}
    Let $G = \set{g_1, \ldots, g_n}$. For all $1 \leq i \leq n$,
    \begin{align*}
        r g_i &= \sum_{j=1}^{n} \alpha_{ij} g_j
    \end{align*}
    The key observation here is that $\alpha_{ij} \in \Z$ for all $1 \leq i,j \leq n$. % WHY????
    Then, if $rv = \lambda v$, we have that $\lambda$ is an eigenvalue of the matrix $A := \parenth{\alpha_{ij}}_{1 \leq i,j \leq n} \in \mat{n}{n}{\Z}$. This makes $\lambda$ a root of the characteristic polynomial of $A$ (over $\Z$), which is monic and of degree $n$.
\end{proof}

What we can gather from lemmas $1$ and $2$ is that for any $\chi \in \Irr{G}$ and $g \in G$, the quantity $\lambda = \frac{\abs{G}}{\abs{C_G(g)}} \frac{\chi(g)}{\chi(1)}$ is an algebraic integer. This leads us to the following important connection between algebraic integers and irreducible representations.

\begin{theorem} \label{Ch2:Thm:Char_div_Ord_Grp}
    For all $\chi \in \Irr{G}$, $\chi(1) \vert \abs{G}$.
\end{theorem}
\begin{proof}
    \verb|Check phone|
\end{proof}

\subsection{The $n$th Roots of Unity}

\begin{lemma} \label{Ch3:Lem:Int_Sum_Coprime_Roots_Unity}
    Let $w$ be an $n$th root of unity in $\C$. Then,
    \begin{align*}
        \sum_{1 \leq i \leq m, \parenth{i,n} = 1} w^i \in \Z
    \end{align*}
\end{lemma}
\begin{proof}
    EXERCISE. Hint: induction on $n$.
\end{proof}

\begin{proposition}
    Let $g \in G$ be of order $n$. Suppose that $g^i$ is conjugate to $g$ for all $1 \leq i \leq n$ such that $\parenth{i,n} = 1$. Then, for all characters $\chi$ of $G$, $\chi(g)$ is an integer.
\end{proposition}
\begin{proof}
    Let $\chi$ be a character of $G$, with associated representation $\Vp$ of degree $m$ over $\C$. We know that $\chi(g) = \sum_{i=1}^{m} w_i$, where each $w_i$ is an $n$th root of unity (cf. Proposition \ref{Ch2:Prop:Bhv_Char}). Indeed, the idempotent linear map $\rho(g)$ is diagonalisable, with matrix $\diag{w_1, \ldots, w_m}$ with respect to an appropriate basis. Hence, we have that $\rhoof{g^i} = \diag{w_1^i, \ldots, w_m^i}$, meaning that $\chi\!\parenth{g^i} = \sum_{j=1}^{m} w_j^i$. But this is nothing but $\sum_{1 \leq i \leq m, \parenth{i,n} = 1} w^i$, by assumption. Hence, $\chi(g)$ is an integer, by Lemma \ref{Ch3:Lem:Int_Sum_Coprime_Roots_Unity}.
\end{proof}

% \subsection{Congruence Conditions}

\begin{lemma} \label{Ch2:Lem:p_and_p'_parts}
    Fix $g \in G$, and let $p$ be a prime number. Then, $\exists! x, y \in G$ such that
    \begin{enumerate}[label= \normalfont \arabic*., noitemsep]
        \item $g = xy = yx$
        \item $\ord{x} = p^k$ for some $k \in \N$
        \item $\parenth{\ord{y}, p} = 1$
    \end{enumerate}
\end{lemma}
\begin{proof}
    Let $\ord{g} = u p^v$, where $\parenth{u,v = 1}$. By Bézout's Lemma, we know that $\exists a, b \in \Z$ such that $au + bp^v = 1$. % WHY?
    Then, we can define $x = g^{au}$ and $y = g^{bp^v}$. This satisfies all the conditions:
    \begin{enumerate}
        \item $g = xy = g^{au} g^{bp^v} = g^{bp^v} g^{au} = yx$.
        \item $x^{p^v} = 1$. % SO WHAT?
        \item $y^v = 1$.    % WHY?
    \end{enumerate}
    One can also show $x$ and $y$ to be the only elements of $G$ satisfying these conditions. %% SHOW!!!! 
\end{proof}

\begin{boxdefinition}[The $p$- $p'$-parts of $g$]
    For any $g \in G$, call $x$ and $y$ from Lemma \ref{Ch2:Lem:p_and_p'_parts} the $p$- and $p'$-parts of $g$ respectively.
\end{boxdefinition}

For the remainder of this subsection, let $n$ = $\abs{G}$, and let $\zeta = e^{\frac{2\pi i}{n}}$ be a primitive $n$th root of unity. Define $\Z[\zeta]$ to be the subring of $\C$ generated by $\Z$ and $\zeta$. Let $p$ be a prime number and $p\Z[\zeta]$ be the principal ideal of $\Z[\zeta]$ generated by $p$. By the Correspondence Theorem, the ideals of $\Z[\zeta]$ containing $p\Z[\zeta]$ are in bijection with the ideals of $\quotient{\Z[\zeta]}{p\Z[\zeta]}$. Since the latter quantity is finite, it contains finitely many ideals, meaning that only finitely many of the ideals of $\Z[\zeta]$ contain $p\Z[\zeta]$. We can then look at the maximal (proper) ideal amongst these, which is a maximal ideal of $\Z[\zeta]$. Denote it by $P$. Indeed, we can show that $P \cap \Z = p\Z$.

\begin{theorem}
    Let $g \in G$, and let $y$ be the $p'$-part of $g$ for some $p$ prime. For all characters $\chi$ of $G$, $\chi(g) - \chi(y)$ lies in the maximal ideal $P$.
\end{theorem}
\begin{proof}
    Let $m = \ord{g} = u p^v$. Let $a,b \in \Z$ be such that $au + b p^v = 1$ (as in the proof of Lemma \ref{Ch2:Lem:p_and_p'_parts}). We have that $y = g^{bp^v}$. We know that $\chi(g), \chi(y) \in \Z[\zeta]$.

    Now, let $w$ be an $m$th root of unity. Since $m \vert n$ (by Lagrange's Theorem), we know that $w \in \Z[\zeta]$. We have that $w = w^{au + bp^v}$. Raising both sides to the $p^v$th power, we get that $w^{p^v} = w^{aup^v} w^{bp^{2v}}$. Since $w^{up^v} = w^m = 1$, we have that $w^{p^v} = w^{bp^{2v}}$.

    Now, consider the binomial expansion of $\parenth{w - w^{bp^v}}^{p^v}$. Some blah blah, use fact that $P$ is a prime ideal (since it is maximal). % FINISH
\end{proof}

This theorem has several important consequences.

\begin{corollary}
    Let $g \in G$ and $y$ be the $p'$-part of $g$ for some prime $p$. Let $\chi$ be a character of $G$.
    \begin{enumerate}[label = \normalfont \arabic*., noitemsep]
        \item If $\chi(g), \chi(y) \in \Z$, then $\chi(g) \equiv \chi(y) \pmod{p}$.
        \item If $\ord{g} = p^k$ for some $k \in \N$, then $\chi(g) \equiv \chi(1) \pmod{p}$.
    \end{enumerate}
\end{corollary}